\documentclass[12pt]{article}
\usepackage{fancyhdr}
\usepackage{graphicx}
\usepackage{amsmath, amscd, amssymb}
\usepackage[pdf,all,2cell]{xy}\SelectTips{cm}{10}\SilentMatrices\UseAllTwocells
\usepackage{enumerate}
\usepackage[bookmarksnumbered=true]{hyperref}
% lscape.sty Produce landscape pages in a (mainly) portrait document.
\usepackage{lscape}
\input symbolindex.tex
%
\textwidth = 6.5 in
\textheight = 9 in
\oddsidemargin = 0.0 in
\evensidemargin = 0.0 in
\topmargin = 0.0 in
\headheight = 0.0 in
\headsep = 0.0 in
\parindent = 0.0in

%\renewcommand{\thesubsection}{\arabic{subsection}}
%
\numberwithin{equation}{section}
%
\newenvironment{eq}{\begin{equation}}{\end{equation}}
%
\newenvironment{myproof}{{\bf Proof}:}{$\blacksquare$ \vskip 5mm }
%
\newtheorem{proposition}{Proposition}[subsection]
\newtheorem{lemma}[proposition]{Lemma}
\newtheorem{definition}[proposition]{Definition}
\newtheorem{theorem}[proposition]{Theorem}
\newtheorem{cor}[proposition]{Corollary}
\newtheorem{remark}[proposition]{Remark}
\newtheorem{cond}[proposition]{Conditions}
%
\newtheorem{problem}[proposition]{Problem}
%
\newtheorem{construction0}[proposition]{Construction}
%
\newenvironment{construction}[1]{\begin{construction0}(for Problem \ref{#1})\ }{$\blacksquare$ \end{construction0}}
\newcommand{\by}[1]{\text{(by #1)}}
\newcommand{\comment}[1]{}
\newcommand{\sr}{\rightarrow}
\newcommand{\lr}{\longrightarrow}
\newcommand{\nn}{{\mathbb N\rm}}
\newcommand{\NN}{{\mathbb N}}
\newcommand{\uu}{\underline}
\newcommand{\iHom}{\uu{Hom}}
\newcommand{\wt}{\widetilde}
\newcommand{\BB}{{\bullet}}
\newcommand{\dd}{\diamond}
\newcommand{\spc}{{\,\,\,\,\,\,\,}}
\newcommand{\toCC}{CC} % for the construction CC(R,LM) of a C-system from a left module over a monad
\newcommand{\CC}{{\mathbb C}}  % for a variable denoting a C-system
\newcommand{\C}{{\mathcal C}}  % for a category

% notations for structures making up a category
\newcommand{\id}{1}            % identity arrow

% notations for the structures making up a C-system
\newcommand{\ft}{\mathsf{ft}}
\newcommand{\p}{\mathsf{p}}
\newcommand{\q}{\mathsf{q}}
\newcommand{\s}{\mathsf{s}}     % making a section from a lifting

% notations for the structures making up a J-system
\newcommand{\Id}{\mathsf{Id}} % originally was "IdT"
\newcommand{\Idx}{\mathsf{Id}_3} % originally was "IdxT"

\newcommand{\refl}{\mathsf{refl}}
\newcommand{\J}{\mathsf{J}}

% notation for universe objects in a category
\newcommand{\U}{\mathcal{U}}

% other notation
\newcommand{\D}{\mathsf{D}}
\newcommand{\I}{\mathsf{I}}
\newcommand{\rf}{\mathsf{rf}}
\newcommand{\Fp}{\mathsf{Fp}}
\newcommand{\pFp}{\mathsf{pFp}}
\newcommand{\Q}{\mathsf{Q}}
\newcommand{\etashriek}{\eta^!}
\newcommand{\etaunshriek}{{\etashriek}^{-1}}
\newcommand{\Ob}{Ob}
\newcommand{\Obwt}{\wt{Ob}}
\newcommand{\st}{\mathsf{st}}
\newcommand{\pr}{\mathsf{pr}}
\newcommand{\prI}{\mathsf{prI}}
\newcommand{\adj}{\mathsf{adj}}

% persistent notations to go into symbol index
%   at definition time, use \gls{foo} to generate a link to the symbol index
%   at subsequent uses, use \foo or the actual code
%   the order here is the order in the index, and is obtained with "sort -f"
\newoperator{adj}{\adj}
\newoperator{CCCp}{\toCC({\C},p)}
\newoperator{coJ}{coJ}
\newoperator{DpV}{\D_p(-,V)}
\newoperator{DfXV}{\D^f(X,V)}
\newoperator{EUtilde}{E\wt{\U}}
\newoperator{fstarX}{f^*(X)}
\newoperator{fstarXi}{f^*(X,i)}
\newoperator{fstars}{f^*(s)}
\newoperator{fstarsi}{f^*(s,i)}
\newoperator{Fstarf}{F^*(f)}
\newoperator{Fp}{\Fp}
\newoperator{ft}{\ft}
\newoperator{HPhi}{H({\bf\Phi})}
\newoperator{Idx}{\Idx}
\newoperator{IdEq}{\Id_{Eq}}
\newoperator{Id}{\Id}
\newoperator{IdGamma}{\Id_\Gamma}
\newoperator{Ip}{\I_p}
\newoperator{IhV}{\I^h(V)}
\newoperator{int}{int}
\newoperator{J}{\J}
\newoperator{l}{l}
\newoperator{ObwtD}{\Obwt(\Delta)}
\newoperator{Obn}{\Ob_n(\Gamma)}
\newoperator{Obwtn}{\Obwt_n(\Gamma)}
\newoperator{omega}{\omega}
\newoperator{pGamman}{\p_{\Gamma,n}}
\newoperator{pX}{\p_X}
\newoperator{pXF}{\p_{X,F}}
\newoperator{pEXF}{\p^E_{X,F}}
\newoperator{pEUtilde}{pE\wt{\U}}
\newoperator{pr}{\pr}
\newoperator{prI}{\prI}
\newoperator{pFp}{\pFp}
\newoperator{qfX}{\q(f,X)}
\newoperator{qfXi}{\q(f,X,i)}
\newoperator{QF}{\Q(F)}
\newoperator{QfF}{\Q(f,F)}
\newoperator{QFE}{\Q(F)_{E}}
\newoperator{QfFE}{\Q(f,F)_{E}}
\newoperator{refl}{\refl}
\newoperator{reflOmega}{\refl_\Omega}
\newoperator{rf}{\rf}
\newoperator{rfT}{\rf_T}
\newoperator{RPhi}{R_{\bf\Phi}}
\newoperator{sg}{\s_g}
\newoperator{st}{\st}
\newoperator{XFE}{(X;F)_{E}}
\newoperator{semicolon}{(X;F)}
\newoperator{u1}{u_1}
\newoperator{u1Gamma}{u_{1,\Gamma}}
\newoperator{utilde1}{\wt u_1}
\newoperator{utilde1Gamma}{\wt u_{1,\Gamma}}
\newoperator{u2}{u_2}
\newoperator{wtu2}{\wt{u}_2}
\newoperator{multisemicolon}{(X;F_1,\dots,F_n)}
\newoperator{starfg}{f*g}
\newoperator{partial}{\partial}
\newoperator{parto}{\partial(o)}
\newoperator{delta}{\delta}
\newoperator{Delta}{\Delta}
\newoperator{zetai}{\zeta_i}
\newoperator{wtzetai}{\wt{\zeta}_i}
\newoperator{zetaPhi}{\zeta_{\bf\Phi}}
\newoperator{wtzetaPhi}{\wt{\zeta}_{\bf\Phi}}
\newoperator{etashriek}{\etashriek}
\newoperator{etaunshriek}{\etaunshriek}
\newoperator{iotaPhiXF}{\iota_{\Phi,X,F}}
\newoperator{xiPhi}{\xi_{{\bf\Phi}}}
\newoperator{wtxiPhi}{\wt\xi_{{\bf\Phi}}}
\newoperator{wtphiE}{\wt{\phi}_{E}}
\newoperator{phiGammaTPs0}{\phi(\Gamma,T,P,s0)}
\newoperator{PhiE}{{\bf\Phi}_E}
\newoperator{PhiXF}{\Phi_{X,F}}
\newoperator{Phi2}{{\bf\Phi}^2}
\newoperator{PhiwtUp}{\Phi\wt{\U}p}
\newoperator{chiPhiV}{\chi_{\bf\Phi}(V)}
\newoperator{psiGamma}{\psi_\Gamma}

%% \newoperator{}{\}

\comment{\pagestyle{fancy}

\renewcommand{\subsectionmark}[1]{\markright{\thesubsection \quad #1}}
\renewcommand{\sectionmark}[1]{}
\renewcommand{\chaptermark}[1]{}}



%%%%%%%%%%%%%%%%%%%%%%%%%%%%%%%%%%%%%%%%%%%%%%%%%%%%%%%%%%%%%%%%%%
\def\UseOption{draft}  % Toggle this line to show/hide todo-notes, table of contents, etc.
\usepackage[draft]{optional}
\usepackage[colorinlistoftodos,prependcaption,textsize=tiny]{todonotes}%to do list and comments

\newcommand{\plan}[1]{}
\newcommand{\BA}[1]{}
\newcommand{\DG}[1]{}
\opt{draft}{
   \renewcommand{\plan}[1]{\todo[color=blue!30]{Plan: #1}\PackageWarning{TODO}{Plan: #1}}
   \renewcommand{\BA}[1]{\todo[color=orange!30]{BA: #1} \PackageWarning{TODO}{BA: #1}}
   \renewcommand{\DG}[1]{\todo[color=green!30]{DG: #1}\PackageWarning{TODO}{DG: #1}}
}

\newcommand{\issue}[1]{\href{https://github.com/DanGrayson/VV-paths-C-systems-univ/issues/#1}{Issue #1}}

%%%%%%%%%%%%%%%%%%%%%%%%%%%%%%%%%%%%%%%%%%%%%%%%%%%%%%%%%%%%%%%%%%






\begin{document}
%
\parskip = 2mm
\begin{center}
{\bf\Large Martin-L\"of identity types in C-systems\footnote{\em 2000 Mathematical Subject Classification: 
18D99, % Categories with structures, none of the above
03B15, % higher-order logic and type theory
18D15 % Closed categories (closed monoidal and Cartesian closed categories, etc.)
}}

\vspace{3mm}

{\large\bf Vladimir Voevodsky}\footnote{School of Mathematics, Institute for Advanced Study,
Princeton NJ, USA. e-mail: vladimir@ias.edu}$^,$\footnote{Work on this paper was supported by NSF grant 1100938.}
\end{center}
%
%
\begin{abstract}
This paper continues a series of papers that develop a new approach to syntax
and semantics of dependent type theories. Here we study the interpretation of
the rules of the identity types in the intensional Martin-L\"of type theories on
the C-systems that arise from universe categories.  In the first part of the
paper we develop constructions that produce interpretations of these rules from
certain structures on universe categories while in the second we study the
functoriality of these constructions with respect to functors of universe
categories. The results of the first part of the paper play a crucial role in
the construction of the univalent model of type theory in simplicial sets.
\end{abstract}

\vskip 4mm
%
\tableofcontents

%

\section{Introduction}

\begin{quote}
{\sl He that delivereth knowledge desireth to deliver it in such form as may be
soonest believed and not as may be easiest examined.}

``On the Impediments of Knowledge'', from Valerius Terminus by Francis Bacon. 
\end{quote}

The concept of a C-system in its present form was introduced in
\cite{Csubsystems}. The type of C-systems is constructively equivalent to
the type of contextual categories defined by Cartmell in \cite{Cartmell0,Cartmell1},
but the definition of a C-system is slightly different from
the Cartmell's foundational definition.

In the past decade or more, it has been a tradition in the study of type
theories to consider, as the main mathematical object associated with a type
theory, not a C-system but a category with families (see \cite{Dybjer}). As was
observed recently all of the constructions of \cite{Cfromauniverse,fromunivwithPiI,presheavesOb,fromunivwithPiII}
and of the present paper (but not of \cite{Csubsystems}
or \cite{Cofamodule}!) can be either used directly or reformulated in a
straightforward way to provide similar results for categories with
families. This modification will be discussed in a separate paper or papers.

In this introductory explanation we will distinguish between the syntactic and
semantic C-systems. By a syntactic C-system we will mean a C-system that is a
regular subquotient of a C-system of the form $\toCC(R,LM)$, where $R$ is a monad
on sets and $LM$ is a left module over $R$, see \cite{Cofamodule,Csubsystems}.
In particular, the C-systems of all of the various versions
of dependent type theory of Martin-L\"of ``genus" are syntactic type systems in
the sense of this definition.

By a semantic C-system we will mean a C-system whose underlying category is a
full subcategory in a category of ``mathematical'' nature such as the category
of sets or the category of sheaves of sets.

Usually one knows some good properties (e.g., consistency) of a given semantic
C-system and tries to prove similar good properties of a syntactic C-system by
constructing a functor from the syntactic one to the semantic one.

To construct such a functor one tries to show that the syntactic C-system is an
initial one among C-systems equipped with some collection of additional
operations and then to construct operations of the required form on the
semantic one. A pioneering example of this approach can be found in
\cite{Streicher}.

In this paper we investigate the set of three interconnected operations on
C-systems that, in the case of the syntactic C-systems, corresponds to the set
of inference rules that is known as the rules for identity types in intensional
Martin-L\"of type theories (first published in \cite{MLTT73})\footnote{There is
  also a simpler set of rules corresponding to the identity types in the
  extensional Martin-L\"of type theory (first published in
  \cite{MLTT79}). Cartmell, in his notion of a strong M-L structure
  \cite[p.3.36]{Cartmell0}, considers the set of rules for the extensional
  theory.}. Since the key ingredient of this structure is known in type theory
as the J-eliminator we call it the J-structure.


We do not use the ``sequent'' notation that is so widespread in the literature
on type theory for general C-systems --- we restrict its use only to examples where
we assume the C-system to be a syntactic one.

The reason for this restriction is that translating sequent-like notation
into the algebraic notation of C-systems or categories with families requires
considerable mastery of various conventions connected to the use of dependently
typed systems. An example of such a translation is the description of an object
$\Idx(T)$ corresponding to the sequent-like expression
$(\Gamma, x:T, y:T, e:\Id\, T\, x\, y\,;)$ in Construction \ref{2015.03.27.constr1}.

%% --- the referee said this paragraph is misleading, and I agree
%% Some of the difficulties that arise here can already be seen by considering the translation
%% of the sequent-like expression $(\Gamma, x:T, y:T;)$. Here the same letter $T$
%% is used to refer to objects of two different types - the first $T$ refers to an
%% element on $Ob_1(\Gamma)$ and the second $T$ refers to an element in
%% $Ob_1(T)$. It is ``understood'' that the second $T$ is the image of the first
%% $T$ under the map $\p_T^*:Ob_1(\Gamma)\sr Ob_1(T)$, but this understanding is a
%% part of a tradition and is not reflected in any mathematical statement that one
%% can refer to.

For a syntactic C-system $\CC$ we are allowed to use sequent notation, for the
following reason. First, since $\CC$ in this case is a subquotient of
$\toCC(R,LM)$ our notation needs only to provide a reference to elements of sets
associated with $\toCC(R,LM)$ itself.  There, the first $T$ refers to an element
of $LM(\{1,\dots,l\})$, where $l$ is the length of $\Gamma$ and $LM(X)$ is the
set of type expressions in the raw syntax with free variables from a set $X$
and the second $T$ refers to an element of $LM(\{1,\dots,l+1\})$ that is the
image of the first $T$ under the map
%
$$LM(\{1,\dots,l\})\sr LM(\{1,\dots,l+1\})$$
%
defined by the inclusion $\{1,\dots,l\}\subset \{1,\dots,l+1\}$. In this case
the map does not depend on $T$. We should distinguish between $\Id$ as a
structure on the C-system and the corresponding syntactic construction (because
they have different types). If we denote the syntactic ``identity types'' by
$\Id^s\, T\, t_1\, t_2$ then for the sequence
%
$$\Gamma, x:T, y:T, e:\Id^s\,T\,x\,y;$$
%
to define an element of $Ob(\toCC(LM,R))$, the expression $\Id^s\,T\,x\,y$ must
refer to an element of $LM(\{1,\dots,l+2\})$ and its form shows that we assume
that there is an operation
%
$$\Id^s:LM\times R\times R\sr LM$$
%
(a natural transformation of functors that is a linear morphism of left
$R$-modules) and $\Id^s\,T\,x\,y$ is the ``named variables'' notation for
$\Id^s_{{1,\dots,l+2}}(T,l+1,l+2)$.

We do not continue this explanation of how to construct J-structures on
syntactic C-systems. This will be done in a separate paper. Let us remark,
however, that constructing J-structures on syntactic C-systems is relatively
easy, and that the difficult questions about J-structures on such C-systems are
those related to the initiality properties of the resulting objects.

While constructing J-structures on syntactic C-systems is relatively
straightforward, constructing non-degenerate\footnote{See Remark
  \ref{2015.05.12.rem1} for the definition of \emph{degenerate} J-structure.} J-structures on semantic C-systems or categories with
families proves to be very difficult.

The first instance of such a construction, due to Martin Hofmann and Thomas
Streicher, appeared in \cite{Hofmann1}. It was done in the language of
categories with families and the underlying category there was the category of
groupoids.

The construction of Hofmann and Streicher was substantially extended and
generalized in the Ph.D.{} thesis of Michael Warren
\cite{WarrenThesisProsp,WarrenThesis} and his subsequent papers such as
\cite{Warreninfty}.

Further important advances were achieved in the work of Benno van den Berg and Richard Garner \cite{BergandGarner}.

\comment{However, the original expectation that it should be possible to
  construct C-systems or categories with families with J-structures
  corresponding to all Quillen closed model categories with sufficiently good
  properties have not been realized. In particular none of the previous methods
  provided a construction of a C-system whose underlying category is the
  category of simplicial sets and whose J-structure corresponds, in an
  appropriately defined sense, to the standard closed model structure on this
  category.

This goal is still not fully realized in this paper, since to achieve it one has
to construct a Kan fibration with certain properties and discussing such a
construction is outside of the scope of this paper. }

The two main results of the first part of this paper provide a new approach to the
construction of J-structures on semantic C-systems, an approach that can be
used to construct the J-structure on the C-system of the univalent model.

Construction \ref{2015.05.22.constr1} provides a simple way of extending a
J1-structure on a universe $p:\wt{\U}\sr \U$ in a category $\cal C$ to a full J-structure.

Construction \ref{2015.04.04.constr2} provides a method of constructing a
J-structure on the C-system $\toCC({\C},p)$ from a J-structure on $p$.

Combined, they provide a method of constructing a J-structure on
$\toCC({\cal C},p)$ from a J1-structure on $p$.

We also discuss two sets of conditions on a pair of classes of morphisms $TC$
and $FB$ in a locally cartesian closed category that can be used in combination
with Construction \ref{2015.05.22.constr1} to construct J-structures. These
conditions often hold for the classes of trivial cofibrations and fibrations in
model categories (or categories with weak factorization systems) providing a
way of constructing C-systems with J-structures starting from such categories.

In this paper we continue to use the diagrammatic order of writing composition
of morphisms, i.e., for $f:X\sr Y$ and $g:Y\sr Z$ the composition of $f$ and
$g$ is denoted by $f\circ g$.

In this paper, as in the preceding papers \cite{Cfromauniverse,fromunivwithPiI,presheavesOb,fromunivwithPiII},
we often have to consider structures on categories that
are not invariant under equivalences and their interaction with structures that
are invariant under equivalence.

The methods used in this paper are fully constructive and the paper is written
in ``formalization ready'' style, with all the proofs provided in enough detail
to ensure that there are no hidden difficulties for the formalization of all of
the results presented here.

Except for the section that discusses the use of classes $TC$ and $FB$, the
methods we use are also completely elementary in the sense that they rely only
on the quasi-algebraic language of categories with various structures.

The key Definition \ref{2015.03.27.def6} and its relation to the J-structures
on categories $\toCC({\cal C},p)$ first appeared in \cite{CMUtalk}.

The author would like to thank the Department of Computer Science and
Engineering of the University of Gothenburg and Chalmers University of
Technology for its hospitality during the work on this paper.


\subsection{A note from the editor}

\newcommand{\editorfootnote}[1]{\footnote{Note from the editor: #1}}

After the death of Vladimir Voevodsky in September, 2017, Daniel Grayson was
appointed as the {\em academic executor}, in order to help arrange for the
publication of his works.  This paper had been submitted for publication,
comments had been received twice from the referee, and Voevodsky had started
rewriting the paper.
Grayson has edited the paper, following the advice of the referee, and has made
additional changes.  Where anything substantively mathematical is
involved, a footnote from the editor has been inserted.\editorfootnote{... such
  as this one.}
In order to make it possible and convenient for readers to review Grayson's
changes, all existing versions of the paper have been entered into a {\em github}
repository, and editing was done in an incremental fashion, with {\em commit
  messages} describing the work done at each step.  The repository is visible
at
\hfill\break\centerline{\tiny{\tt \href{https://github.com/DanGrayson/VV-paths-Csystems-univ}{https://github.com/DanGrayson/VV-paths-Csystems-univ}%
  } .}
The paper is visible as an entry in the archival record of Voevodsky's work at
\hfill\break\centerline{\tiny{\tt \href{http://www.math.ias.edu/Voevodsky/voevodsky-publications\_abstracts.html\#1505.06446}{http://www.math.ias.edu/Voevodsky/voevodsky-publications\_abstracts.html\#1505.06446}%
  } ,}
and a link there will allow the reader to view all existing drafts of the paper.
If those links are ever broken, perhaps a search for the randomly chosen string
\hfill\break\centerline{{\tt 864338b2cec01ffb32fdaaa6bd8fafb803c070e7}}
will allow the files to be located.

The editor thanks Benedikt Ahrens for many useful contributions to the editing
process.

\section{J-structures on C-systems and on universe categories}

\begin{remark}
  We recall some notation for use with a C-system $\CC$ and record some identities for later reference.

  Recall the length function $\gls{l} : \Ob(\CC) \to \NN$, which is part of the data forming a C-system.

%%  We may let $\Ob$ denote the set $\Ob(\CC)$ of objects of $\CC$ when there is no danger of confusion.

  As in \cite[Def.~2.1 and Prop.~2.4]{Csubsystems}, except for the use of a sans-serif font, we use
  \begin{eq}
    \label{qfX-diag}
    \xymatrix@C=5pc{
      \gls{fstarX} \ar[r]^{\gls{qfX}} \ar[d]_{\p_{f^*(X)}}  & X \ar[d]^{\gls{pX}} \\
      Y \ar[r]_f & \gls{ft}(X)
    }
  \end{eq}%
  as notation for the parts of the pull-back square that is part of the
  structure of a C-system; here $l(X) > 0$.
  Further, if $s : \ft(X) \sr X$ is a section of $\p_X$, then $\gls{fstars}$ denotes
  the unique section of $\p_{f^*(X)}$ such that $f^*(X) \circ \q(f,X) = f \circ
  s$, as in the following diagram.
  \begin{eq}
    \xymatrix@C=5pc{
      f^*(X) \ar[r]^{\q(f,X)} \ar@<3pt>[d]^{\p_{f^*(X)}}  & X \ar@<3pt>[d]^{\p_X} \\
      Y \ar[r]_f \ar@<4pt>[u]^{f^*(s)} & \ft(X) \ar@<4pt>[u]^s
    }
  \end{eq}%

  As in \cite[Def.~2.3 and Prop.~2.4]{Csubsystems}, for a map $g : Y \sr X$
  such that $g \circ p_X = f$, one may regard
  $\gls{sg} : Y \sr f^*(X)$ as part of the structure of a C-system, or
  as the unique section of $p_{f^*(X)}$ such that $\s_g \circ q(f,X) = g$.
  \begin{eq}
    \xymatrix@C=5pc{
      f^*(X) \ar[r]^{\q(f,X)} \ar@<-3pt>[d]_{\p_{f^*(X)}} & X \ar[d]^{\p_X} \\
      Y \ar[ru]_g \ar[r]_f \ar@<-4pt>[u]_{\s_g} & \ft(X)
    }
  \end{eq}%
  In the case where $Y = \ft(X)$ and $f = 1_X$, and thus $g$ is a section of $\p_X$,
  one has the following identity.
  \begin{eq}
    \label{section-section}
    \s_g = g
  \end{eq}%
  The formation of the section $\s_g$ is natural in the sense that, for a map
  $h : Z \to Y$, one has the following identity.
  \begin{eq}
    \label{s-naturality}
    h^* (\s_g) = \s_{h \circ g}
  \end{eq}%

  In the case where one takes $Y=X$, $f=p_X$, and $g = 1_X$, we let $\gls{delta}(X)$ denote
  the diagonal morphism $\delta(X) := \s_{\id_X} : X\sr \p_X^*(X)$.
  \begin{eq}
    \xymatrix@C=5pc{
      \p_X^*(X) \ar[r]^{\q(\p_X,X)} \ar@<-3pt>[d]_{\p_{\p_X^*(X)}} & X \ar[d]^{\p_X} \\
      X \ar[ru]_{1_X} \ar[r]_{\p_X} \ar@<-4pt>[u]_{\delta(X)} & \ft(X)
    }
  \end{eq}%
  Using \eqref{s-naturality}, one shows, for a map $g : Y \to X$, the following identity.
  \begin{eq}
    \label{delta-pullback}
    g^* (\delta(X)) = \s_g
  \end{eq}%

  Now we also recall the notations ${{Ob}}_n(\Gamma)$ and
  $Obwt_n(\Gamma)$ from \cite[\S 3]{fromunivwithPiI}, as well as the
  notation ${partial}(o)$ from \cite[\S 3]{Csubsystems}.

  Let $\Gamma$ be an object in $\CC$.  The set $\gls{Obn}$ is the set of
  objects $\Delta$ of $\CC$ such that $l(\Delta)-n = l(\Gamma)$ and
  $\ft^n(\Delta) = \Gamma$.

  For $n>0$, the set $\gls{Obwtn}$ is the set of morphisms of the form
  $o : \ft(\Delta) \sr \Delta$ for some $\Delta \in \Ob_n(\Gamma)$ that are sections
  of the corresponding canonical projection, i.e., where $o \circ p_\Delta = \id$;
  in that case $\gls{parto}$ denotes $\Delta$.

  Similarly, for an object $\Delta$ with $l(\Delta) > 0$, we let
  $\gls{ObwtD}$ denote the set of sections of $\p_\Delta$.

  Observe that $\s_g$ above is an element of $\Obwt_1(Y)$, and $\delta(X)$ above
  is an element of $\Obwt_1(X)$.
\end{remark}

\subsection{The notion of J-structure on a C-system}
%


To define the notion of ``J-structure'' on a C-system $\CC$ we will actually define three types of
structure: a J0-structure, a J1-structure over a J0-structure, and a J2-structure
over a J1-structure --- then a J-structure will be the same as a triple
$(\Id,\refl,\J)$, where $\Id$ is a J0-structure, $\refl$ is a J1-structure over
$\Id$, and $\J$ is a J2-structure over $\refl$.

\begin{definition}
\label{2015.03.27.def1}
A {\em J0-structure} on a C-system $\CC$ is a family of functions 
%
$$\gls{IdGamma}:\{(o_1,o_2)\,|\,o_1, o_2 \in \Obwt_1(\Gamma), \partial(o_1)=\partial(o_2)\}\sr Ob_1(\Gamma),$$

given for all $\Gamma\in Ob$, that is natural in $\Gamma$, i.e., such that for any
$f:\Gamma'\sr \Gamma$, one has the identity
\begin{eq}
  \label{Id-naturality}
  f^*(\Id_{\Gamma}(o_1,o_2))=\Id_{\Gamma'}(f^*(o_1),f^*(o_2)).
\end{eq}%
\end{definition}

When there is no risk of ambiguity, we will write $\Id(o_1,o_2)$ for $\Id_{\Gamma}(o_1,o_2)$.

\begin{definition}
\label{2015.03.27.def2}
Let $\Id$ be a J0-structure on $\CC$. A {\em J1-structure} over $\Id$ is a family of
functions
%
$$\gls{refl}:\Obwt_1(\Gamma)\sr \Obwt_1(\Gamma)$$
%
given for all $\Gamma\in Ob$ such that:
%
\begin{enumerate}
\item $\refl$ is natural in $\Gamma$,
\item for any $\Gamma$ and $o\in \Obwt_1(\Gamma)$ one has 
%
\begin{eq}
\label{2015.03.27.eq8}
\partial(\refl(o))=\Id_\Gamma(o,o)
\end{eq}%
\end{enumerate}
\end{definition}
%
To define the notion of a J2-structure over a given J1-structure we will need
to describe two constructions first.

%
\begin{problem}
\label{2015.03.27.prob1} Given a J0-structure $\Id$, to construct a family of
functions
%
$$\gls{Idx}:Ob_1(\Gamma)\sr Ob_3(\Gamma),$$
%
such that for $f:\Gamma'\sr \Gamma$ and $T\in Ob_1(\Gamma)$ one has
$f^*(\Idx(T))=\Idx(f^*(T))$.
\end{problem}
%
\begin{construction}{2015.03.27.prob1}\label{2015.03.27.constr1}\rm
The objects and some of the morphisms involved
in this construction can be seen in the following diagram, in which the
downward maps are canonical projections.
%
\begin{eq}
  \begin{xy}
    \xymatrix@C=5pc{
      \p_{\p_T^*(T)}^*(\p_T^*(T)) \ar[r] \ar[d]                             &
      \p_T^*(T) \ar[r]^{\q(\p_T,T)} \ar[d]_{\p_{\p_T^*(T)}}                     &
      T \ar[d]^{\p_T}                                                      \\
      \p_T^*(T) \ar[r]^{\p_{\p_T^*(T)}}
                \ar@<-6pt>[u]_{\delta(\p_T^*(T))} 
                \ar@<6pt> [u]^{\p_{\p_T^*(T)}^*(\delta(T))}
                                                                         &
      T \ar[r]^{\p_T} \ar@<-6pt>[u]_{\delta(T)}                             &
      \Gamma
    }
  \end{xy}
\end{eq}%

Since $\p_{\p_T^*(T)}^*(\delta(T))$ and $\delta(\p_T^*(T))$ are sections of the
same canonical projection, we may make the following definition.
\begin{eq}
\label{2015.04.06.eq1}
\Idx(T):=\Id_{\\p_T^*(T)}(\p_{\p_T^*(T)}^*(\delta(T)), \delta(\p_T^*(T)))
\end{eq}%
The fact that $\Idx(T)\in Ob_3(\Gamma)$ follows now from the fact that
$\ft^2(\p_T^*(T))=\ft(T)=\Gamma$.
The proof that $\Idx$ is natural in $f:\Gamma'\sr \Gamma$ is omitted.
\end{construction}

\begin{problem}
\label{2015.03.27.prob2} Given a J0-structure $\Id$ and a J1-structure $\refl$
over it, to construct for all $\Gamma\in Ob$ and $T\in Ob_1(\Gamma)$ a morphism
%
$$\gls{rfT} : T \sr \Idx(T)$$
%
over $\Gamma$,
such that for any $f:\Gamma'\sr \Gamma$ one has $f^*(\rf_T)=\rf_{f^*(T)}$.
\end{problem}
%
\begin{construction}{2015.03.27.prob2}
\label{2015.03.27.constr2}\rm We have the following chain of equations.
%
\begin{align*}
  \delta(T)^*(\Idx(T)) & = \delta(T)^*(\Id_{\p_T^*(T)}((\p_{\p_T^*(T)}^*(\delta(T)))),\delta(\p_T^*(T)))
  & \by{definition \ref{2015.04.06.eq1}} \\
  & = \Id_T(\delta(T)^*(\p_{\p_T^*(T)}^*(\delta(T))),\delta(T)^*(\delta(\p_T^*(T))))
  & \by{naturality \ref{Id-naturality}} \\
  & = \Id_T((\delta(T) \circ \p_{\p_T^*(T)})^*(\delta(T)),\delta(T)^*(\delta(\p_T^*(T))))
  & \by{def'n{.} of C-system} \\
  & = \Id_T((1_T)^*(\delta(T)),\delta(T)^*(\delta(\p_T^*(T)))) 
  & \by{def'n{.} of C-system} \\
  & = \Id_T(\delta(T),\delta(T)^*(\delta(\p_T^*(T)))) 
  & \by{def'n{.} of C-system} \\
  & = \Id_T(\delta(T),\s_{\delta(T)})
  & \by{\ref{delta-pullback}}  \\
  & = \Id_T(\delta(T),\delta(T))
  & \by{\ref{section-section}} 
\end{align*}
%
This shows that we have the canonical square in the following diagram.
%
\begin{eq}
\label{2015.03.31.eq3}
\begin{xy}
          \xymatrix@C=7pc{ \Id_\Gamma(\delta(T),\delta(T))
            \ar[r]^-{\q(\delta(T),\Idx(T))} \ar[d]^{\p_{\Id_\Gamma(\delta(T),\delta(T))}} & \Idx(T) \ar[d]\\ T
            \ar[r]^{\delta(T)} \ar@<6pt>[u]^{\refl(\delta(T))} \ar@{-->}[ru]^{\rf_T}
            & \p_T^*(T) }
       \end{xy}
\end{eq}%
%
Since $\refl(\delta(T))$ is a morphism $T\sr \Id(\delta(T),\delta(T))$
and is a section of the corresponding canonical projection,
we may introduce the following definition.
%
\begin{eq}
\label{2015.04.02.eq1} \gls{rfT} :=\refl(\delta(T))\circ \q(\delta(T),\Idx(T))
\end{eq}%
%
The proof that for any $f:\Gamma'\sr \Gamma$ one has $f^*(\rf_T)=\rf_{f^*(T)}$ is
omitted.
\end{construction}
%
\begin{definition}
\label{2015.03.27.def3} Let $\Id$ and $\refl$ be a J0-structure and a
J1-structure over it. A {\em J2-structure} over $(\Id,\refl)$ is data of the form: for
all $\Gamma\in Ob$, for all $T\in Ob_1(\Gamma)$, for all $P\in Ob_1(\Idx(T))$,
for all $s0\in \Obwt(\rf_T^*(P))$, an element $\gls J(\Gamma,T,P,s0)$ of
$\Obwt(P)$ such that:
%
\begin{enumerate}
\item $\J$ satisfies the $\iota$-rule. For $\Gamma, T, P, s0$ as above one has
%
$$\rf_T^*(\J(\Gamma,T,P,s0))=s0$$
%
\item $\J$ is natural in $\Gamma$, i.e., for any $f:\Gamma'\sr \Gamma$ and
  $T,P,s0$ as above one has
%
$$f^*(\J(\Gamma,T,P,s0))=\J(\Gamma',f^*(T),f^*(P),f^*(s0)),$$
%
where the right hand side of the equation is well-defined because of the
naturality in $f$ of $\Idx$ and $\rf^*$.
\end{enumerate}
\end{definition}
%

\begin{remark}
\label{2015.05.12.rem1}\rm A J0-structure is called degenerate or extensional
if for all $T\in Ob_{\ge 1}(\CC)$ and $o,o'\in \Obwt(T)$ one has\footnote{The
  condition stated is the classical way of saying that there is an equivalence between
  the types $\Obwt(\Idx(o,o'))$ and $(o=o')$.}
%
$$\Obwt(\Idx(o,o'))=\left\{
\begin{array}{ll}
\emptyset&\mbox{\rm if $o\ne o'$}\\ pt&\mbox{\rm if $o=o'$}
\end{array}
\right.
$$\DG{We should replace $pt$ by the one element set containing $\refl$.}
%
One can easily see that any two extensional J0-structures are equal and that
any extensional J0-structure has a unique extension to a full J-structure that
is also called extensional.

We will not consider these extensional versions of $J$ in the present version of
the paper.
\end{remark}


%
\begin{remark}\rm
\label{2015.05.24.rem1} When one studies J-structures on C-systems that have
no $(\Pi,\lambda)$-structures it is important, as emphasized, for example, in
\cite{vandenBergGarner2011} and perhaps in earlier work by Garner,
to consider a more complex structure than the one
that we consider here.\editorfootnote{So presumably, this paper will find its applications in situations
  where the C-systems do have $(\Pi,\lambda)$-structures.}
This more complex structure can be seen as a family of
structures $eJ_n$, where $eJ_0=\J$ (as above in \ref{2015.03.27.def3}), and where $eJ_n$ over $(\Id,\refl)$ is a
collection of data of the form: for all $\Gamma\in Ob$, for all $T\in
Ob_1(\Gamma)$, for all $\Delta\in Ob_{n}(\Idx(T))$, for all $P\in
Ob_1(\Delta)$, for all $s0\in \Obwt(\rf_T^*(P))$, an element
$eJ_n(\Gamma,T,\Delta,s0)$ in $\Obwt(P)$ such that $eJ_n$ satisfies the
obvious analogue of $\iota$-rule and such that it is natural in $\Gamma$. See
also Remark \ref{2015.05.24.rem2}.
\end{remark}



\subsection{The notion of J-structure on a universe in a category}
%
Let $\C$ be a category\editorfootnote{Not all of the constructions appearing in the sequel
will be invariant under equivalence of categories, and hence, in a formalization
using Voevodsky's Univalent Foundations, such categories will not be assumed to
be ``univalent''.  Indeed, in Voevodsky's view, univalent categories are so
important that non-univalent categories shouldn't even be called categories,
hence the introduction of the term ``C-system''.},
and let $p:\wt{\U}\sr \U$ be a morphism in
$\C$. Recall \cite{Cfromauniverse} that a {\em universe structure} on $p$ is a choice of pullback
squares of the form
\[
\begin{xy}
          \xymatrix@C=4pc{ \gls{semicolon} \ar[r]^-{\gls{QF}} \ar[d]_{\gls{pXF}} & \wt{\U}
            \ar[d]^{p}\\ X \ar[r]^{F} & \U }
\end{xy}
\label{univstrdiag}
\]
for all $X$ and all morphisms $F:X\sr \U$.
We refer to the pullbacks given by a universe structure as \emph{canonical pullbacks}.
A universe in $\C$ is a
morphism with a universe structure on it, and a universe category is a category
with a universe and a choice of a final object $pt$.

\begin{definition}\label{iter-univ-str}
  Given a sequence of maps $$F_{i+1} : (\dots(X;F_1);\dots;F_i) \to \U$$ for each $i$,
  we will use the notation $\gls{multisemicolon}$ for $(\dots(X;F_1);\dots;F_n)$.
\end{definition}

For $f:W\sr X$ and $g:W\sr \wt{\U}$ satisfying $f\circ F = g \circ p$ we will let $\gls{starfg} : W \sr (X;F)$ denote the unique morphism
such that
%
\begin{align}
  (f*g)\circ \p_{X,F} & =f \label{star-eq1} \\
  (f*g)\circ \Q(F) & =g \label{star-eq2}
\end{align}
Observe that if $h : W' \to W$ is a map, then
\begin{align}
  \label{star-functoriality}
  h \circ ( f * g ) & = (h \circ f) * (h \circ g)
\end{align}


%
When we need to distinguish canonical squares arising from different universe structures we may
write $(X;F)_{p}$, $Q_p(F)$, and $f*_p g$.  We may also write $(X;F)'$ for $(X;F)_{p'}$ and $(X;F)'_i$ for $(X;F)_{p'_i}$,
and similarly for $Q(F)$.
%
\begin{remark}\rm
\label{2015.03.29.rm1} Note that we made no assumption about $\Q(\id_\U)$ being
equal to $\id_{\wt{\U}}$. In fact, since we want the results of this paper to be
constructive, we are not allowed to make such an assumption, since 
whether a morphism is an identity morphism need not be
decidable, and therefore we can not normalize a construction of a universe structure by doing
something different when a morphism is not the identity.  The importance
of this observation (in the context of whether a simplex is degenerate)
was emphasized by \cite{BCH}.
\end{remark}

\begin{definition}\label{QfF-defn}
For $X'\stackrel{f}{\sr}X\stackrel{F}{\sr}\U$ we let $\gls{QfF}$ denote the
morphism
%
$$(\p_{X',f\circ F}\circ f)*\Q(f\circ F):(X';f\circ F)\sr (X;F)$$
%
\end{definition}

As shown in \cite[Lemma 2.3]{Cfromauniverse}, the left hand square in the following diagram is a pullback square.
%
\begin{eq}
\label{2015.04.06.l0.sq}
  \begin{xy}
     \xymatrix@C=4pc{
       (X';f\circ F) \ar[r]_{\Q(f,F)}
                     \ar@/^1.5pc/[rr]^{\Q(f \circ F)}
                     \ar[d]_{\p_{X',f\circ F}}            & (X;F) \ar[d]^{\p_{X,F}} \ar[r]_{\Q(F)} & \wt \U \ar[d]_{p} \\
       X' \ar[r]^{f} \ar@/_1.5pc/[rr]_{f \circ F}         & X \ar[r]^F                            & \U
     }
  \end{xy}
\end{eq}%

Following \cite[2.30]{presheavesOb} we define for any universe $p:\wt{\U}\sr \U$ and
any $V\in {\C}$ a contravariant functor from $\C$ to the category of sets,
%
$$\gls{DpV}:X\mapsto \amalg_{F:X\sr \U}Hom((X;F), V).$$
%
For $F:X\sr \U$ and $G : (X;F) \sr V$, the corresponding element of
$\D_p(X,V)$ will be written as the pair $(F,G)$.
%
The action of the functor on a morphism $f : X' \to X$ is given by
%
$$\D_p(f,V):(F,G)\mapsto (f\circ F, \Q(f,F)\circ G).$$

When $\C$ is a locally cartesian closed category, any morphism
$p:\wt{\U}\sr \U$ defines a functor\editorfootnote{Here $(\wt{\U},p)$ and
  $(\U\times V,\pr_1)$ are objects in the category of objects over $\U$, and
  $\iHom$ denotes the internal Hom in that category.}
%
$$\gls{Ip}:V\mapsto \iHom((\wt{\U},p),(\U\times V,\pr_1)),$$
%
into the slice category over $\U$; we denote by
\begin{equation}\label{eq:def-prI}
  \gls{prI}_p(V) :\I_p(V)\sr \U
\end{equation}
the arrow of $\I_p(V)$.

We have constructed in \cite[Construction 2.6.4]{presheavesOb} (originally in \cite[Construction 3.9]{fromunivwithPi}) a family of
bijections
%
$$\gls{etashriek}_{p,X,V}:Hom(X,\I_p(V))\sr \D_p(X,V),$$
%
which are natural in $X$ and $V$.  We let $\etaunshriek$ denote the inverse bijections
%
$$\gls{etaunshriek}_{p,X,V}:\D_p(X,V)\sr Hom(X,\I_p(V)).$$
For $F:X\sr \U$ and $G : (X;F) \sr V$, we will abbreviate
$\etaunshriek_{p,X,V}(F,G)$ to $\etaunshriek_p(F,G)$; this should cause no
confusion, because $X$ and $V$ are determined by $F$ and $G$.
Similarly, for $H : X \sr \I_p(V)$, we will abbreviate $\etashriek_{p,X,V}(H)$
to $\etashriek_p(H)$, provided that $X$ and $V$ can be determined from the context.
We may even write $\etashriek$ for $\etashriek_p$ and $\eta^{',!}$ for $\etashriek_{p'}$.

Using the functorial structure on the mapping $V\mapsto (\U\times V,\pr_1)$
together with the naturality of internal Hom-objects in the second argument we
get a functoriality structure on $\I_p$,
%
\[(f:V\sr V')\mapsto (\I_p(f):\I_p(V)\sr \I_p(V')).\]
%
Similarly, using the functoriality of $\iHom$ in the second argument (see,
e.g., \cite[\S 4.2]{presheavesOb}) we obtain, for any universe $p:\wt{\U}\sr \U$,
any universe $p':\wt{\U}'\sr \U$, any map $h:\wt{\U}'\sr \wt{\U}$ over $\U$, and and object $V$, a morphism
%
\[\gls{IhV} : \I_p(V) \sr \I_{p'}(V).\]
%
\begin{lemma}
  \label{2015.04.10.l2}
  Given a map $f:V\sr V'$ and a map $h:\wt{\U}'\sr \wt{\U}$ over $\U$,
  as in the notation introduced above, the following square is commutative.
%
$$
\begin{xy}
          \xymatrix@C=4pc{ \I_{p'}(V) \ar[r]^{\I_{p'}(f)} \ar[d]_{\I^h(V)} &
            \I_{p'}(V') \ar[d]^{\I^h(V')}\\ \I_p(V) \ar[r]^{\I_p(f)} & \I_p(V') }
\end{xy}
$$
\end{lemma}
%
\begin{myproof}
This is a particular case of the commutative square of \cite[Lemma 4.1.5]{presheavesOb}.
\end{myproof}


%
%
\begin{lemma}
\label{2015.04.02.l4} Let $p:\wt{\U}\sr \U$ and $p':\wt{\U}'\sr \U$ be two
morphisms with universe structures and $f:\wt{\U}'\sr \wt{\U}$ be a morphism over
$\U$. For $V\in {\C}$ let $\I^f(V)$ be the corresponding morphism
$\I_{p'}(V)\sr \I_p(V)$. Then for any $X$ the square
%
$$
\begin{xy}
          \xymatrix@C=3pc{ \D_p(X,V) \ar[r]^-{\etaunshriek_{p,X,V}} \ar[d]_{\D^f(X,V)} &
            Hom(X, \I_p(V)) \ar[d]^{-\circ \I^f(V)}\\ \D_{p'}(X,V)
            \ar[r]^-{\etaunshriek_{p',X.V}} & Hom (X, \I_{p'}(V)), }
\end{xy}
$$
%
where the left hand vertical arrow is defined by
%
$$\gls{DfXV}:(F, F')\mapsto (F,F^*(f)\circ F'),$$
%
commutes.
\end{lemma}
%
\begin{myproof}
Since $\etaunshriek$ is defined as an inverse to $\etashriek$ it is sufficient to show that
for any $g\in Hom(X,\I_p(V))$ one has $\eta^{',!}(g\circ
\I^f(V))=D^f(X,V)(\etashriek(g))$. Let
%
\begin{align*}
  \gls{pr}=\prI_p(V) & :\I_p(V)\sr \U \\
  \pr'=\prI_{p'}(V) & :\I_{p'}(V)\sr \U
\end{align*}
%
(cf.\ \eqref{eq:def-prI}) be the canonical projections. Let
%
\begin{align*}
  \st=\gls{st}_p(V)&:(\I_p(V);\pr)\sr V \\
  \st'=\st_{p'}(V)&:(\I_{p'}(V);\pr')'\sr V
\end{align*}
%
be the morphisms introduced in \cite[(2.60)]{presheavesOb}. By the definition introduced in \cite[(2.65)]{presheavesOb} we have
%
$$\eta^{',!}(g\circ \I^f(V))=(g\circ \I^f(V)\circ \pr', \Q'(g\circ \I^f(V), \pr')\circ \st')$$
%
and
%
$$D^f(X,V)(\etashriek(g))=D^f(X,V)(g\circ \pr, \Q(g,\pr)\circ \st)=(g\circ \pr, (g\circ \pr)^*(f)\circ \Q(g,\pr)\circ \st).$$
%
Therefore it is sufficient to show that
%
\begin{align}
  \I^f(V)\circ \pr' & =\pr \label{Ipr'=pr}
\end{align}
%
and
%
\begin{align}
   \Q'(g\circ \I^f(V), \pr')\circ \st'=(g\circ \pr)^*(f)\circ \Q(g,\pr)\circ \st.     \label{Q'stnat}
\end{align}
%
The first equality asserts that $\I^f(V)$ is a morphism over $\U$, which follows
from its construction.

By Lemma \ref{2015.04.20.l1} we have
%
\[(g\circ \pr)^*(f)\circ \Q(g,\pr)=\Q'(g,\pr)\circ \pr^*(f).\]
%
Next we have the following equations.
%
\begin{align*}
  \Q'(g\circ \I^f(V), \pr')
  & =\Q'(g,\I^f(V)\circ \pr')\circ \Q'(\I^f(V),\pr') & \by{\cite[Lemma 2.5]{Cfromauniverse}} \\
  & =\Q'(g,\pr)\circ \Q'(\I^f(V),\pr') & \by{\ref{Ipr'=pr}}
\end{align*}

It remains to check that
%
\[\Q'(\I^f(V),\pr')\circ \st'=\pr^*(f)\circ \st.\]
%
This requires opening up the definitions \cite[(2.60)]{presheavesOb} of $\st$ and $\st'$, which gives us
%
\[\Q'(\I^f(V),\pr')\circ \iota'\circ ev'\circ \pr_2=\pr^*(f)\circ \iota\circ ev\circ \pr_2.\]
%
It will suffice to prove the following equation:
\begin{equation}\label{eq:proof-two-universes}
  \Q'(\I^f(V),\pr')\circ \iota'\circ ev'=\pr^*(f)\circ \iota\circ ev.
\end{equation}
We will obtain Equation~\eqref{eq:proof-two-universes} as a consequence of commutativity of the three
squares in the following diagram.
\begin{eq}
  \begin{xy}
    \xymatrix{
                    & (\I_p(V);\pr)' \ar[ld] _{\pr^*(f)} \ar[rd]^{\ \Q'(\I^f(V),\pr)} \ar[dd]_{\iota'} & \\
      (\I_{p}(V);\pr) \ar[dd]^{\iota} & & (\I_{p'}(V);\pr')' \ar[dd]^{\iota'} \\
                    & (\I_p(V),\pr)\times_\U(\wt{\U}',p') \ar[ld]^{\id\times f} \ar[rd]_{\I^f(V)\times \id} & \\
      (\I_{p}(V),\pr)\times_\U(\wt{\U},p) \ar[rd]^{ev} & & (\I_{p'}(V),\pr')\times_\U(\wt{\U}',p') \ar[ld]_{ev'} \\
      & \U \times V &
      }
  \end{xy}
\end{eq}%
The top two squares are particular cases of \cite[Lemma 4.1.3]{presheavesOb} applied to the category of objects over $\U$;
the maps $\iota$ and $\iota'$ are defined in the statement of the lemma.
To obtain the upper right square one sets $b=\id_{\wt{\U}'}$ and $a=\I^f(V)$.
To obtain the upper left square one sets $b=f$ and $a=\id_{\I_p(V)}$.
The lower square is a particular case of \cite[Lemma 4.1.6]{presheavesOb}.
%
% This is a lengthy but straightforward computation that applies to any locally cartesian closed category.
\end{myproof}
%
\begin{definition}
\label{2015.03.27.def4} A {\em J0-structure} on a universe $p$ in a category
$\C$ is a morphism $Eq:(\wt{\U};p)\sr \U$.
\end{definition}
%

Let $Eq$ be a J0-structure on $p$. Consider the object\editorfootnote{As
  introduced in \ref{iter-univ-str}, $(\wt{\U};p,Eq)$ is notation for $((\wt{\U};p);Eq)$.}
$$\gls{EUtilde} := (\wt{\U};p,Eq)$$
of $\C$ together with the composite
$$E\wt{\U} \xrightarrow{\ \p_{(\wt{\U};p),Eq}\ } (\wt{\U};p) \xrightarrow{\p_{\wt{\U},p}} \wt{\U}\xrightarrow{p} \U$$
of projections as an object over $\U$; let $\gls{pEUtilde}$ denote that composite map.

\begin{problem}
\label{2015.05.08.prob1} To construct a universe structure on $pE\wt{\U}$.
\end{problem}
%
\begin{construction}{2015.05.08.prob1}\rm
\label{2015.05.08.constr1} The three squares in the following diagram are pullback squares.
\begin{eq}\label{pEU-diagram}
  \begin{xy}
    \xymatrix@C=7pc{
      (X;F,\Q(F)\circ p, \Q(\Q(F),p)\circ Eq)
          \ar[r]^-{\Q(\Q(\Q(F),p),Eq)}
          \ar[d]^-{\p_{(X;F,\Q(F)\circ p), \Q(\Q(F),p)\circ Eq}}                  & (\wt{\U}; p, Eq) \ar[d]_{\p_{(\wt{\U};p),Eq}} \ar@/^2pc/[ddd]^{pE\wt{\U}} \\
      (X;F,\Q(F)\circ p) 
          \ar[r]^-{\Q(\Q(F),p)} 
          \ar[d]^-{\p_{(X;F,\Q(F)), p}}                                           & (\wt{\U};p) \ar[d]_{\p_{\wt{\U},p}} \\
      (X;F) 
          \ar[r]^-{\Q(F)} 
          \ar[d]_{\p_{X,F}}                                                       & \wt{\U} \ar[d]_{p} \\
      X \ar[r]^{F}                                                                & \U
    }
  \end{xy}
\end{eq}%
Remarking that the composite of the right-hand vertical maps is $pE\wt{\U}$, we
define the canonical square for $F$ relative to $pE\wt{\U}$ to be the
square obtained by concatenating these three squares.
\end{construction}
%
Let us denote the components of the canonical squares for $pE\wt{\U}$ as
follows:
%
\begin{eq}
          \xymatrix@C=4pc{ \gls{XFE} \ar[r]^-{\gls{QFE}} \ar[d]_{\gls{pEXF}} &
            E\wt{\U} \ar[d]^{pE\wt{\U}}\\ X \ar[r]^{F} & \U }
\end{eq}
%
Explicitly we have
%
\begin{align}
  (X;F)_{E}&=(X;F,\Q(F)\circ p, \Q(\Q(F),p)\circ Eq) \\
  \Q(F)_{E}&=\Q(\Q(\Q(F),p),Eq) \label{defQFE} \\
  \p_{X,F}^E&=\p_{(X;F,\Q(F)\circ p),\Q(\Q(F),p)\circ Eq}\circ \p_{(X;F),\Q(F)\circ p}\circ \p_{X,F}
\end{align}
%
\begin{definition}
  \label{QfFE-defn}
  For any map $f:X' \sr X$, we will write $\gls{QfFE}$ for the canonical morphism from
  $(X;f\circ F)_{E}$ to $(X;F)_{E}$, defined analogously to \ref{QfF-defn}.  It fits into
  the following diagram.
  \begin{eq}
  \label{QfFE-diag}
    \begin{xy}
       \xymatrix@C=4pc{
         (X';f\circ F)_E \ar[r]_{\Q(f,F)_E}
                       \ar@/^1.5pc/[rr]^{\Q(f \circ F)_E}
                       \ar[d]_{\p^E_{X',f\circ F}}          & (X;F)_E \ar[d]^{\p^E_{X,F}} \ar[r]_{\Q(F)_E} & E\wt\U \ar[d]_{pE\wt\U} \\
         X' \ar[r]^{f} \ar@/_1.5pc/[rr]_{f \circ F}         & X \ar[r]^F                                   & \U
       }
    \end{xy}
  \end{eq}%
  
\end{definition}
%
\begin{definition}
\label{2015.03.27.def5} Let $p:\wt{\U}\sr \U$ be a universe in $\C$ and $Eq$ be a
J0-structure on $p$. A {\em J1-structure} on $p$ over $Eq$ is a morphism
$\Omega:\wt{\U}\sr \wt{\U}$ such that the square
%
\begin{eq}\label{2015.03.27.sq1}
\begin{xy}
          \xymatrix@C=4pc{ \wt{\U} \ar[r]^-{\Omega} \ar[d]_{\Delta} & \wt{\U}
            \ar[d]^{p}\\ (\wt{\U};p) \ar[r]^{Eq} & \U, }
\end{xy}
\end{eq}%
%
where $\gls{Delta} := (\id_{\wt{\U}})*(\id_{\wt{\U}})$ is the diagonal of $\wt{\U}$,
commutes.

The square (\ref{2015.03.27.sq1}) defines a morphism $\wt{\U}\sr E\wt{\U}$, which
will be denoted by $\gls{omega}$, as in the following diagram.
\begin{eq}\label{diag4}
  \xymatrix@C=4pc{
    \wt{\U}
    \ar[rd]^-{\omega}
    \ar@/^1pc/[rrd]^-{\Omega}
    \ar@/_1pc/[drd]_{\Delta} \\
                      & E\wt\U \ar[r]^{\Q(Eq)} 
                               \ar[d]^{\p_{(\wt\U;p),Eq}}   & \wt{\U} \ar[d]^{p}  \\
                      & (\wt{\U};p) \ar[r]_{Eq}             & \U
    }
\end{eq}

\end{definition}

To define a J2-structure on a universe we will need to assume that $\C$
is a locally cartesian closed category.  Recall that a locally cartesian closed
category is a category with the choice of fiber squares based on all pairs of
morphisms with a common codomain as well as the choice of relative internal
Hom-objects and co-evaluation morphisms for all such pairs. For our notation
related to the locally cartesian closed categories as well as for some other
notations used below see \cite{fromunivwithPiI,presheavesOb,fromunivwithPiII}.

When a universe is considered in a locally cartesian closed category we make no
assumption about the compatibility of choices of the pullback squares of the
universe structure on $p$ and pullback squares of the locally cartesian closed
structure.

Consider the functors $\I_{p}$ and $\I_{pE\wt{\U}}$. We have the following
commutative square:
%
\begin{eq}\label{2010.sq1}
\begin{xy}
          \xymatrix@C=4pc{ \I_{pE\wt{\U}}(\wt{\U}) \ar[r]^-{\I^{\omega}(\wt{\U})}
            \ar[d]_{\I_{pE\wt{\U}}(p)} & \I_p(\wt{\U})
            \ar[d]^{\I_p(p)}\\ \I_{pE\wt{\U}}(\U) \ar[r]^{\I^{\omega}(\U)} & \I_p(\U) }
\end{xy}
\end{eq}%
%
and therefore a morphism
%
$$\gls{coJ} : \I_{pE\wt{\U}}(\wt{\U}) \lr (\I_{pE\wt{\U}}(\U), \I^{\omega}(\U))
\times_{\I_p(\U)} (\I_p(\wt{\U}), \I_p(p))
$$
%
\begin{definition}
\label{2015.03.27.def6} A {\em J2-structure} on $p:\wt{\U}\sr \U$, relative to a J0-structure $Eq$
and a J1-structure $\Omega$ over $Eq$, is a morphism
%
$$ Jp:( \I_{pE\wt{\U}}(\U), \I^{\omega}(\U))\times_{\I_p(\U)} (\I_p(\wt{\U}), \I_p(p))\sr
\I_{pE\wt{\U}}(\wt{\U}) $$
%
such that $Jp\circ coJ = \id$.
\end{definition}

\begin{definition}
  A J-structure on $p:\wt{\U}\sr \U$ is a triple $(Eq,\Omega,Jp)$, where $Eq$
  is a J0-structure, $\Omega$ is a J1-structure relative to $Eq$, and $Jp$ is a
  J2-structure relative to $Eq$ and $\Omega$.
\end{definition}

\begin{definition}
  \label{Fp-defn}
Let $\gls{Fp} = \gls{Fp}_{Eq,\Omega}$ denote the fiber product
%
$$(\I_{pE\wt{\U}}(\U), \I^{\omega}(\U)) \times_{\I_p(\U)} (\I_p(\wt{\U}), \I_p(p))$$
%
and let $\gls{pFp}_{Eq,\Omega}$ be the projection
$\Fp_{Eq,\Omega}\sr \U$. Let further $\pr_1$ be the projection from $\Fp$ to
$\I_{pE\wt{\U}}(\U)$, and let $\pr_2$ be the projection from $\Fp$ to $\I_p(\wt{\U})$.
\end{definition}

Note that we have the following two equations.
\begin{eq}
  \label{2015.04.04.eq2}
  Jp \circ \I_{pE\wt{\U}}(p)=Jp \circ coJ\circ \pr_1 = \pr_1
\end{eq}
\begin{eq}
  \label{2015.04.04.eq1}
  Jp \circ \I^{\omega}(\wt{\U}) = Jp \circ coJ\circ \pr_2 = \pr_2 
\end{eq}

\begin{definition}
  Let $\C$ be a cartesian closed category.  For any objects $X$, $Y$, $Z$ of $\C$, we let
  $$ \gls{adj} : Hom (X, \iHom(Y,Z) ) \xrightarrow \cong Hom(X \times Y, Z) $$
  denote the corresponding adjunction bijection.
\end{definition}

\begin{remark}
  We will use the adjunction $\adj$ as follows.  Consider the map
  $\pr_1 : \Fp \to \I_{pE\wt{\U}}(\U) = \iHom_U(E\wt\U,\U\times\U)$ over $\U$.
  Applying $\adj$ yields a map $\adj(\pr_1) : \Fp \times_\U E\wt\U \to \U\times\U$ over $\U$.
  Composing with $\pr_2 : E\wt\U \times \U \to \U$ yields a map
  $\adj(\pr_1) \circ \pr_2 : \Fp \times_\U \wt\U \to \U$ (not over $\U$).

  The same reasoning yields the map $\adj(\pr_2) \circ \pr_2 : \Fp \times_\U \wt\U \to \wt\U$.

  These maps will be used below.
\end{remark}

Our solution to the following problem is the key to the construction of
J-structures over a given J1-structure in categories with weak factorization
systems, particularly in Quillen model categories.

\begin{problem}
\label{2015.05.12.l1} Let $\C$ be a category with a locally cartesian
closed structure and $Eq,\Omega$ be a J1-structure on $({\C},p)$. To
construct a bijection between the set of J-structures on $p$ over $(Eq,\Omega)$
and the set of morphisms $(\Fp,\pFp)\times_\U(E\wt{\U},pE\wt{\U})\sr \wt{\U}$ that
split the following commutative square into two commutative triangles:
%
\begin{eq}\label{2015.05.22.sq1}
\begin{xy}
  \xymatrix@C=7pc{
    (\Fp,\pFp)\times_\U(\wt{\U},p) \ar[r]^-{\adj(\pr_2)\circ \pr_2} \ar[d]_{\id_{\Fp}\times \omega} & \wt{\U} \ar[d]^{p} \\
    (\Fp,\pFp)\times_\U(E\wt{\U},pE\wt{\U}) \ar[r]^-{\adj(\pr_1)\circ \pr_2} \ar@{.>}[ur]           & \U
          }
\end{xy}
\end{eq}%
(Commutativity of the square follows from the naturality of the adjunction $\adj$ and
the equation $\pr_1 \circ \I^\omega(\U) = \pr_2 \circ \I_p(p)$.)
\end{problem}
%
\begin{remark}\rm
If we omit, as is customarily done, the explicit functions to the base in the
notation of the fiber products, then the square (\ref{2015.05.22.sq1}) takes the
following form.
%
\begin{eq}\label{2017.08.16.sq1}
\begin{xy}
          \xymatrix@C=7pc{ (\I_{pE\wt{\U}}(\U)\times_{\I_p(\U)}
            \I_p(\wt{\U}))\times_\U\wt{\U} \ar[r]^-{\adj(\pr_2)\circ \pr_2}
            \ar[d]_{\id_{\Fp}\times \omega} & \wt{\U}
            \ar[d]^{p}\\ (\I_{pE\wt{\U}}(\U)\times_{\I_p(\U)} \I_p(\wt{\U}))\times_\U
            E\wt{\U} \ar[r]^-{\adj(\pr_1)\circ \pr_2} \ar@{.>}[ur] & \U }
\end{xy}
\end{eq}%




\end{remark}
%
\begin{construction}{2015.05.12.l1}\rm
\label{2015.05.22.constr1} Observe first that there is a bijection between the
set of morphisms
%
$$(\Fp,\pFp)\times_\U(E\wt{\U},pE\wt{\U})\sr \wt{\U}$$
%
that split the square (\ref{2015.05.22.sq1}) into two commutative triangles and
the set of morphisms
%
$$(\Fp,\pFp)\times_\U(E\wt{\U},pE\wt{\U})\sr \U\times\wt{\U}$$
%
that split into two commutative triangles the commutative square:
%
$$
\begin{xy}
          \xymatrix@C=7pc{ (\Fp,\pFp)\times_\U(\wt{\U},p) \ar[r]^-{\adj(\pr_2)}
            \ar[d]_{\id_{\Fp}\times \omega} & \U\times \wt{\U} \ar[d]^{\id_\U\times
              p}\\ (\Fp,\pFp)\times_\U(E\wt{\U},pE\wt{\U}) \ar[r]^-{\adj(\pr_1)} \ar@{.>}[ur] &
            \U\times \U }
\end{xy}
$$
%
The rule $f\mapsto \adj(f)$ gives us a bijection of the form
%
$$Hom_\U((\Fp,\pFp),(\I_{pE\wt{\U}}(\wt{\U}),\_))\sr Hom_\U((\Fp,\pFp)\times_\U
(E\wt{\U},pE\wt{\U}), (\U\times\wt{\U}, \pr_2))$$
%
All sections of $coJ$ are automatically morphisms over $\U$. Therefore it
remains to show that this bijection defines a bijection of the subset of
morphisms that are sections of $coJ$ and morphisms that make the two triangles
commutative.

One verifies first that a morphism $f:\Fp\sr \I_{pE\wt{\U}}(\wt{\U})$ is a section
of $coJ$ if and only if $f\circ \I_{pE\wt{\U}}(p)=\pr_1$ and $f\circ
\I^{\omega}(\wt{\U})=\pr_2$. This is omitted.

Next we have
%
\begin{align*}
  \I_{pE\wt{\U}}(p)&=\iHom_\U((E\wt{\U},pE\wt{\U}),\id_\U\times p) \\
  \I^{\omega}(\wt{\U})&=\iHom_\U(\omega,(\U\times\wt{\U},\pr_2))
\end{align*}
%
Therefore by \cite[Lemma 4.1.7]{presheavesOb} one has
%
\begin{align*}
  \adj(f\circ \I_{pE\wt{\U}}(p))&=\adj(f)\circ (\id_\U\times p) & \text{and} \\
  \adj(f\circ \I^{\omega}(\wt{\U}))&=(\id_{\Fp}\times_{\U}\omega)\circ \adj(f),
\end{align*}
%
and we conclude that $f$ is a section of $coJ$ if and only if
%
\begin{align*}
  \adj(f)\circ (\id_\U\times p)&=\adj(\pr_1) & \text{and} \\
  (\id_{\Fp}\times_{\U}\omega)\circ \adj(f)&=\adj(\pr_2).
\end{align*}
% 
These two equations are the ones that assert the two triangles involving $\adj(f)$ commute.
This completes the construction.
\end{construction}
%
\begin{remark}\rm
\label{2015.05.24.rem2} It is likely to be relatively easy to generalize the
constructions of this paper to the extended J-structures $eJ_n$ (see Remark
\ref{2015.05.24.rem1}). The key to such a generalization is \cite[Remark
  3.13]{fromunivwithPi}\DG{The cited remark does not appear to have been
  published -- search more carefully for it in the three published papers that
  this preprint evolved into.}.
The structures $eJp_n$ can be defined in the same way
as $Jp$ but with the square (\ref{2010.sq1}) replaced by the square
%
\begin{eq}
\label{2015.05.24.sq1}
\begin{xy}
          \xymatrix@C=4pc{ \I_{pE\wt{\U}}(\I_p^n(\wt{\U}))
            \ar[r]^-{\I^{\omega}(\I_p^n(\wt{\U}))} \ar[d]_{\I_{pE\wt{\U}}(\I_p^n(p))}
            & \I_p(\I_p^n(\wt{\U}))
            \ar[d]^{\I_p(\I_p^n(p))}\\ \I_{pE\wt{\U}}(\I_p^n(\U))
            \ar[r]^-{\I^{\omega}(\I_p^n(\U))} & \I_p(\I_p^n(\U)) }
\end{xy}
\end{eq}%
\end{remark}
%



\subsection{J-structures on universes in categories with two classes of morphisms}
%
%

Recall that a collection of morphisms $R$ is said to have the right lifting
property for the collection of morphisms $L$ if for any commutative square of
the form
%
\[
\begin{xy}
          \xymatrix@C=2pc{ Z \ar[r]^-{f_Z} \ar[d]_{i} & E \ar[d]^{p}\\ W
            \ar[r]^-{f_W} & B }
\end{xy}
\]
%
such that $i\in L$ and $p\in R$ there exists a morphism $g:W\sr E$ that makes
the two triangles into which it splits the square commute, i.e., a morphism
$g$ such that $i\circ g=f_Z$ and $g\circ p=f_W$.
%

We are going to consider two sets of conditions (Conditions
\ref{2015.05.22.cond2} and \ref{2015.05.22.cond1}) on a pair of classes of
morphisms $FB$ and $TC$ in a category with fiber products and then show in
Theorems \ref{2015.05.22.th1} and \ref{2015.05.16.th1} how pairs satisfying
conditions of each of these two sets can be used to construct J-structures on
elements of $FB$.

\begin{remark}\rm
  This is the only part of the paper where we depart from constructions that are
conservatively algebraic over the theory of categories, i.e., from
constructions that can be expressed in terms of adding new quasi-algebraic
operations to the theory of categories without adding new sorts to this theory.

Considering classes of morphisms in categories can be expressed in the
quasi-algebraic way, but this requires adding new sorts to the theory.

This is also the only context where we use the concept ``there exists'' in this
paper. In all the previous cases the objects that we considered were given
(specified). To make the lemmas proved below into constructions and to avoid
the use of ``there exists'' one would have to define the collection $FB$ as a
collection of pairs of a morphism $p$ together with, for all $i\in TC$, $f_W$
and $f_Z$ such that $f_Z\circ p=i\circ f_W$, a morphism $g$ such that $i\circ
g=f_Z$ and $g\circ p=f_W$.

\end{remark}


Our first set of conditions is as follows:
%
\begin{cond}\label{2015.05.22.cond2}
\begin{enumerate}
\item A morphism is in $FB$ if and only if it has the right lifting property
  for $TC$,
\item consider morphisms $f: B'\sr B$, $p_1:E_1\sr B$, $p_2:E_2\sr B$ and
  $i:E_1\sr E_2$ such that $p_1,p_2\in FB$, $i \circ p_2 = p_1$, and $i\in TC$. Then the morphism
%
$$\id_{B'}\times i: (B',f)\times_B(E_1,p_1)\sr (B',f)\times_B(E_2,p_2)$$
%
is in $TC$.
\end{enumerate}
\end{cond}
%
\begin{theorem}
\label{2015.05.22.th1} Let $FB$ and $TC$ be two classes of morphisms in a
locally cartesian closed category $\cal C$ that satisfy Conditions
\ref{2015.05.22.cond2}. Let $p:\wt{\U}\sr \U$ be a universe in $\cal C$, let 
$Eq$ be a J0-structure on $p$, and let $\Omega$ be a J1-structure over $Eq$.
Assume further that:
%
\begin{enumerate}
\item $p$ is in $FB$,
\item $\omega : \wt{\U}\sr E\wt{\U}$ is in $TC$ (see Def.~\ref{2015.03.27.def5} for the definition of $\omega$).
\end{enumerate}
%
Then there exists an extension of $(Eq,\Omega)$ to a full J-structure on $p$.
\end{theorem}
%
\begin{myproof}
Let us apply Construction \ref{2015.05.22.constr1} to $(Eq,\Omega)$. To
construct the required morphism it is sufficient to establish that
$\id_{\Fp}\times\omega$ is in $TC$. It follows from the first of our conditions
that $FB$ is closed under pullbacks and compositions. Therefore, $pE\wt{\U}$ is
in $FB$. It remains to apply the second of our conditions.
\end{myproof}
% 





Our second set of conditions is more involved. Conditions of this set can be
satisfied in the situations arising when one attempts to localize Quillen model
structures and when the resulting sets of morphisms do not form a model
structure. The difference is mainly concerned with the fact that the good
behavior is required only for fibrations over fibrant objects. One particular
example of such a situation is considered in \cite[Section 3.3]{SRF}.

%
\begin{cond}\label{2015.05.22.cond1}
%
\begin{enumerate}
\item $\id_{pt}$ is in $FB$,
\item let $B$ be such that the morphism $B\sr pt$ is in $FB$, then a morphism
  $p:E\sr B$ is in $FB$ if and only if it has the right lifting property for
  $TC$,
\item if $p:E\sr B$ and $B\sr pt$ are in $FB$, $i:Z\sr W$ is in $TC$ and
  $f:W\sr B$ is an arbitrary morphism, then
%
$$(i\times_\U \id_E):(Z,i\circ f)\times_B (E,p)\sr (W,f)\times_B (E,p)$$
%
is in $TC$.
%
\end{enumerate}
\end{cond}
%
We will say that $B$ is fibrant if the morphism $B\sr pt$ is in $FB$.
%
\begin{lemma}
  \label{2015.05.14.l2}
  Assume conditions \ref{2015.05.22.cond1} are satisfied.
Let $p:E\sr B$ be in $FB$ and $f:B'\sr B$ be a
morphism. Assume in addition that $B$ and $B'$ are fibrant, then for any
pullback square of the form
%
\[
\begin{xy}
          \xymatrix@C=2pc{ E' \ar[r] \ar[d]_{p'} & E \ar[d]^{p}\\ B'
            \ar[r]^-{f} & B }
\end{xy}
\]
%
the morphism $p'$ is in $FB$.
\end{lemma}
%
\begin{myproof}
Since $B'$ is fibrant it is sufficient to verify that $p'$ has the right
lifting property for $TC$. This can be shown in the standard way to be a
consequence of $p$ having the right lifting property for $TC$. That $p$ has
this property we know because $p$ is in $FB$ and $B$ is fibrant.
\end{myproof}
%
\begin{lemma}
  \label{2015.05.14.l4}
  Assume conditions \ref{2015.05.22.cond1} are satisfied.
  Let $B$ be fibrant and $p_2:E_2\sr E_1$, $p_1:E_1\sr B$
  be in $FB$. Then $p_2\circ p_1$ is in $FB$.
\end{lemma}
%
\begin{myproof}
Let us show first that $E_1$ is fibrant, i.e., that $\pi_{E_1}:E_1\sr pt$ is in
$FB$. Since $pt$ is fibrant it is sufficient to show that $\pi_{E_1}$ has the
right lifting property for $TC$. It is shown in the standard way from the fact
that both $p_1$ and $\pi_B:B\sr pt$ have the right lifting property for $TC$
and $\pi_{E_1}=p_1\circ \pi_B$.

Since $E_1$ is fibrant we know that $p_2$ has the right lifting property for
$TC$, and since $B$ is fibrant we know that $p_1$ has the right lifting property
for $FB$. From this we conclude in the standard way that $p_2\circ p_1$ have
the right lifting property for $TC$, and since $B$ is fibrant this implies that
$p_2\circ p_1$ is in $FB$.
\end{myproof}
%
\begin{lemma}
  \label{2015.05.14.l1}
  Assume conditions \ref{2015.05.22.cond1} are satisfied.
  Assume that $\U,V$ are fibrant and that $p:\wt{\U}\sr \U$
  is in $FB$. Then the morphism $\prI_p(V):\I_p(V)\sr \U$ is in $FB$.
\end{lemma}
%
\begin{myproof}
Since $\U$ is fibrant it is sufficient to check that $\pr=\prI_p(V)$ has the right
lifting property for $TC$. Consider a commutative square of the form
%
$$
\begin{xy}
          \xymatrix@C=2pc{ Z \ar[r]^-{f_Z} \ar[d]_{i} &
            \iHom_\U((\wt{\U},p),(\U\times V,\pr_1)) \ar[d]^{\pr}\\ W
            \ar[r]^-{f_W} & \U }
\end{xy}
$$
%
We need to construct a morphism $f:W\sr \iHom_\U((\wt{\U},p),(\U\times
V,\pr_1))$ that would make the two triangles commutative. The commutativity of
the lower triangle means that $f$ is a morphism over $\U$ which is equivalent to
the assumption that $f=\adj^{-1}(g)$ for some $g:(W,f_W)\times_\U (\wt{\U},p)\sr
\U\times V$ over $\U$.

Consider the square
%
$$
\begin{xy}
          \xymatrix@C=4pc{ (Z, i\circ f_W)\times_\U (\wt{\U},p)
            \ar[r]^-{\adj(f_Z)} \ar[d]_{i\times \id_{\wt{\U}}} & \U\times V
            \ar[d]^{\pr_1}\\ (W,f_W)\times_\U (\wt{\U},p) \ar[r] & \U },
\end{xy}
$$
where the bottom horizontal arrow is the projection to $\U$.
%
By Lemma \ref{2015.05.14.l2} we know that $\pr_1$ belongs to $FB$. By our
assumptions on $TC$ and $FB$ we know that $i\times \id_{\wt{\U}}$ is in
$TC$. Therefore there exists a morphism $g:(W,f_W)\times_\U (\wt{\U},p) \sr
\U\times V$ that makes the two triangles commute.  The commutativity of the
lower triangle means that this is a morphism over $\U$. Therefore $\adj^{-1}(g)$
is defined. Set $f=\adj^{-1}(g)$. It remains to check that $i\circ f=f_Z$. This
is equivalent to $\adj(i\circ f)=\adj(f_Z)$. Since $\adj(i\circ f)=(i\times
\id_{\wt{\U}})\circ \adj(f)$ by \cite[Lemma 8.7(3)]{fromunivwithPi}, this is
equivalent to $(i\times \id_{\wt{\U}})\circ g=\adj(f_Z)$ which is the
commutativity of the upper triangle.
\end{myproof}
%
\begin{lemma}
  \label{2015.05.14.l3}
  Assume conditions \ref{2015.05.22.cond1} are satisfied.
  Assume that $\U$ and $V$ are fibrant and that $p:\wt{\U}\sr \U$
  and $r:V'\sr V$ are in $FB$. Then $\I_p(r):\I_p(V')\sr \I_p(V)$ is in $FB$.
\end{lemma}
%
\begin{myproof}
By Lemmas \ref{2015.05.14.l1} and \ref{2015.05.14.l4} we know that $\I_p(V)$ is
fibrant. Therefore it is sufficient to show that $\I_p(r)$ has the right lifting
property for $TC$. Consider a commutative square of the form
%
\begin{eq}
\label{2015.05.14.sq1}
\begin{xy}
          \xymatrix@C=4pc{ Z \ar[r]^-{f_Z} \ar[d]_{i} &
            \iHom_\U((\wt{\U},p),(\U\times V',\pr_1))
            \ar[d]^{\iHom_\U((\wt{\U},p), \id_\U\times r)}\\ W \ar[r]^-{f_W} &
            \iHom_\U((\wt{\U},p), (\U\times V, \pr_1)) }
\end{xy}
\end{eq}%
%
The lower right corner is an object over $\U$ through the morphism $p\triangle
\pr_1$. Let $\p_W=f_W\circ (p\triangle \pr^{\U,V}_\U)$ and
%
\[ \p_Z=i\circ \p_W=f_Z\circ (p\triangle \pr^{\U,V'}_\U) . \]
%
Consider the square
%
\begin{eq}
\label{2015.05.14.sq2}
\begin{xy}
          \xymatrix@C=4pc{ (Z,\p_Z)\times_\U (\wt{\U},p) \ar[r]^-{\adj(f_Z)}
            \ar[d]_{i\times \id_{\wt{\U}}} & \U\times V' \ar[d]^{\id_\U\times
              r}\\ (W,\p_W)\times_\U (\wt{\U},p) \ar[r]^-{\adj(f_W)} & \U\times V }
\end{xy}
\end{eq}%
%
This square commutes. Indeed,
%
\begin{align*}
  \adj(f_Z)\circ (\id_\U\times r )
  & = \adj(f_Z\circ \iHom_\U((\wt{\U},p), \id_\U\times r)) \\
  & = \adj(i\circ f_W) \\
  & = (i\times \id_{\wt{\U}})\circ \adj (f_W),
\end{align*}
%
where the first equality is by \cite[Lemma 8.7(1)]{fromunivwithPi} and the
third by \cite[Lemma 8.7(3)]{fromunivwithPi}. By Lemmas \ref{2015.05.14.l2} and
\ref{2015.05.14.l4} we know that $\id_\U\times r$ is in $FB$. By our assumption
(3) on $FB$ and $TC$ we know that $i\times \id_{\wt{\U}}$ is in $TC$. Therefore,
there exists a morphism $g:(W,\p_W)\times_\U (\wt{\U},p) \sr \U\times V'$ that
splits this square into two commutative triangles. Since the lower triangle
commutes, $g$ is a morphism over $\U$ and, in particular, $g=\adj(f)$ for some
$f:W\sr \iHom_\U((\wt{\U},p),(\U\times V',\pr_1))$. Let us show that $f$ splits
the square (\ref{2015.05.14.sq1}) into two commutative triangles, i.e., that we
have $i\circ f= f_Z$ and $f\circ \iHom_\U((\wt{\U},p), \id_\U\times r)=f_W$.

The first equality is equivalent to $\adj(i\circ f)=\adj(f_Z)$ which is
equivalent, by \cite[Lemma 8.7(3)]{fromunivwithPi} to
$(i\times \id_{\wt{\U}})\circ g=\adj(f_Z),$
which is the commutativity of the upper of the
two triangles into which $g$ splits (\ref{2015.05.14.sq2}).

The second equality is equivalent to
$\adj(f\circ \iHom_\U((\wt{\U},p), \id_\U\times r))=\adj(f_W)$,
which is equivalent by \cite[Lemma 8.7(1)]{fromunivwithPi} to
$g\circ (\id_\U\times r)=\adj(f_W),$ which is the
commutativity of the lower of the two triangles into which $g$ splits
(\ref{2015.05.14.sq2}).
\end{myproof}
%

We can now prove the second main theorem of this section.
%
\begin{theorem}
\label{2015.05.16.th1} Let $({\C},p,pt)$ be a universe category, let
$\C$ be given a locally cartesian closed structure and let $TC$ and
$FB$ be two collections of morphisms in $\C$ that satisfy Conditions
\ref{2015.05.22.cond1}. Let further $Eq:(\wt{\U};p)\sr \U$ be a J0-structure on $p$,
and let $\Omega:\wt{\U}\sr \wt{\U}$ be a J1-structure over $Eq$.
Assume that the following conditions hold:
%
\begin{enumerate}
\item $\U$ is fibrant,
\item $p:\wt{\U}\sr \U$ is in $FB$,
\item the map $\omega : \wt{\U}\sr E\wt{\U}$ constructed in \ref{diag4} is in $TC$.
\end{enumerate}
%
Then there exists a J2-structure $Jp$ relative to $Eq$ and $\Omega$.
\end{theorem}
%
\begin{myproof}
Let us use the notation of Problem \ref{2015.05.12.l1}. We need to show that
under the assumptions of the current theorem there exists a morphism that
splits the square of Problem \ref{2015.05.12.l1} into two commutative
triangles. Observe first that constructing such a splitting is equivalent to
constructing the splitting of the square
%
$$
\begin{xy}
          \xymatrix@C=4pc{ (\wt{\U},p)\times_\U (\Fp,\pFp) \ar[r]^-{\sigma\circ
              \adj(\pr_2)} \ar[d]_{\omega \times \id_{\Fp}} & \U\times\wt{\U}
            \ar[d]^{\id_\U\times p}\\ (E\wt{\U},pE\wt{\U})\times_\U (\Fp,\pFp)
            \ar[r]^-{\sigma'\circ \adj(\pr_1)} & \U\times \U, }
\end{xy}
$$
%
where
%
\begin{align*}
  \sigma & :(\wt{\U},p)\times_\U (\Fp,\pFp)\sr (\Fp,\pFp)\times_\U (\wt{\U},p) \\
  \sigma'& :(E\wt{\U},pE\wt{\U})\times_\U (\Fp,\pFp) \sr (\Fp,\pFp) \times_\U (E\wt{\U},pE\wt{\U})
\end{align*}
%
are permutations of the factors.

It is easy to show that $\U\times \U$ is fibrant. Therefore it is sufficient to
show that $\id_\U\times p$ is in $FB$ and $\omega\times_\U \id_{\Fp}$ is in
$TC$. The first fact follows from the assumption that $p$ is in $FB$ and that
$\U$ is fibrant. To obtain the second fact let us apply condition (3) on the
classes $FB$ and $TC$ to $B=\U$, $f=pE\wt{\U}$, $i=\omega$ and $p=\pFp$.  It
remains to show that $\pFp$ is in $FB$. We can represent $\pFp$ as the
composition
%
$$\Fp\stackrel{\pr_1}{\sr} \I_{pE\wt{\U}}(\U) \stackrel{\prI_{pE\wt{\U}}}{\sr} \U$$
%
The morphism $pE\wt{\U}$ is in $FB$ as a composition of pullbacks of $p$ with
respect to morphisms with fibrant domains through repeated application of
Lemmas \ref{2015.05.14.l2} and \ref{2015.05.14.l4}. Therefore, the morphism
$\prI_{pE\wt{\U}}$ is in $FB$ by Lemma \ref{2015.05.14.l1} and as a corollary we
know that $\I_{pE\wt{\U}}(\U)$ is fibrant. Similarly $\I_p(\U)$ is fibrant and
$\I_p(p)$ is in $FB$ and applying again Lemma \ref{2015.05.14.l2} we see that
$\pr_1$ is in $FB$. And again by Lemma \ref{2015.05.14.l4} we see that $\pFp$ is
in $FB$ which finishes the proof of the theorem.
\end{myproof}
%
\begin{cor}
\label{2015.05.18.cor1} Let $\cal C$ be a locally cartesian closed category
with a Quillen model structure, $p$ a universe in $\cal C$ and $(Eq,\Omega)$ a
J1-structure on $p$. Assume further that $p$ is a fibration and $\omega : \wt{\U}\sr E\wt{\U}$ is a
trivial cofibration and that in addition one of the following two conditions
holds:
%
\begin{enumerate}
\item consider morphisms $f: B'\sr B$, $p_1:E_1\sr B$, $p_2:E_2\sr B$ and
  $i:E_1\sr E_2$ such that $p_1,p_2$ are fibrations, $i \circ p_2 = p_1$, and $i$ is a trivial
  cofibration. Then the morphism
%
$$\id_{B'}\times i: (B',f)\times_B(E_1,p_1)\sr (B',f)\times_B(E_2,p_2)$$
%
is a trivial cofibration; or
%
\item $\U$ is fibrant and the pullback of a trivial cofibration along a
  fibration is a trivial cofibration.
\end{enumerate}
%
Then $(Eq,\Omega)$ can be extended to a full J-structure on $p$.
\end{cor}
%

The following result can be used to produce many examples of (non-univalent) universes with
J-structures.

Let $\cal C$ be a locally
cartesian closed category with coproducts $\amalg_{n\in\nn}X_n$ of sequences of objects.
We let $in_n:X_n\sr \amalg_n X_n$ the canonical inclusion.
Given a sequence of maps $f_n : X_n \sr Y$, we let
$\langle f_n \rangle_{n} : \amalg_n X_n\sr Y$ denote the associated morphism.
Given two sequences of objects, $X_n$ and $Y_n$, along with a sequence of maps $f_n : X_n \sr Y_n$,
we let $\amalg f_n : \amalg_n X_n \sr \amalg_n Y_n$ denote the morphism $\langle f_n\circ in_n \rangle_{n}$.

Assume that these coproducts satisfy the following two conditions:
%
\begin{enumerate}
\item for a sequence of morphisms $f_n:E_n\sr B_n$ the square
%
$$
\begin{xy}
          \xymatrix@C=4pc{ \amalg_n (E_n,f_n)\times_{B_n}(E_n,f_n)
            \ar[r]^-{\amalg_n \pr_2} \ar[d]_{\amalg_n \pr_1} & \amalg_n E_n
            \ar[d]^{\amalg_n f_n}\\ \amalg_n E_n \ar[r]^-{\amalg_n f_n} &
            \amalg_n B_n }
\end{xy}
$$
%
is a pullback square,
%
\item for a sequence of morphisms $f_n:E_n\sr B_n$ the square
%
$$
\begin{xy}
          \xymatrix@C=4pc{ \amalg_n E_{n+1} \ar[r]^-{\langle in_{n+1}
              \rangle_{n}} \ar[d]_{\amalg_n f_{n+1}} & \amalg_n E_n
            \ar[d]^{\amalg_n f_n}\\ \amalg_n B_{n+1} \ar[r]^-{\langle in_{n+1}
              \rangle_{n}} & \amalg_n B_n }
\end{xy}
$$
%
is a pullback square.
\end{enumerate}
%
%
\begin{problem}
\label{2015.05.22.th2} Let $\cal C$ be as above $FB$ and $TC$ two classes of
morphisms satisfying one of the sets of conditions \ref{2015.05.22.cond1} or
\ref{2015.05.22.cond2}. Assume in addition the following:
%
\begin{enumerate}
\item the coproduct of a sequence of morphisms from $TC$ is in $TC$ and the
  coproduct of a sequence of morphisms from $FB$ is in $FB$,
\item the composition of a morphism from $TC$ with an isomorphism is in $TC$,
\item for any morphism $f: X \sr Y$ there is given an object $P(f)$ and
  morphisms $i_f:X\sr P(f)$, $q_f:P(f)\sr Y$ such that $i_f\in TC$, $q_f\in FB$
  and $f=i_f\circ q_f$.
\end{enumerate}
%
To construct, for any universe $p: \wt{\U}\sr \U$ such that $p\in FB$ a sequence
of morphisms $\p_n:\wt{\U}_n\sr \U_n$ such that $\p_0=p$, $\p_n\in FB$ and $\amalg_n
p$, with the universe structure defined by the fiber squares of $\cal C$, has a
J-structure with $\omega\in TC$.
\end{problem}
%
\begin{construction}{2015.05.22.th2}\rm\label{2015.05.23.constr1}
Define $\p_n:\wt{\U}_n\sr \U_n$ inductively as follows. For $n=0$ we take
$\p_0=p$. To define $\p_{n+1}$ consider the diagonal $\Delta_n:\wt{\U}_n\sr
(\wt{\U}_n,\p_n)\times_{\U_n}(\wt{\U}_n,\p_n)$ and let
%
$$\p_{n+1}=q_{\Delta_n}:P(\Delta_n)\sr
(\wt{\U}_n,\p_n)\times_{\U_n}(\wt{\U}_n,\p_n)$$
%
so that, in particular, $\U_{n+1}=(\wt{\U}_n,\p_n)\times_{\U_n}(\wt{\U}_n,\p_n)$.

Let $\U_*=\amalg_n \U_n$, $\wt{\U}_*=\amalg_n\wt{\U}_n$ and $\p_*=\amalg_n
\p_n$. According to the first of the two properties that we required from the
coproducts the canonical morphism
%
$$\iota:\amalg_n (\wt{\U}_n,\p_n)\times_{\U_n}(\wt{\U}_n,\p_n)\sr
(\wt{\U}_*,\p_*)\times_{\U_*}(\wt{\U}_*,\p_*)$$
%
is an isomorphism. Together with the second property applied to the right-most
square this gives us a diagram with pullback squares of the form:
%
$$
\begin{xy}
          \xymatrix@C=4pc{
                 \amalg_n\wt{\U}_{n+1}  \ar[r]^-{=} \ar[d]_{r\circ \iota} & 
                 \amalg_n\wt{\U}_{n+1} \ar[r]^-{=} \ar[d]^{r} &
		 \amalg_n\wt{\U}_{n+1} \ar[r]^-{\langle in_{n+1} \rangle_{n}} \ar[d]^{r} &
		\wt{\U}_* \ar[d]^{\p_*} \\
		(\wt{\U}_*,\p_*)\times_{\U_*}(\wt{\U}_*,\p_*) \ar[r]^-{\iota^{-1}} &
		\amalg_n (\wt{\U}_n,\p_n)\times_{\U_n}(\wt{\U}_n,\p_n) \ar[r]^-{=} &
		\amalg_n \U_{n+1} \ar[r]^-{\langle in_{n+1} \rangle_{n}} &
		\U_*,
                }
\end{xy}
$$
%
where  $r=\amalg_n  \p_{n+1}$. Define  $Eq$  as  the  composition of  the  lower
horizontal arrows of  this diagram (up to an isomorphism  this is just $\langle
in_{n+1}\rangle_{n}$).  Since the squares of the diagram are pullback squares, the
natural morphism
%
\[\iota':\amalg_n\wt{\U}_{n+1}\sr
((\wt{\U}_*,\p_*)\times_{\U_*}(\wt{\U}_*,\p_*),Eq)_{\U_*} (\wt{\U}_*,\p_*)\]
%
is an isomorphism. Define
%
\[\Omega=(\amalg_n    i_{\Delta_n})\circ     \iota'\circ    \langle    in_{n+1}
\rangle_{n}\]
%
such that then
%
\[\omega=(\amalg_n i_{\Delta_n})\circ \iota'.\]
%
By our assumptions  $\omega\in TC$ and then by  Theorem \ref{2015.05.22.th1} if
$FB$  and  $TC$  satisfied  Conditions  \ref{2015.05.22.cond2}  or  by  Theorem
\ref{2015.05.16.th1}     if    $FB$     and    $TC$     satisfied    Conditions
\ref{2015.05.22.cond1} we conclude that $(Eq,\Omega)$ can be extended to a full
J-structure on $\p_*$.
\end{construction}
%






\subsection{Constructing a J-structure on $\toCC({\C},p)$ from a J-structure on~$p$}
%
The construction of a C-system $\gls{CCCp}$ from a category with a
universe $p$ and a final object $pt$ was presented in \cite{Cfromauniverse} and
summarized in \cite{fromunivwithPi}. Let us recall some facts and
notation. The underlying category of $\toCC({\C},p)$ is equipped with a
functor $\gls{int}$ to $\C$. Note that while $int$ is the identity function on
Hom-sets by the construction of $\toCC({\C},p)$, the notations for an
element of $Hom(\Gamma',\Gamma) = Hom(int(\Gamma'),int(\Gamma))$ may
differ.  In particular, for $f:\Gamma'\sr \Gamma$ in $\toCC({\C},p)$ and $F:int(\Gamma) \sr \U$ in $\C$,
we have the following equation.\editorfootnote{The
  notation $(\Gamma,F)$ refers to the inductive construction of objects of $\toCC({\C},p)$ presented
  in \cite{Cfromauniverse}.  Here $\Gamma$
  is an object of $\toCC({\C},p)$, so $(\Gamma,F)$ is a new object of $\toCC({\C},p)$, with
  $\ft(\Gamma,F) = \Gamma$ and $int(\Gamma,F) = (\Gamma;F)$.  In the equation displayed,
  the left side, $\q(f,(\Gamma,F))$, refers to the C-system structure on $\toCC({\C},p)$, while
  the right side, $\Q(f,F)$, refers to the universe structure on $\C$.  The equation follows from
  the definition \cite[2.6 (4)]{Cfromauniverse}.}
%
\begin{eq}
\label{2015.04.02.eq2} \q(f,(\Gamma,F)) = \Q(f,F).
\end{eq}%
%

For each $\Gamma\in Ob(\toCC({\C},p))$ we have natural bijections
%
\begin{eq}
\label{2015.03.27.eq7b} \gls{u1} = \gls{u1Gamma}:Ob_1(\Gamma) \xrightarrow\cong Hom(int(\Gamma),\U)
\end{eq}%
%
\begin{eq}
\label{2015.03.27.eq7a} \gls{utilde1} = \gls{utilde1Gamma}:\Obwt_1(\Gamma) \xrightarrow\cong Hom(int(\Gamma),\wt{\U}),
\end{eq}%
%
where $u_1^{-1}(F)=(\Gamma,F)$ and where
%
\begin{eq}
\label{2015.03.31.eq5} \wt{u}_1(s)=s\circ \Q(u_1(\partial(s))).
\end{eq}%
%
In particular,
%
$$\wt{u}_1(s)\circ p=s\circ \Q(u_1(\partial(s)))\circ p=s\circ
\p_{\partial(s)}\circ u_1(\partial(s))=u_1(\partial(s)),$$
%
i.e., with respect to these bijections the function
$\partial:\Obwt_1(\Gamma)\sr Ob_1(\Gamma)$ is given by composition with
$p:\wt{\U}\sr \U$.
%
\begin{problem}
\label{2015.03.27.prob3} Let $Eq:(\wt{\U};p)\sr \U$ be a J0-structure on a
universe $p$ in a category $\C$. To construct a J0-structure on
$\toCC({\C},p)$.
\end{problem}
%
\begin{construction}{2015.03.27.prob3}\rm
\label{2015.03.27.constr3} Since the canonical squares given by the universe structure on $p$ are pullback squares,
the bijections $u_1$ and $\wt{u}_1$ give us a bijection
%
$$\wt{uu}:\{o,o'\in\Obwt_1(\Gamma)\,|\,\partial(o)=\partial(o')\} \xrightarrow \cong
Hom(int(\Gamma),(\wt{\U};p)),$$
%
where $\wt{uu}(o,o')=\wt{u}_1(o)*\wt{u}_1(o')$. We set:
%
$$\Id(o,o') := u_1^{-1}(\wt{uu}(o,o')\circ Eq).$$
%
\end{construction}
%
We let $\gls{IdEq}$ denote the J0-structure on $\toCC({\C},p)$ constructed
from $Eq$ in Construction \ref{2015.03.27.constr3}. Note that
%
\begin{eq}
\label{2015.03.31.eq1}
int(\Id(o,o'))=(int(\Gamma);(\wt{u}_1(o)*\wt{u}_1(o'))\circ Eq).
\end{eq}%
and
\begin{eq}
  \label{IdEq-eqn}
  u_1(\Id(o,o')) = ( \wt{u}_1(o) * \wt{u}_1(o') ) \circ Eq
\end{eq}

%
Recall that in \cite{Csubsystems} we let $\gls{pGamman}:\Gamma\sr \ft^n(\Gamma)$
denote the composition of $n$ canonical projections $\p_{\Gamma}\circ \dots\circ
\p_{\ft^{n-1}(\Gamma)}$.
%
\begin{lemma}
\label{2015.03.27.l1} Let $Eq$ be a J0-structure on $p$. Let $\Gamma\in Ob$
and $F:int(\Gamma)\sr \U$. Then one has:
%
\begin{align}
  \Idx(\Gamma,F) &= (((X,F),\Q(F)\circ p), \Q(\Q(F),p)\circ Eq) \label{2015.03.27.l1-IDx} \\
  int(\Idx(\Gamma,F))&=(int(\Gamma);F)_{E} \label{2015.03.27.l1-eqn2} \\
  int(\p_{\Idx((\Gamma,F)),3}) &= \p^E_{\Gamma,F} \label{2015.03.27.l1-eqn3}  \\
  \Q(F)_{E}\circ \Q(Eq)&=\Q(\Q(\Q(F)\circ p)\circ Eq)
\end{align}
%
%
\end{lemma}
%
\begin{myproof}
It will be helpful for the reader to refer to diagram \ref{pEU-diagram}.
Let $T=(\Gamma,F)$, and set
\begin{align*}
  o &:=\p_{\p_T^*(T)}^*(\delta(T)) \in \Obwt_1(\Gamma) & \text{and}\\
  o'&:=\delta(\p_T^*(T)) \in \Obwt_1(\Gamma) .
\end{align*}
We then have
%
$$\Idx(T)=\Id_{\p_T^*(T)}(o,o')= (\p_T^*(T), (\wt{u}_1(o)*\wt{u}_1(o')) \circ Eq).$$
%
%
%
Furthermore, we have
%
$$\p_T^*(T) =((\Gamma,F),\Q(F)\circ p)$$
%
and
%
\begin{align*}
  \wt{u}_1(\p_{\p_T^*(T)}^*(\delta(T)))&=\p_{(int(\Gamma,F, \Q(F)\circ p)}\circ \Q(F) \\
  \wt{u}_1(\delta(\p_T^*(T)))&=\Q(\Q(F)\circ p)
\end{align*}
%
which shows that $\wt{u}_1(o)*\wt{u}_1(o')=\Q(\Q(F),p)$ and completes the proof
of the first three equations.

The last equality is a corollary of the equality $\Q(F)_{E}=\Q(\Q(\Q(F),p),Eq)$
and the equality $\Q(f,F)\circ \Q(F)=\Q(f\circ F)$.
\end{myproof}
%
\begin{problem}
\label{2015.03.27.prob4} Let $Eq:(\wt{\U};p)\sr \U$
be a J0-structure on a universe $p$ in a category $\C$,
and let $\Omega:\wt{\U}\sr \wt{\U}$ be a J1-structure over $Eq$.
To construct a J1-structure $\refl$ over $\Id_{Eq}$ on $\toCC({\C},p)$.
\end{problem}
%
\begin{construction}{2015.03.27.prob4}\rm
\label{2015.03.27.constr4} Due to the natural bijections
(\ref{2015.03.27.eq7a}) the morphism $\Omega$ defines maps
%
$$\refl:\Obwt_1(\Gamma)\sr \Obwt_1(\Gamma)$$
%
by the formula
%
\begin{eq}
  \label{refl-defn}
  \refl(s)=\wt{u}_1^{-1}(\wt{u}_1(s)\circ \Omega),
\end{eq}%
which are natural in $\Gamma$. The equation (\ref{2015.03.27.eq8}) follows from
the commutativity of the square (\ref{2015.03.27.sq1}).
\end{construction}
%
We let $\gls{reflOmega}$ denote the J1-structure constructed from $\Omega$ in
Construction \ref{2015.03.27.constr4}.
%
The following technical lemma is needed only in the proof of Lemma
\ref{2015.03.31.l2}.
%
\begin{lemma}
\label{2015.04.02.l3} For $s\in \Obwt_1(\Gamma)$ one has:
%
$$\refl_{\Omega}(s)\circ \Q(s\circ \Q(F)\circ \Omega\circ p)=s\circ \Q(F)\circ
\Omega,$$
%
where $F=u_1(\partial(s))$.
\end{lemma}
%
\begin{myproof}
We have
%
$$u_1(\partial(\refl_{\Omega}(s)))=\wt{u}_1(\refl(s))\circ p=\wt{u}_1(s)\circ
\Omega\circ p=s\circ \Q(F)\circ \Omega\circ p$$
%
therefore
%
\begin{align*}
  \wt{u}_1(\refl_{\Omega}(s))
  & =\refl_{\Omega}(s)\circ \Q(u_1(\partial(\refl_{\Omega}(s)))) \\
  & = \refl_{\Omega}(s)\circ \Q(s\circ \Q(F)\circ \Omega\circ p).
\end{align*}
%
On the other hand, by definition of $\refl_{\Omega}$,
%
$$\wt{u}_1(\refl_{\Omega}(s))=\wt{u}_1(s)\circ \Omega=s\circ \Q(F)\circ \Omega.$$
%
\end{myproof}



\begin{lemma}
\label{2015.03.31.l2} Given $Eq$ and $\Omega$ consider the corresponding $\Id$
and $\refl$. For $\Gamma \in Ob(\toCC({\C},p))$ and for $T\in Ob_1(\Gamma)$ let
$$\rf_T:int(T)\sr int(\Idx(T))$$
%
be the morphism constructed in Construction \ref{2015.03.27.constr2}. On the
other hand let
%
$$F^*(\omega):(int(\Gamma);F)\sr (int(\Gamma);F)_{E}$$
%
be the pullback of $\omega : \wt{\U}\sr E\wt{\U}$ (defined in \ref{2015.03.27.def5}) with respect to $F:=u_1(T)$,
i.e., the unique morphism
%
$$(int(\Gamma);F)\sr (int(\Gamma);F)_{E}$$
%
such that
%
\begin{align}
  F^*(\omega)\circ p^{E}_{int(\Gamma),F} & = \p_{int(T)}          \label{eqn1} \\
  F^*(\omega)\circ \Q(F)_{E}             & = \Q(F) \circ \omega   \label{eqn2}
\end{align}
%
Then
%
\[\rf_T=F^*(\omega).\]
%
\end{lemma}
%
\begin{myproof}
In view of Lemma \ref{2015.03.27.l1}, both $\rf_T$ and $F^*(\omega)$ are
morphisms from $(int(\Gamma);F)$ to $(int(\Gamma);F)_{E}$. Let us denote
$int(\Gamma)$ by $X$ and $(int(\Gamma);F,\Q(F)\circ p)$ by $Y$. We have
%
$$(X;F)_{E}=(Y;\Q(\Q(F),p)\circ Eq)$$
%
and we can see this object as a part of the diagram with two pullback squares:
%
\begin{eq} \label{h1h2squares}
  \xymatrix@C=4pc{
    (X;F)_{E} \ar[r]^-{h_1} \ar[d]_{\p_{Y,\Q(\Q(F),p)\circ Eq}} & E\wt{\U} \ar[r]^-{h_2} \ar[d]^{\p_{(\wt\U;p),Eq}} & \wt{\U} \ar[d]^{p} \\
      Y \ar[r]^-{\Q(\Q(F),p)} & (\wt{\U},p) \ar[r]^-{Eq} & \U
    },
\end{eq}
where $h_1 := \Q(F)_E$ and $h_2 := \Q(Eq)$.

%

Let $\Delta := (\id_{\wt{\U}})*(\id_{\wt{\U}})$ be the diagonal of $\wt{\U}$ over $\U$.

The following commutative diagram of canonical pullback squares clarifies some
of the forthcoming computations.

\begin{eq}
    \label{diag3}
    {
      \xymatrix{
                &                                & \wt \U \ar[d]^{\Delta}                                                         \\
                &  (X;F) \ar[ru]^{\Q(F)}           
                        \ar[d]_{\delta(T)}       & (\wt \U; p) \ar[dd]^(.33){\p_{\wt \U, p}}
                                                               \ar[rr]^{\Q(p)} &                                & \wt \U \ar[dd]^p \\
     int(\p_T^*(T))
       \ar[r]^-=
                &  (X;F,\Q(F) \circ p)
                     \ar[ru]^(.6){\Q(\Q(F),p)}         
                     \ar[dd]^-{\p_{\p_T^*(T)}}
                     \ar[rr]                     &                            & (X;F) \ar[dd]^(.3){\p_{(X;F)}}
                                                                                      \ar[ur]^(.4){\Q(F)}       &                  \\
                &                               & \wt \U \ar[rr]^(.33)p      &                                 & \U               \\
     int(T) 
       \ar[r]^-=
                & (X;F) \ar[rr]^{\p_{(X;F)}} 
                        \ar[ru]^{\Q(F)}           &                            & X \ar[ur]_F
      }}
\end{eq}


We have the following two projections.
%
\begin{align}
  h := h_1\circ h_2=\Q(\Q(\Q(F),p)\circ Eq)&:(X;F)_{E}\sr \wt{\U} \label{defnh} \\
  v := \p_{Y,\Q(\Q(F),p)\circ Eq}&:(X;F)_{E} \sr Y                \label{defnv}
\end{align}
%
It suffices to check the following equations.
%
\begin{align*}
  \rf_T\circ h &=F^*(\omega)\circ h \\
  \rf_T\circ v &=F^*(\omega)\circ v
\end{align*}
%
We have
%
\begin{align*}
  \rf_T \circ h
  & = \refl_\Omega(\delta(T))\circ \q(\delta(T),\Idx(T))         \circ h & \by{def'n \ref{2015.04.02.eq1}} \\
  & = \refl_\Omega(\delta(T))\circ \Q(\delta(T),\Q(\Q(F),p)      \circ Eq) \circ h & \by{\ref{2015.04.02.eq2} and \ref{2015.03.27.l1-IDx}} \\
  & = \refl_\Omega(\delta(T))\circ \Q(\delta(T),\Q(\Q(F),p)      \circ Eq) \circ \Q(\Q(\Q(F),p)\circ Eq) & \by{\ref{defnh}} \\
  & = \refl_\Omega(\delta(T))\circ \Q(\delta(T)\circ \Q(\Q(F),p) \circ Eq) & \by{\ref{2015.04.06.l0.sq}}\\
  & = \refl_\Omega(\delta(T))\circ \Q(\Q(F)\circ \Delta          \circ Eq) & \by{\ref{diag3}}\\
  & = \refl_\Omega(\delta(T))\circ \Q(\Q(F)\circ \Omega          \circ p), & \by{\ref{2015.03.27.sq1}}
\end{align*}
and thus we deduce that
\begin{eq}
  \label{eqnrfth}
  \rf_T \circ h = \refl_\Omega(\delta(T))\circ \Q(\Q(F)\circ \Omega \circ p).
\end{eq}

%

We have
%
\begin{align*}
        \delta(T) \circ \Q(u_1(\partial(\delta(T)))) 
    & = \delta(T) \circ \Q(u_1((X;F,\Q(F) \circ p))) & \by{def'n of $\partial$} \\
    & = \delta(T) \circ \Q(\Q(F) \circ p)            & \by{def'n of $u_1$} \\
    & = \delta(T) \circ \Q(\p_{X,F} \circ F)         & \by{\ref{univstrdiag}} \\
    & = \delta(T) \circ \Q(\p_{X,F},F) \circ \Q(F)   & \by{\ref{2015.04.06.l0.sq}} \\
    & = \id_{int(T)} \circ \Q(F)                     & \by{def'n of $\delta(T)$} \\
    & = \Q(F).                                       &
\end{align*}
%

In particular, we have the following equation, because $T = (X,F)$.
\[
  \delta(T) \circ \Q(\p_T \circ F) = \Q(F)
\]
Using that and applying Lemma \ref{2015.04.02.l3} with $s$ as $\delta(T)$ and $F$ as $p_T \circ F$,
we derive the following equation.
%
\[
  \refl_\Omega(\delta(T))\circ \Q(\Q(F)\circ \Omega\circ p)=\Q(F)\circ \Omega
\]
Combining that with \ref{eqnrfth}, we deduce that 
\[
  \rf_T \circ h = \Q(F)\circ \Omega.
\]

On the other hand, we have the following sequence of equations.
\begin{align*}
  F^*(\omega) \circ h & = F^*(\omega) \circ h_1 \circ h_2    & \by{def'n of $h$} \\
                      & = F^*(\omega) \circ \Q(F)_E \circ h_2 & \by{def'n of $h_1$} \\
                      & = \Q(F) \circ \omega \circ h_2       & \by{\ref{eqn2}} \\
                      & = \Q(F) \circ \omega \circ \Q(Eq)    & \by{def'n of $h_2$} \\
                      & = \Q(F) \circ \Omega                 & \by{\ref{diag4}}
\end{align*}

This proves that $\rf_T\circ h=F^*(\omega)\circ h$.

%% We have $\rf_T\circ v=\delta(T)$ because the square (\ref{2015.03.31.eq3})
%% commutes.

Both $\rf_T\circ v$ and $F^*(\omega)\circ v$ are morphisms
$int(T)\sr int(\p_T^*(T))$. Since $int(\p_T^*(T))$ is a part of a pullback square with the
projections being $\p_{\p_T^*(T)}$ and $\Q(\Q(F), p)$, to prove
$\rf_T\circ v = F^*(\omega)\circ v$ it suffices to verify the following equations.
\begin{align}
  \rf_T\circ v \circ \p_{\p_T^*(T)}   & = F^*(\omega)\circ v \circ \p_{\p_T^*(T)}   \label{eqn43}  \\
  \rf_T\circ v \circ \Q(\Q(F), p) & = F^*(\omega)\circ v \circ \Q(\Q(F), p) \label{eqn44}
\end{align}
Similarly, because the common target in the first equation above is $(X;F)$,
which is a pullback whose projections are $\p_T$ and $\Q(F)$, to verify \ref{eqn43} it suffices to
verify the following equations.
\begin{align}
  \rf_T\circ v \circ \p_{\p_T^*(T)} \circ \p_T   & = F^*(\omega)\circ v \circ \p_{\p_T^*(T)} \circ \p_T      \label{eqn45}  \\
  \rf_T\circ v \circ \p_{\p_T^*(T)} \circ \Q(F)  & = F^*(\omega)\circ v \circ \p_{\p_T^*(T)} \circ \Q(F)     \label{eqn46}  
\end{align}

We verify \ref{eqn45} as follows.
\begin{align*}
  \rf_T\circ v \circ \p_{\p_T^*(T)} \circ \p_T
  & = \delta(T) \circ \p_{\p_T^*(T)} \circ \p_T          & \by{\ref{2015.03.31.eq3}} \\
  & = \id_{int(T)} \circ \p_T                            & \by{def'n of $\delta(T)$} \\
  & = p_T                                                & \\
  & = F^*(\omega)\circ v \circ \p_{\p_T^*(T)} \circ \p_T & \by{def'n of $F^*(\omega)$}
\end{align*}

We verify \ref{eqn46} as follows.
\begin{align*}
  \rf_T \circ v \circ \p_{\p_T^*(T)} \circ \Q(F)
  & = \delta(T) \circ \p_{\p_T^*(T)} \circ \Q(F)         & \by{\ref{2015.03.31.eq3}} \\
  & = \id_{int(T)} \circ \Q(F)                           & \by{def'n of $\delta(T)$} \\
  & = \Q(F)                                              & \by{def'n of $\delta(T)$} \\
  & = \Q(F) \circ \Delta \circ \p_{\wt{\U},p}            & \by{def'n of $\Delta$} \\
  & = \Q(F) \circ \omega \circ \p_{(\wt\U;p),Eq} \circ \p_{\wt{\U},p}                        & \by{\ref{diag4}} \\
  & = F^*(\omega) \circ \Q(F)_E \circ \p_{(\wt\U;p),Eq} \circ \p_{\wt{\U},p}                 & \by{\ref{eqn2}} \\
  & = F^*(\omega) \circ h_1 \circ \p_{(\wt\U;p),Eq} \circ \p_{\wt{\U},p}                     & \by{def'n of $h_1$} \\
  & = F^*(\omega) \circ \p_{Y,\Q(\Q(F),p)\circ Eq} \circ \Q(\Q(F),p) \circ \p_{\wt{\U},p}    & \by{\ref{h1h2squares}} \\
  & = F^*(\omega) \circ v \circ \Q(\Q(F),p) \circ \p_{\wt{\U},p}                             & \by{\ref{defnv}} \\
  & = F^*(\omega) \circ v \circ \p_{\p_T^*(T)}    \circ \Q(F)           & \by{\ref{diag3}}
\end{align*}

We verify \ref{eqn44} as follows.
\begin{align*}
  \rf_T \circ v \circ \Q(\Q(F), p)
  & = \delta(T) \circ \Q(\Q(F), p)                                                  & \by{\ref{2015.03.31.eq3}} \\
  & = \Q(F) \circ \Delta                                                            & \by{\ref{diag3}} \\
  & = \Q(F) \circ \omega \circ \p_{(\wt\U;p),Eq}                                    & \by{\ref{diag4}} \\
  & = F^*(\omega) \circ \Q(F)_{E} \circ \p_{(\wt\U;p),Eq}                           & \by{\ref{eqn2}} \\
  & = F^*(\omega) \circ \Q(\Q(\Q(F),p),Eq) \circ \p_{(\wt\U;p),Eq}                  & \by{\ref{defQFE}} \\
  & = F^*(\omega) \circ \p_{Y,\Q(\Q(F),p) \circ Eq} \circ \Q(\Q(F), p)              & \by{\ref{pEU-diagram}} \\
  & = F^*(\omega) \circ v \circ \Q(\Q(F), p)                                        & \by{\ref{defnv}}
\end{align*}
This completes the proof of Lemma \ref{2015.03.31.l2}.
\end{myproof}
 
\comment{
\begin{problem}
\label{2015.05.04.prob2} Let $Eq$ be a J0-structure on a universe $p$ in $\cal
C$. To construct, for all objects $\Gamma\in Ob(\toCC({\C},p))$ two maps
%
\begin{align*}
  \psi & :\D_{pE\wt{\U}}(int(\Gamma),\U)\sr Ob(\toCC({\C},p)) \\
  \wt{\psi} & :\D_{pE\wt{\U}}(int(\Gamma),\wt{\U})\sr \Obwt(\toCC({\C},p))
\end{align*}
%
such that
%
\begin{eq}
\label{2015.05.04.eq3}
\partial(\wt{\psi}(d))=\psi(\D_{pE\wt{\U}}(int(\Gamma),p)(d))
\end{eq}%
%
%$\psi(F,\wt{G})\in \Obwt_1(\Idx(\Gamma,F))$.
\end{problem}
%
\begin{construction}{2015.05.04.eq3}
\label{2015.05.04.constr2}\rm Let us construct $\wt{\psi}$.  An element
$(F,\wt{G})$ of $\D_{pE\wt{\U}}(int(\Gamma),\wt{\U})$ is a pair where
$F:int(\Gamma)\sr \U$ and $\wt{G}:(int(\Gamma);F)_{E}\sr \wt{\U}$. By Lemma
\ref{2015.03.27.l1} we have
$(int(\Gamma),F)_{E}=int(\Idx(\Gamma,F))$. Therefore
$\wt{u}_{1,\Idx(\Gamma,F)}^{-1}(\wt{G})$ is defined and belongs to
$\Obwt_1(\Idx(\Gamma,F))$. We set
%
$$\wt{\psi}(F,\wt{G})=\wt{u}_{1,\Idx(\Gamma,F)}^{-1}(\wt{G})$$
%
Similarly we set
%
$$\psi(F,G)=u_{1,\Idx(\Gamma,F)}^{-1}(G)$$
%
The proof of the relation (\ref{2015.05.04.eq3}) is omitted.
\end{construction}
%
}
%
\begin{problem}
\label{2015.04.04.prob1} Let $(Eq,\Omega,Jp)$ be a J-structure on a universe
$p$. To construct for all $\Gamma\in Ob=Ob(\toCC({\C},p))$, for all $T\in
Ob_1(\Gamma)$, for all $P\in Ob_1(\Idx(T))$, for all $s0\in
\Obwt(\rf_T^*(P))$, an element $\J(\Gamma,T,P,s0)$ of $\Obwt(P)$.
(This is the type of element required by the definition \ref{2015.03.27.def3} of a J2-structure on the C-system $\toCC({\C},p)$.)
\end{problem} 
%
\begin{construction}{2015.04.04.prob1}\rm
\label{2015.04.04.constr1} Let $X:=int(\Gamma)$, $F:=u_1(T):X\sr \U$, so $T = (\Gamma,F)$. By Lemma
\ref{2015.03.27.l1} we have
$int(\Idx(T))=int(\Idx(\Gamma,F))=(X;F)_{E}$. Therefore we further
have $G:=u_1(P):(X;F)_{E}\sr \U$ and $\wt{H}:=\wt{u}_1(s0):(X;F)\sr
\wt{\U}$.

Let us show first that
%
$$\etaunshriek_{pE\wt{\U}}(F,G)\circ \I^{\omega}(\U)=\etaunshriek_p(F,\wt{H})\circ \I_p(p)$$
%
We show this as follows.
%
\begin{align*}
  \etaunshriek_{pE\wt{\U}}(F,&G) \circ \I^{\omega}(\U) \\
    & = \etaunshriek_p(F,G\circ F^*(\omega)) & \by{Lemma \ref{2015.04.02.l4}} \\
    & = \etaunshriek_p(F,\rf_T\circ G) & \by{Lemma \ref{2015.03.31.l2}}  \\
    & = \etaunshriek_p(F,s0\circ \Q(\rf_T\circ G)\circ p) & \by{commutativity of the canonical square}  \\
    & = \etaunshriek_p(F,\wt{H}\circ p) & \by{\ref{2015.03.31.eq5}}  \\
    & = \etaunshriek_p(F,\wt{H})\circ \I_p(p) & \by{naturality of $\etaunshriek_{p,X,V}$}.
\end{align*}

Therefore the pair $(\etaunshriek_{pE\wt{\U}}(F,G),\etaunshriek_p(F,\wt{H}))$ gives us a
morphism
%
$$\gls{phiGammaTPs0} : X \sr (\I_{pE\wt{\U}}(\U), \I^{\omega}(\U)) \times_{\I_p(\U)}
(\I_p(\wt{\U}),\I_p(p))$$
%
and composing it with $Jp$ (cf. Definition \ref{2015.03.27.def6}) we obtain a
morphism
%
$$\phi(\Gamma,T,P,s0)\circ Jp: X\sr \I_{pE\wt{\U}}(\wt{\U})$$
%
Consider the element
%
$$(F_1,F_2) := \etashriek_{pE\wt{\U}}(\phi(\Gamma,T,P,s0)\circ Jp)\in
\D_{pE\wt{\U}}(X,\wt{\U})$$
%
We have the following computation.
%
\begin{align*}
  (F_1,F_2\circ p)
  & = \D_{pE\wt{\U}}(X,p)(F_1,F_2)  & \by{def'n of $\D_{pE\wt{\U}}$} \\
  & = \D_{pE\wt{\U}}(X,p)(\etashriek_{pE\wt{\U}}(\phi(\Gamma,T,P,s0)\circ Jp))  & \by{def'n of $(F_1,F_2)$} \\
  & = \etashriek_{pE\wt{\U}}(\phi(\Gamma,T,P,s0)\circ Jp\circ \I_{pE\wt{\U}}(p))  & \by{naturality of $\etashriek_{pE\wt{\U}}$} \\
  & = \etashriek_{pE\wt{\U}}(\phi(\Gamma,T,P,s0)\circ Jp\circ coJ \circ \pr_1)  & \by{def'n of $coJ$} \\
  & = \etashriek_{pE\wt{\U}}(\phi(\Gamma,T,P,s0)\circ \pr_1)  & \by{\ref{2015.03.27.def6}} \\
  & = \etashriek_{pE\wt{\U}}(\etaunshriek_{pE\wt{\U}}(F,G)) & \by{def'n of $\phi(\Gamma,T,P,s0)$} \\
  & =(F,G)
\end{align*}

(For naturality of $\etashriek_{pE\wt{\U}}$, refer to \cite[Problem 3.8(1)]{fromunivwithPi}.)

Therefore, $F_1 = F$ and $F_2 \circ p = G$.
Thus $F_2$ is of type $(X;F)_E\sr \wt{\U}$, and thus it is of the form
$F_2 = \wt{u}_{1}(\J(\Gamma,T,P,s0)))$ for some unique $\J(\Gamma,T,P,s0)$ such that $\partial(\J(\Gamma,T,P,s0)) = u_1^{-1}(F_2 \circ p)= u_1^{-1}(G) = P$.
\end{construction}

\begin{remark}
  When more than one J2-structure $Jp$ is under consideration, we may write $\J_{Jp} = \J_{Jp}(\Gamma,T,P,s0)$ instead of $\J = \J(\Gamma,T,P,s0)$
  to indicate which J2-structure is involved in the construction of $\J$.
\end{remark}

\begin{remark}\rm
\label{2015.05.08.rem1} Note that the defining property of
$\J := \J(\Gamma,T,P,s0)$ is that it is the unique element of
$\Obwt(\toCC({\C},p))$ that satisfies the equation
%
$$\etaunshriek_{pE\wt{\U}}(u_{1,\Gamma}(T),\wt{u}_{1,\Idx(T)}(\J))=\phi(\Gamma,T,P,s0)\circ
Jp,$$
%
where
%
$$\phi(\Gamma,T,P,s0):int(\Gamma)\sr
(\I_{pE\wt{\U}}(\U),\I^{\omega}(\U))\times_{\I_p(\U)}(\I_p(\wt{\U}),\I_p(p))$$
%
is given by the pair of morphisms $(\etaunshriek_{pE\wt{\U}}(u_{1,\Gamma}(T),
u_{1,\Idx(T)}(P)), \etaunshriek_p(u_{1,\Gamma}(T),\wt{u}_{1,\Gamma}(s0)))$.
\end{remark}
%




\begin{lemma}
\label{2015.04.04.l4} Let $Eq$ be a J0-structure on a universe $p$,
$f:\Gamma'\sr \Gamma$ a morphism in $\toCC({\C},p)$ and $F:int(\Gamma)\sr \U$ a morphism in $\C$.
Let $q3:int(\Idx(\Gamma',f\circ F))\sr int(\Idx(\Gamma,F))$ be the morphism $\q(f,\Idx(\Gamma,F),3)$ defined\footnote{The notation $\gls{qfXi}$
is defined in section 3 of \cite{Csubsystems}, by induction on the third parameter, which is a natural number.} by
$\Gamma$, using $\ft^3(\Idx(\Gamma,F))=\Gamma$.  Then $q3=\Q(f,F)_{E}$.
\end{lemma}
%
\begin{myproof}
Let $X:=int(\Gamma)$ and $X':=int(\Gamma')$. By definition, $\Q(f,F)_{E}$ is the
unique morphism such that
%
\begin{align*}
  \Q(f,F)_{E}\circ \Q(F)_{E}&=\Q(f\circ F)_{E} & \text{and} \\
  \Q(f,F)_{E}\circ p^{E}_{X,F}&=p^{E}_{X',f\circ F}\circ f.
\end{align*}
%
We will be building the proof using the following diagram.
%
$$
\begin{xy}
          \xymatrix@C=4pc{ (X',f\circ F)_{E} \ar[rrr]^-{\Q(\Q(\Q(f,F),\Q(F)\circ
              p),\Q(\Q(F),p)\circ Eq)} \ar[d]_{p_3} &&& (X;F)_{E}
            \ar[rr]^-{\Q(\Q(\Q(F),p),Eq)} \ar[d] && E\wt{\U} \ar[r]^-{\Q(Eq)}
            \ar[d]^{p_1} & \wt{\U} \ar[d]^{p}\\ \BB
            \ar[rrr]^-{\Q(\Q(f,F),\Q(F)\circ p)} \ar[d]^{=} &&& \BB
            \ar[rr]^-{\Q(\Q(F),p)} \ar[d]^{=} && \BB \ar[r]^-{Eq} \ar[d]^{=}&
            \U\\ \BB \ar[rrr]^-{\Q(\Q(f,F),\Q(F)\circ p)} \ar[d] &&& \BB
            \ar[rr]^-{\Q(\Q(F),p)} \ar[d] && \BB \ar[r]^-{\Q(p)} \ar[d]^{p_2}&
            \wt{\U} \ar[d]^{p}\\ \BB \ar[rrr]^-{\Q(f,F)} \ar[d]^{=}&&& \BB
            \ar[rr]^-{\Q(F)} \ar[d]^{=} && \wt{\U} \ar[r]^-{p} \ar[d]^{=} & \U
            \\ \BB \ar[rrr]^-{\Q(f,F)} \ar[d] &&& \BB \ar[rr]^-{\Q(F)} \ar[d] &&
            \BB \ar[d]^{p}\\ X' \ar[rrr]^-{f} &&& X \ar[rr]^-{F} && \U }
\end{xy}
$$
%
By construction that is seen on this diagram we have:
%
\begin{align*}
  q3&=\Q(\Q(\Q(f,F),\Q(F)\circ p),\Q(\Q(F),p)\circ Eq) \\
  \Q(X,F)_{E}&=\Q(\Q(\Q(F),p),Eq)
\end{align*}
%
and
%
$$\Q(X', f\circ F)_{E}=\Q(\Q(\Q(f\circ F),p),Eq)$$
%
Therefore, the first equation that we need to verify is
%
$$\Q(\Q(\Q(f,F),\Q(F)\circ p),\Q(\Q(F),p)\circ Eq)\circ \Q(\Q(\Q(F),p),Eq)=\Q(\Q(\Q(f\circ
F),p),Eq)$$
%
By \cite[Lemma 3.2]{fromunivwithPi} we have, together with the defining rule
$\Q(a,A)\circ \Q(A)=\Q(a\circ A)$, also the rule:
%
$$\Q(a_1,a_2\circ A)\circ \Q(a_2,A)=\Q(a_1\circ a_2, A)$$
%
Applying it twice and then the defining rule we get:
%
\begin{align*}
  \Q(\Q(\Q(f,F),&\Q(F)\circ p),\Q(\Q(F),p)\circ Eq)\circ \Q(\Q(\Q(F),p),Eq) \\
    & = \Q(\Q(\Q(f,F),\Q(F)\circ p)\circ \Q(\Q(F),p), Eq) \\
    & = \Q(\Q(\Q(f,F)\circ \Q(F),p),Eq) \\
    & = \Q(\Q(\Q(f\circ F),p),Eq),
\end{align*}
%
which gives us the first equation. The second equation is immediate from the
commutativity of the three squares that define $q3$.
\end{myproof}
%
\begin{lemma}
\label{2015.04.04.l1} Let $(Eq,\Omega,Jp)$ be a J-structure on a universe
$p$. Then the morphisms of Construction \ref{2015.04.04.constr1} are natural in
$\Gamma$, i.e., for any $f:\Gamma'\sr \Gamma$ one has
%
\begin{eq}\label{2015.04.04.eq3}
f^*(\J_{Jp}(\Gamma,T,P,s0))=\J_{Jp}(\Gamma',f^*(T),f^*(P),f^*(s0)).
\end{eq}%
(This is the second condition of the definition \ref{2015.03.27.def3} of a J2-structure on the C-system $\toCC({\C},p)$.)
\end{lemma}
%
\begin{myproof}
Let us write $\J$ for $\J_{Jp}(\Gamma,T,P,s0)$ and $\J'$ for
$\J_{Jp}(\Gamma',f^*(T),f^*(P),f^*(s0))$ and use the notation of Construction
\ref{2015.04.04.constr1}. Recall that for $f:\Gamma'\sr \Gamma$ the operation
$f^*$ is defined only on $Ob_1(\Gamma)$. In all other uses it is an
abbreviation for the operation $X\mapsto \gls{fstarXi}$ or the operation $s\mapsto \gls{fstarsi}$,
for various values of $i$; these operations are defined in \cite[\S 3]{Csubsystems} by induction on $i$.
In particular, (\ref{2015.04.04.eq3}) is an abbreviation for
%
$$f^*(\J(\Gamma,T,P,s0),4)=\J(\Gamma',f^*(T),f^*(P,4),f^*(s0,2))$$
%
which in its turn translates into the equation in $\Obwt_1(\Idx(f^*(T)))$ of
the form
%
$$\q(f,\Idx(T),3)^*(\J,1)=\J'$$
%
We have:
%
\begin{align*}
  \etaunshriek_{pE\wt{\U}}(F,\wt{u}_1(\J))&=\phi(\Gamma,T,P,s0)\circ Jp \\
  \etaunshriek_{pE\wt{\U}}(f\circ F, \wt{u}_1(\J'))&=\phi(\Gamma',f^*(T),f^*(P),f^*(s0))\circ Jp
\end{align*}
%
By naturality of $\etaunshriek$ with respect to the first argument we have
%
$$f\circ \etaunshriek_{pE\wt{\U}}(F,\wt{u}_1(\J))=\etaunshriek_{pE\wt{\U}}(f\circ F,
\Q(f,F)_{E}\circ \wt{u}_1(\J))$$
%
Therefore, by Lemma \ref{2015.04.04.l4} we have
%
\begin{align*}
  f\circ \etaunshriek_{pE\wt{\U}}(F,\wt{u}_1(\J))
    & = \etaunshriek_{pE\wt{\U}}(f\circ F, \wt{u}_1(\Q(f,F)_{E}^*(\J,1))) \\
    & = \etaunshriek_{pE\wt{\U}}(f\circ F, \wt{u}_1(\q(f,\Idx(T),3)^*(\J,1)))
\end{align*}
%
Since both $\etaunshriek_{pE\wt{\U}}$ and $\wt{u}_1$ are bijections and thus
injections, it is sufficient to show that
%
$$f\circ \phi(\Gamma,T,P,s0)\circ Jp = \phi(\Gamma',f^*(T),f^*(P),f^*(s0))\circ
Jp$$
%
or that
%
$$f\circ \phi(\Gamma,T,P,s0)=\phi(\Gamma',f^*(T),f^*(P),f^*(s0))$$
%
Since both $\phi$ expressions are morphism into a product this amounts to two
equations that, taking into account the definition of $\phi$ in Construction
\ref{2015.04.04.constr1} are:
%
$$f\circ \etaunshriek_{pE\wt{\U}}(F,G)=\etaunshriek_{pE\wt{\U}}(f\circ F, u_1(f^*(P)))$$
%
and
%
$$f\circ \etaunshriek_p(F,\wt{H})=\etaunshriek_p(f\circ F, \wt{u}_1(f^*(s0)))$$
%
The first equality follows from naturality of $\etaunshriek$ and Lemma
\ref{2015.04.04.l4}. The second equality follows from naturality of
$\etaunshriek$. This finishes the proof of Lemma \ref{2015.04.04.l1}.
\end{myproof}
%
\begin{lemma}
\label{2015.04.04.l5} Let $(Eq,\Omega,Jp)$ be a J-structure on a universe
$p$. Then the morphism of Construction \ref{2015.04.04.constr1} satisfies the
first condition of the definition \ref{2015.03.27.def3} of a J2-structure on the C-system $\toCC({\C},p)$, i.e., for all $\Gamma$,
$T$, $P$ and $s0$ as above one has
%
$$\rf_T^*(\J_{Jp}(\Gamma,T,P,s0))=s0$$
%
\end{lemma}
%
\begin{myproof}
Let $\J=\J_{Jp}(\Gamma,T,P,s0)$. Then, using the notation of Construction
\ref{2015.04.04.constr1} we have
%
$$\etaunshriek_{E\wt{\U}}(F,\wt{u}_1(\J)) = \etaunshriek_{E\wt{\U}}(F_1,F_2) =\phi(\Gamma,T,P,s0)\circ Jp$$
%
Observe that
%
$$\etaunshriek_{E\wt{\U}}(F,\wt{u}_1(\J))\circ
\I^{\omega}(\wt{\U})=\etaunshriek_p(F,F^*(\omega)\circ \wt{u}_1(\J)),$$
by naturality of $\etashriek$ with respect to change of universe, established in \ref{2015.04.02.l4}.
%
By Lemma \ref{2015.03.31.l2} we have $F^*(\omega)=\rf_T$. Therefore,
%
$$\etaunshriek_{E\wt{\U}}(F,\wt{u}_1(\J))\circ \I^{\omega}(\wt{\U})=\etaunshriek_p(F,\rf_T\circ
\wt{u}_1(\J))=\etaunshriek_p(F,\wt{u}_1(\rf_T^*(\J)))$$
%
On the other hand, by \ref{2015.04.04.eq1},
%
$$\phi(\Gamma,T,P,s0) \circ Jp\circ \I^{\omega}(\wt{\U})=\phi(\Gamma,T,P,s0) \circ \pr_2$$
%
which equals, by construction,
$\etaunshriek_p(F,\wt{u}_1(s0))$. Therefore,
%
$$\etaunshriek_p(F,\wt{u}_1(\rf_T^*(\J)))=\etaunshriek_p(F,\wt{u}_1(s0))$$
%
and using again that both $\etaunshriek$ and $\wt{u}_1$ are injective we conclude that
$\rf_T^*(\J)=s0$.
\end{myproof}
%
\begin{problem}
\label{2015.04.04.prob2} Let $(Eq,\Omega,Jp)$ be a J-structure on a universe
$p$. To construct a J-structure on $\toCC({\C},p)$ relative to $\Id_{Eq}$
and $\refl_{\Omega}$.
\end{problem} 
%
\begin{construction}{2015.04.04.prob2}\rm
\label{2015.04.04.constr2} One has to combine Construction
\ref{2015.04.04.constr1} with Lemmas \ref{2015.04.04.l1} and
\ref{2015.04.04.l5}.
\end{construction}
%





\section{Functoriality of J-structures}




\subsection{A theorem about functors between categories with two universes}
\label{twouniv}

Before we can formulate the definition of what it means for a universe category
functor to be compatible with J-structures we need some general results about
functors between categories with two universes, which we will later apply to the
universes $p:\wt{\U}\sr \U$ and $pE\wt{\U}:E\wt{\U}\sr \U$ in a locally cartesian
closed category $\cal C$.

Given two universes 
$p:\wt{\U}\sr \U$ and $p':\wt{\U}'\sr \U$, with canonical squares of the form
%
$$
\begin{xy}
          \xymatrix@C=4pc{ (X;F) \ar[r]^-{\Q(F)} \ar[d]_{\p_{X,F}} & \wt{\U}
            \ar[d]^{p}\\ X \ar[r]^{F} & \U }
\end{xy}
\spc
\begin{xy}
          \xymatrix@C=4pc{ (X;F)' \ar[r]^-{\Q'(F)} \ar[d]_{p'_{X,F}} & \wt{\U}'
            \ar[d]^{p'}\\ X \ar[r]^{F} & \U }
\end{xy}
$$
%
and given $f:\wt{\U}'\sr \wt{\U}$ over $\U$, we let $\gls{Fstarf}$ denote the unique morphism
$(X;F)'\sr (X;F)$ such that
%
\begin{eq}\label{2015.04.08.eq3}
F^*(f)\circ \Q(F)=\Q'(F)\circ f
\end{eq}%
%
\begin{eq}\label{2015.04.08.eq4}
F^*(f)\circ \p_{X,F}=\p_{X,F}'
\end{eq}%
%
Note that $F^*(f)$ depends on the universe structures on $p$ and $p'$. Even
when two universe structures give the same choices of the objects $(X;F)$ and
$(X;F)'$, the difference in the choice of some of the morphisms, e.g., $\Q(F)$
will affect morphisms $F^*(f)$. We will need the following lemma about these
morphisms.

As in \ref{QfF-defn}, for $X'\stackrel{f}{\sr}X \stackrel{F}{\sr}\U$ we let $\Q(f,F)$ denote the
morphism
%
$$(\p_{X',f\circ F}\circ f)*\Q(f\circ F):(X';f\circ F)\sr (X;F)$$
%
We let $\Q'(-)$ and $\Q'(-,-)$ denote the morphisms $\Q(-)$ and $\Q(-,-)$ relative
to the universe $p'$.
%
\begin{lemma}
\label{2015.04.20.l1} Let $X'\stackrel{g}{\sr}X\stackrel{F}{\sr}\U$ be two
morphisms. Then the square
%
$$
\begin{xy}
          \xymatrix@C=4pc{ (X';g\circ F)' \ar[r]^-{\Q'(g,F)} \ar[d]_{(g\circ
              F)^*(f)} & (X;F)' \ar[d]^{F^*(f)}\\ (X';g\circ F) \ar[r]^{\Q(g,F)}
            & (X';g\circ F) }
\end{xy}
$$
%
commutes.
%
\end{lemma}
%
\begin{myproof}
Since $(X;F)$ is a fiber product relative to the projections $\p_{X,F}$ and
$\Q(F)$ it is sufficient to verify that
%
$$\Q'(g,F)\circ F^*(f)\circ \Q(F)=(g\circ F)^*(f) \circ \Q(g,F)\circ \Q(F)$$
%
and
%
$$\Q'(g,F)\circ F^*(f)\circ \p_{X,F}=(g\circ F)^*(f) \circ \Q(g,F)\circ \p_{X,F}$$
%
which easily follows from the defining equations for $\Q(-,-)$ and $(-)^*$.
\end{myproof}

Let $({\C},p,pt)$, $({\C}',p',pt')$ be two universe categories
such that $\C$ and $\C'$ are equipped with locally cartesian
closed structures.
Consider now a \emph{functor of universe categories} $${\bf\Phi}=(\Phi,\phi,\wt{\phi}):({\C},p,pt)\sr ({\mathcal C}',p',pt').$$
The notion was defined in \cite[4.1]{Cfromauniverse} and in \cite[\S 5]{fromunivwithPi}.  It means that $\Phi : \C \to \C'$ is a functor,
$\phi : \Phi(\U) \to \U'$ and $\wt\phi : \Phi(\wt\U) \to \wt\U'$ are morphisms,
that $\Phi$ sends the chosen final object to a final object,
that $\Phi$ takes canonical squares of $p$ to pullback squares, and that the square
%
$$
\begin{xy}
  \xymatrix@C=4pc{
    \Phi(\wt{\U}) \ar[r]^-{\wt{\phi}} \ar[d]_{\Phi(p)}     & \wt{\U}' \ar[d]^{p'} \\
    \Phi(\U) \ar[r]^{\phi}                                 & \U' }
\end{xy}
$$
%
is a pullback square.
%
In \cite[Construction 5.2]{fromunivwithPi} we have constructed, for any map $F : X \to V$ in $\C$, a canonical isomorphism 
$$\iota = \gls{iotaPhiXF} : (\Phi(X) ; \Phi(F) \circ \phi)' \xrightarrow\cong \Phi((X;F)),$$
which results from the two objects involved being pullbacks of the same diagram, as illustrated here:
\begin{eq}
  \label{iota-diagram}
  \begin{xy}
    \xymatrix@C=5pc{
      (\Phi(X);\Phi(F)\circ\phi)'
      \ar@/^1.5pc/[rrrd]^{\Q'(\Phi(F)\circ\phi)}
      \ar@/_1.5pc/[rdd]_{\p'_{\Phi(X),\Phi(F)\circ\phi}}
      \ar[rd]^{\iota}_{\cong}                 \\
      & \Phi((X;F)) \ar[r]_-{\Phi(\Q(F))} \ar[d]^{\Phi(\p_{X,F})} & \Phi(\wt\U) \ar[r]_{\wt\phi} \ar[d]^{\Phi(p)} & \wt\U' \ar[d]^{p'} \\
      & \Phi(X) \ar[r]_{\Phi(F)} & \Phi(U) \ar[r]_{\phi} & \U'
      }
    \end{xy}
\end{eq}
%
In \cite[Construction 5.2]{fromunivwithPi} we have defined, for any $X, V\in {\C}$, a map
$$\gls{Phi2} : \D_{p}(X,V) \sr \D_p(\Phi(X),\Phi(V)),$$
by setting
\begin{eq}
  \label{defn-Phi2}
        {\bf\Phi}^2(F,G) := (\Phi(F) \circ \phi, \iota_{{\bf\Phi},X,F} \circ \Phi(G)).
\end{eq}
In \cite[Construction 5.6]{fromunivwithPi} we have also defined
a morphism
$$\gls{chiPhiV} : \Phi(\I_p(V))\sr \I_{p'}(\Phi(V)),$$
by setting
$$\chi_{\bf\Phi}(V) := \etaunshriek_{p',\Phi(\I_p(V)),\Phi(V)} \big(  {\bf\Phi}^2  \big( \etashriek_{p,\I_p(V),V} ( 1_{\I_p(V)} ) \big)\big).$$
These constructions will be used later.

We now need to consider the situation where we have the following collection of data:
%
\begin{enumerate}
\item two universes $p_1$, $p_2$ in $\C$ with the common codomain $\U$
  and a morphism $g:\wt{\U}_1\sr \wt{\U}_2$ over $\U$,
\item two universes $p_1'$, $p_2'$ in $\C'$ with the common codomain
  $\U'$ and a morphism $g':\wt{\U}_1'\sr \wt{\U}_2'$ over $\U'$,
\item a functor $\Phi:{\C}\sr {\C}'$,
\item a morphism $\phi:\Phi(\U)\sr \U'$,
\item two morphisms $\wt{\phi}_i:\Phi(\wt{\U}_i)\sr \wt{\U}_i'$, $i=1,2$, and
\item final objects $pt$ of $\C$ and $pt'$ of $\C'$,
\end{enumerate}
%
and this data is such that:
%
\begin{enumerate}
\item the square
%
\begin{eq}\label{two-univ-comm}
\begin{xy}
          \xymatrix@C=4pc{ \Phi(\wt{\U}_1) \ar[r]^-{\wt{\phi}_1}
            \ar[d]_{\Phi(g)} & \wt{\U}_1' \ar[d]^{g'}\\ \Phi(\wt{\U}_2)
            \ar[r]^{\wt{\phi}_2} & \wt{\U}_2' }
\end{xy}
\end{eq}
%
commutes, and
%
\item the triples ${\bf\Phi}_i:=(\Phi,\phi,\wt{\phi}_i)$, for $i=1,2$, are universe
  category functors from $(\C,p_i,ptk)$ to $(\C',p'_i,pt')$.
\end{enumerate}

We will use the notations $(X;F)_1$, $(X;F)_2$, $(X;F)'_1$, and $(X;F)'_2$ for
the pullback objects that are part of the four universe structures under
consideration. 

Let us denote the morphisms
%
$$\chi_{{\bf\Phi}_i}(V):\Phi(\I_{p_i}(V))\sr \I_{p'_i}(\Phi(V))$$
%
by $\chi_i(V)$. The maps ${\bf\Phi}_i^2$ in the following lemma were introduced above.
%
\begin{lemma}
\label{2015.04.08.l1} Under the previous assumptions and notation the squares
%
$$
\begin{xy}
          \xymatrix@C=4pc{ \D_{p_2}(X,V) \ar[r]^-{{\bf\Phi}_2^2}
            \ar[d]_{D^g(X,V)} & \D_{p_2}(\Phi(X),\Phi(V))
            \ar[d]^{D^{g'}(\Phi(X),\Phi(V))}\\ \D_{p_1}(X,V)
            \ar[r]^-{{\bf\Phi}_1^2} & \D_{p_1'}(\Phi(X),\Phi(V)) }
\end{xy}
$$
%
commute.
\end{lemma}
%
\begin{myproof}
We will use $\iota_i$ as an abbreviation for an isomorphism $\iota$ derived from ${\bf\Phi}_i$.
Given $(F_1,F_2)\in \D_{p_2}(X,V)$, we see that:
%
\begin{align*}
  D^{g'}(\Phi(X),&\Phi(V))({\bf\Phi}_2^2(F_1,F_2)) \\
    & = D^{g'}(\Phi(X),\Phi(V))(\Phi(F_1)\circ \phi, \iota_2\circ \Phi(F_2)) & \by{def'n of ${\bf\Phi}_2^2$} \\
    & = (\Phi(F_1)\circ \phi, (\Phi(F_1)\circ \phi)^*(g')\circ \iota_2\circ \Phi(F_2)) & \by{def'n of $D^{g'}$ in \ref{2015.04.02.l4}}
\end{align*}

On the other hand
%
\begin{align*}
  {\bf\Phi}_1^2(D^g(X,V)(F_1,F_2)) 
    & = {\bf\Phi}_1^2(F_1,F_1^*(g)\circ F_2)  & \by{def'n of $D^g$ in \ref{2015.04.02.l4}} \\
    & = (\Phi(F_1)\circ\phi,\iota_1\circ \Phi(F_1^*(g)\circ F_2)) & \by{def'n of ${\bf\Phi}_1^2$} 
\end{align*}
%
Thus it remains to check that
%
$$(\Phi(F_1)\circ \phi)^*(g')\circ \iota_2\circ \Phi(F_2)=\iota_1\circ
\Phi(F_1^*(g)\circ F_2).$$
%
For that it is sufficient to check that
%
$$(\Phi(F_1)\circ \phi)^*(g')\circ \iota_2=\iota_1\circ \Phi(F_1^*(g)).$$
%
The codomain of both morphisms is $\Phi((X;F_1)_2)$, and since $\Phi$ takes
canonical squares based on $p_2$ to pullback squares it is sufficient to check
that
%
$$(\Phi(F_1)\circ \phi)^*(g')\circ \iota_2\circ
\Phi(\Q_2(F_1))\circ\wt{\phi}_2=\iota_1\circ \Phi(F_1^*(g))\circ
\Phi(\Q_2(F_1))\circ\wt{\phi}_2$$
%
and
%
$$(\Phi(F_1)\circ \phi)^*(g')\circ \iota_2\circ \Phi(\p_{2,X,F_1})=\iota_1\circ
\Phi(F_1^*(g))\circ \Phi(\p_{2,X,F_1})$$
%

We prove the first equation as follows.
%
\begin{align*}
  (\Phi(F_1)\circ \phi&)^*(g')\circ \iota_2\circ   \Phi(\Q_2(F_1))\circ\wt{\phi}_2 \\
  & =(\Phi(F_1)\circ \phi)^*(g')\circ \Q'_2(\Phi(F_1)\circ \phi) & \by{commutativity of \ref{iota-diagram}} \\
  & =\Q'_1(\Phi(F_1)\circ\phi)\circ g' & \by{\ref{2015.04.08.eq3}} \\
  & = \iota_1\circ \Phi(\Q_1(F_1))\circ \wt{\phi}_1\circ g' & \by{commutativity of \ref{iota-diagram}}  \\
  & = \iota_1\circ \Phi(\Q_1(F_1))\circ \Phi(g)\circ \wt{\phi}_2 & \by{\ref{two-univ-comm}} \\
  & = \iota_1\circ \Phi(\Q_1(F_1)\circ g)\circ \wt{\phi}_2  & \by{functoriality of $\Phi$} \\
  & = \iota_1\circ \Phi(F_1^*(g)\circ \Q_2(F_1))\circ \wt{\phi}_2 & \by{\ref{2015.04.08.eq3}} \\
  & = \iota_1\circ \Phi(F_1^*(g))\circ \Phi(\Q_2(F_1))\circ\wt{\phi}_2 & \by{functoriality of $\Phi$}
\end{align*}

The second equation is proved as follows.
%
\begin{align*}
  (\Phi(F_1)\circ \phi&)^*(g')\circ \iota_2\circ \Phi(\p_{2,X,F_1}) \\
  & = (\Phi(F_1)\circ \phi)^*(g')\circ \p'_{2,\Phi(X),\Phi(F_1)\circ\phi} & \by{commutativity of \ref{iota-diagram}} \\
  & = \p'_{1,\Phi(X),\Phi(F_1)\circ\phi} & \by{\ref{2015.04.08.eq4}} \\
  & = \iota_1\circ \Phi(\p_{1,X,F_1}) & \by{commutativity of \ref{iota-diagram}} \\
  & = \iota_1\circ \Phi(F_1^*(g)\circ \p_{2,X,F_1}) & \by{\ref{2015.04.08.eq4}} \\
  & = \iota_1\circ \Phi(F_1^*(g))\circ \Phi(\p_{2,X,F_1}) & \by{functoriality of $\Phi$}
\end{align*}

This finishes the proof of Lemma \ref{2015.04.08.l1}.
\end{myproof}

\begin{lemma}
\label{2015.04.06.l7} Under the previous assumptions and notation the squares
%
$$
\begin{xy}
          \xymatrix@C=4pc{ \Phi(\I_{p_2}(V)) \ar[r]^-{\chi_2(V)}
            \ar[d]_{\Phi(\I^g(V))} & \I_{p_2'}(\Phi(V))
            \ar[d]^{\I^{g'}(\Phi(V))}\\ \Phi(\I_{p_1}(V)) \ar[r]^{\chi_1(V)} &
            \I_{p_1'}(\Phi(V)) }
\end{xy}
$$
%
commute.
\end{lemma}
%
\begin{myproof}
Let $X=\I_{p_2}(V)$.  Then we have:
%
\begin{align*}
  \chi_2(V) \circ \I^{g'}&(\Phi(V)) \\
  & = \etaunshriek_{p'_2,\Phi(X),\Phi(V)}({\bf\Phi}_2^2(\etashriek_{p_2,X,V}(\id_X))) \circ \I^{g'}(\Phi(V)) & \by{definition of $\chi_2(V)$} \\
  & = \etaunshriek_{p'_1,\Phi(X),\Phi(V)}(D^{g'}(\Phi(X),\Phi(V))({\bf\Phi}_2^2(\etashriek_{p_2,X,V}(\id_X)))) & \by{\ref{2015.04.02.l4}} \\
  & = \etaunshriek_{p'_1,\Phi(X),\Phi(V)}({\bf\Phi}_1^2(D^g(X,V)(\etashriek_{p_2,X,V}(\id_X)))) & \by{\ref{2015.04.08.l1}} \\
  & = \etaunshriek_{p'_1,\Phi(X),\Phi(V)}({\bf\Phi}_1^2(\etashriek_{p_1,X,V}(\id_X \circ \I^g(V)))  & \by{\ref{2015.04.02.l4}} \\
  & = \etaunshriek_{p'_1,\Phi(X),\Phi(V)}({\bf\Phi}_1^2(\etashriek_{p_1,X,V}(\I^g(V)))
\end{align*}

It remains to show that
%
$$\Phi(\I^g(V))\circ \chi_1(V) = \etaunshriek_{p'_1,\Phi(X),\Phi(V)}({\bf\Phi}_1^2(\etashriek_{p_1,X,V}(\I^g(V)))$$
%
Let $a$ be any element of $Hom(\I_{p_2}(V),\I_{p_1}(V))$.  It will suffice to show that
%
$$\Phi(a)\circ \chi_1(V) = \etaunshriek_{p'_1,\Phi(X),\Phi(V)}({\bf\Phi}_1^2(\etashriek_{p_1,X,V}(a)).$$
%
We have
%
\begin{align*}
  \Phi&(a)\circ \chi_1(V) \\
    & = \Phi(a) \circ \etaunshriek_{p'_1,\Phi(\I_{p_1}(V)),\Phi(V)}({\bf\Phi}^2_1(\etashriek_{p_1,\I_{p_1}(V),V}(\id_{\I_{p_1}(V)})))              & \by{def'n of $\chi_1(V)$} \\
    & = \etaunshriek_{p'_1,\Phi(\I_{p_2}(V)),\Phi(V)}(\D_{p'_1}(\Phi(a),\Phi(V))({\bf\Phi}^2_1(\etashriek_{p_1,\I_{p_1}(V),V}(\id_{\I_{p_1}(V)}))) & \by{naturality of $\etaunshriek$}  \\
    & = \etaunshriek_{p'_1,\Phi(\I_{p_2}(V)),\Phi(V)}({\bf\Phi}^2_1(\D_{p_1}(a,V)(\etashriek_{p_1,\I_{p_1}(V),V}(\id_{\I_{p_1}(V)})))              & \by{\cite[Lemma 5.3]{fromunivwithPi}} \\
    & = \etaunshriek_{p'_1,\Phi(\I_{p_2}(V)),\Phi(V)}({\bf\Phi}^2_1(\etashriek_{p_1,\I_{p_2}(V),V}(a \circ \id_{\I_{p_1}(V)}))                     & \by{naturality of $\etaunshriek$} \\
    & = \etaunshriek_{p'_1,\Phi(\I_{p_2}(V)),\Phi(V)}({\bf\Phi}^2_1(\etashriek_{p_1,\I_{p_2}(V),V}(a))                                             & \\
    & = \etaunshriek_{p'_1,\Phi(X),\Phi(V)}({\bf\Phi}^2_1(\etashriek_{p_1,X,V}(a))                                             & 
\end{align*}
%
This finishes the proof of Lemma \ref{2015.04.06.l7}.
\end{myproof}

Consider the morphisms
%
$$\gls{zetai}:\Phi(\I_{p_i}(\U))\sr \I_{p'_i}(\U')$$
%
given by $\zeta_i:=\chi_i(\U)\circ \I_{p'_i}(\phi)$ and
%
$$\gls{wtzetai}:\Phi(\I_{p_i}(\wt{\U}_1))\sr \I_{p'_i}(\wt\U'_1)$$
%
given by $\wt{\zeta}_i:=\chi_i(\wt{\U}_1)\circ \I_{p'_i}(\wt\phi_1)$.
%
(Recall from \cite[\S 6]{fromunivwithPi} the morphisms $\xi_{{\bf\Phi}} : \Phi(\I_{p}(\U)) \to \I_{p'})(\U')$ and
$\wt\xi_{{\bf\Phi}} : \Phi(\I_{p}(\wt\U)) \to \I_{p'}(\wt\U')$ introduced, for a universe category functor ${\bf\Phi} = (\Phi,\phi,\wt\phi)$, by defining
$\gls{xiPhi} := \chi_{{\bf\Phi}}(\U) \circ \I_{p'}(\phi)$ and
$\gls{wtxiPhi} := \chi_{{\bf\Phi}}(\wt\U) \circ \I_{p'}(\wt\phi)$.
Note that $\zeta_i=\xi_{{\bf\Phi}_i}$ and $\wt{\zeta}_1=\wt\xi_{{\bf\Phi}_1}$, but $\wt{\zeta}_2 \ne \wt\xi_{{\bf\Phi}_2}$.)

\begin{theorem}
\label{2015.04.10.th3} Under the previous assumptions and notation the
morphisms $\zeta_1,\zeta_2,\wt{\zeta}_1,\wt{\zeta}_2$ form a morphism from the
square
%
$$
\begin{xy}
          \xymatrix@C=4pc{ \Phi(\I_{p_2}(\wt{\U}_1))
            \ar[r]^-{\Phi(\I^g(\wt{\U}_1))} \ar[d]_{\Phi(\I_{p_2}(p_2))} &
            \Phi(\I_{p_1}(\wt{\U}_1))
            \ar[d]^{\Phi(\I_{p_1}(p_2))}\\ \Phi(\I_{p_2}(\U))
            \ar[r]^{\Phi(\I^g(\U))} & \Phi(\I_{p_1}(\U)) }
\end{xy}
$$
%
to the square
%
$$
\begin{xy}
          \xymatrix@C=4pc{
                 \I_{p_2'}(\wt{\U}_1')   \ar[r]^-{\I^{g'}(\wt{\U}_1')} \ar[d]_{\I_{p_2'}(p_2')} & 
                 \I_{p_1'}(\wt{\U}_1') \ar[d]^{\I_{p_1'}(p_2')}\\
		\I_{p_2'}(\U')   \ar[r]^{\I^{g'}(\U')} &
		\I_{p_1'}(\U')
                }
\end{xy}
$$
%
\end{theorem}
%
\begin{myproof}
We need to prove commutativity of the outer squares of the following four diagrams:
%
$$
\begin{xy}
          \xymatrix@C=4pc@R=4pc{
                 \Phi(\I_{p_2}(\wt{\U}_1))  \ar[r]^-{\chi_2(\wt{\U}_1)} \ar[d]_-{\I^{g'}(\Phi(\wt{\U}_1))} & 
                 \I_{p_2'}(\Phi(\wt{\U}_1)) \ar[r]^-{\I_{p_2'}(\wt{\phi}_1)} \ar[d]^-{\I^{g'}(\Phi(\wt{\U}_1))} &
		\I_{p_2'}(\wt{\U}_1') \ar[d]^-{ \I^{g'}(\wt{\U}_1')} \\
		\Phi(\I_{p_1}(\wt{\U}_1))  \ar[r]^-{\chi_1(\wt{\U}_1)} &
		\I_{p_1'}(\Phi(\wt{\U}_1)) \ar[r]^-{\I_{p_1'}(\wt{\phi}_1)} &
		\I_{p_1'}(\wt{\U}_1')
                }
\end{xy}
$$
%
$$
\begin{xy}
          \xymatrix@C=4pc@R=4pc{ \Phi(\I_{p_2}(\U_1)) \ar[r]^-{\chi_2(\U_1)}
            \ar[d]_-{\Phi(\I^g(\U_1))} & \I_{p_2'}(\Phi(\U_1))
            \ar[r]^-{\I_{p_2'}(\wt{\phi}_1)} \ar[d]^-{\I^{g'}(\Phi(\U_1))} &
            \I_{p_2'}(\U_1') \ar[d]^-{\I^{g'}(\U_1')} \\ \Phi(\I_{p_1}(\U_1))
            \ar[r]^-{\chi_1(\U_1)} & \I_{p_1'}(\Phi(\U_1))
            \ar[r]^-{\I_{p_1'}(\wt{\phi}_1)} & \I_{p_1'}(\U_1') }
\end{xy}
$$
%
$$
\begin{xy}
          \xymatrix@C=4pc@R=4pc{ \Phi(\I_{p_2}(\wt{\U}_1))
            \ar[r]^-{\chi_2(\wt{\U}_1)} \ar[d]_-{\Phi(\I_{p_2}(p_1))} &
            \I_{p_2'}(\Phi(\wt{\U}_1)) \ar[r]^-{\I_{p_2'}(\wt{\phi}_1)}
            \ar[d]^-{\I_{p_2'}(\Phi(p_1))} & \I_{p_2'}(\wt{\U}_1')
            \ar[d]^-{\I_{p_2'}(p_1')} \\ \Phi(\I_{p_2}(\U_1))
            \ar[r]^-{\chi_2(\U_1)} & \I_{p_2'}(\Phi(\U_1))
            \ar[r]^-{\I_{p_2'}(\wt{\phi}_1)} & \I_{p_2'}(\U_1') }
\end{xy}
$$
%
$$
\begin{xy}
          \xymatrix@C=4pc@R=4pc{ \Phi(\I_{p_1}(\wt{\U}_1))
            \ar[r]^-{\chi_1(\wt{\U}_1)} \ar[d]_-{\Phi(\I_{p_1}(p_1))} &
            \I_{p_1'}(\Phi(\wt{\U}_1)) \ar[r]^-{\I_{p_1'}(\wt{\phi}_1)}
            \ar[d]^-{\I_{p_1'}(\Phi(p_1))} & \I_{p_2'}(\wt{\U}_1')
            \ar[d]^-{\I_{p_1'}(p_1')} \\ \Phi(\I_{p_1}(\U_1))
            \ar[r]^-{\chi_1(\U_1)} & \I_{p_1'}(\Phi(\U_1))
            \ar[r]^-{\I_{p_1'}(\wt{\phi}_1)} & \I_{p_1'}(\U_1') }
\end{xy}
$$
%
The left squares in the first and the second diagram are commutative by Lemma
\ref{2015.04.06.l7}.

The left squares in the third and the fourth diagram are commutative by
\cite[Lemma 5.7]{fromunivwithPi}.

The right hand side squares in the first and second diagram commute by Lemma
\ref{2015.04.10.l2}.

The right hand side squares of the third and the fourth diagram commute because
$\I_{p'_i}$ are functorial and therefore take commutative squares to commutative
squares.
\end{myproof}



\subsection{Universe category functors compatible with J-structures}


Let us define now conditions on functors of universe categories that reflect
the idea of compatibility with the J0- J1- and J2-structures on the
universes. 
For any functor ${\bf\Phi} = (\Phi,\phi,\wt{\phi}) : ({\C},p,pt) \to ({\C}',p',pt') $ of universe categories (the notion
was recalled in the previous section), for any $X\in \C$, and for any $F:X\sr \U$, the morphism
%
$$\Phi(\p_{X,F})*(\Phi(\Q(F))\circ\wt{\phi}):\Phi((X;F))\sr
(\Phi(X);\Phi(F)\circ\phi)'$$
%
is an isomorphism, and it will be denoted by $\gls{PhiXF}$.
(This isomorphism appears in \ref{iota-diagram} above, as the inverse $\Phi_{X,F} = \iota_{{\bf\Phi},X,F}^{-1}$.)
Let $\gls{PhiwtUp}$ be the composition
%
$$\Phi((\wt{\U};p)) \stackrel{\Phi_{\wt{\U},p}}{\lr} (\Phi(\wt{\U});\Phi(p)\circ
\phi)=(\Phi(\wt{\U});\wt{\phi}\circ p')\stackrel{\Q'(\wt{\phi},p')}{\lr}
(\wt{\U}';p')$$
%
We have another description of this morphism given by the following lemma.
%
\begin{lemma}
\label{2015.04.10.l5} One has:
%
$$\Phi\wt{\U}p=(\Phi(\p_{\wt{\U},p})\circ\wt{\phi})*(\Phi(\Q(p))\circ\wt{\phi})$$
%
\end{lemma}
%
\begin{myproof}
One has
%
$$\Phi\wt{\U}p\circ p'_{\wt{\U}',p'}=\Phi_{\wt{\U},p}\circ \Q'(\wt{\phi},p')\circ
p'_{\wt{\U}',p'} =\Phi_{\wt{\U},p}\circ \p_{\Phi(\wt{\U}),\wt{\phi}\circ
  p'}\circ\wt{\phi}=\Phi(\p_{\wt{\U},p})\circ \wt{\phi},$$
%
where the second equality is by definition of $\Q'(-,-)$ and the third equality
is by definition of $\Phi_{\wt{\U},p}$. Then
%
\begin{align*}
  \Phi\wt{\U}p\circ \Q'(p')
    & = \Phi_{\wt{\U},p}\circ \Q'(\wt{\phi},p')\circ \Q'(p') \\
    & = \Phi_{\wt{\U},p}\circ \Q(\wt{\phi}\circ p') \\
    & = \Phi_{\wt{\U},p}\circ \Q(\Phi(p)\circ \phi) \\
    & = \Phi(\Q(p))\circ \wt{\phi},
\end{align*}
%
where again the second equality is by definition of $\Q(-,-)$ and the fourth
equality is by definition of $\Phi_{\wt{\U},p}$.
\end{myproof}
%
\begin{lemma}
\label{2015.04.10.l6} For $s,s':Y\sr \wt{\U}$ such that $s\circ p=s'\circ p$
one has
%
$$\Phi(s*s')\circ\Phi\wt{\U}p=\Phi(s\circ \wt{\phi})*\Phi(s'\circ\wt{\phi})$$
%
and thus
%
$$\Phi(\Delta)\circ \Phi\wt{\U}p = \wt{\phi}*\wt{\phi}.$$
%
\end{lemma}
%
\begin{myproof}
Using Lemma \ref{2015.04.10.l5} we have
%
$$\Phi(s*s')\circ \Phi\wt{\U}p\circ
p'_{\wt{\U}',p'}=\Phi(s*s')\circ\Phi(\p_{\wt{\U},p})\circ\wt{\phi}=s\circ
\wt{\phi}$$
%
and
%
$$\Phi(s*s')\circ \Phi\wt{\U}p\circ \Q'(p')=\Phi(s*s')\circ \Phi(\Q(p))\circ
\wt{\phi}=s'\circ\wt{\phi}$$
%
The particular case of $\Delta$ follows from the fact that
$\Delta=\id_{\wt{\U}}*\id_{\wt{\U}}$.
\end{myproof}
%
\begin{lemma}
\label{2015.04.06.l5} The square
%
\begin{eq}\label{2015.04.06.eq10-bis}
\begin{xy}
          \xymatrix@C=3pc{ \Phi((\wt{\U};p)) \ar[r]^-{\Phi\wt{\U}p} \ar[d]_{
              \Phi(\p_{\wt{\U},p})} & (\wt{\U}';p')
            \ar[d]^{p'_{\wt{\U}',p'}}\\ \Phi(\wt{\U}) \ar[r]^{\wt{\phi}} &
            \wt{\U}' }
\end{xy}
\end{eq}%
%
is a pullback square.
\end{lemma}
%
\begin{myproof}
This square is equal to the composition of two squares
%
$$
\begin{xy}
          \xymatrix@C=3pc{ \Phi((\wt{\U};p)) \ar[r]^-{\Phi_{\wt{\U},p}}\ar[d]_{
              \Phi(\p_{\wt{\U},p})} & (\Phi(\wt{\U});\wt{\phi}\circ p')
            \ar[r]^-{\Q'(\wt{\phi},p')}\ar[d]_{\p_{\Phi(\wt{\U}),\wt{\phi}\circ
                p'}} & (\wt{\U}';p') \ar[d]^{p'_{\wt{\U}',p'}}\\ \Phi(\wt{\U})
            \ar[r]^{=} & \Phi(\wt{\U}) \ar[r]^{\wt{\phi}} & \wt{\U}' }
\end{xy}
$$
%
The right hand side square is a pullback square (\ref{2015.04.06.l0.sq}). The
left hand side square is a pullback square as a commutative square whose sides
are isomorphisms. We conclude that the composition of these two squares is a
pullback square.
\end{myproof}
%
\begin{definition}
\label{2015.04.06.def4} Let $Eq$ be a J0-structure on $p$ and $Eq'$ a
J0-structure on $p'$. A universe category functor $(\Phi,\phi,\wt{\phi})$ is
said to be compatible with $Eq$ and $Eq'$ if the square
%
\begin{eq}\label{2015.04.06.eq6}
\begin{xy}
          \xymatrix@C=3pc{ \Phi((\wt{\U};p)) \ar[r]^-{\Phi(Eq)}
            \ar[d]_{\Phi\wt{\U}p} & \Phi(\U) \ar[d]^{\phi}\\ (\wt{\U}';p')
            \ar[r]^{Eq'} & \U' }
\end{xy}
\end{eq}%
%
commutes.
\end{definition}
%
Let $Eq$, $Eq'$ be as above. Let $(\Phi,\phi,\wt{\phi})$ be a universe functor
compatible with $Eq$ and $Eq'$. Define a morphism
%
$$\wt{\phi}_{E}:\Phi(E\wt{\U})\sr E\wt{\U}'=((\wt{\U}';p'),Eq')$$
%
as
$$\gls{wtphiE} := (\Phi(\p_{(\wt{\U};p),Eq})\circ \Phi\wt{\U}p)*(\Phi(\Q(Eq))\circ \wt{\phi}).$$

\begin{lemma}
\label{2015.04.06.l4} Let $Eq$, $Eq'$ be as above. Let $(\Phi,\phi,\wt{\phi})$
be a universe functor compatible with $Eq$, and $Eq'$. Then the square
%
$$
\begin{xy}
          \xymatrix@C=3pc{ \Phi(E\wt{\U}) \ar[r]^-{\wt{\phi}_{E}}
            \ar[d]_{\Phi(\p_{(\wt{\U};p),Eq})} & E\wt{\U}'
            \ar[d]^{\p_{(\wt{\U}';p'),Eq'}}\\ \Phi((\wt{\U};p))
            \ar[r]^{\Phi\wt{\U}p} & (\wt{\U}';p') }
\end{xy}
$$
%
is a pullback square.
%
\end{lemma} 
%
\begin{myproof}
Consider the diagram
%
\begin{eq}\label{2015.04.06.eq8}
\begin{xy}
          \xymatrix@C=3pc{ \Phi(E\wt{\U})
            \ar[r]^-{\wt{\phi}_{E}}\ar[d]_{\Phi(\p_{(\wt{\U};p),Eq})} & E\wt{\U}'
            \ar[r]^-{\Q(Eq')}\ar[d]_{\p_{(\wt{\U}';p'),Eq'}} & \wt{\U}'
            \ar[d]^{p'}\\ \Phi((\wt{\U};p)) \ar[r]^{\Phi\wt{\U}p} & (\wt{\U}';p')
            \ar[r]^{Eq'} & \U' }
\end{xy}
\end{eq}%
%
The outer square of this diagram is equal to the outer square of the diagram
%
\begin{eq}\label{2015.04.06.eq7}
\begin{xy}
          \xymatrix@C=3pc{ \Phi(E\wt{\U})
            \ar[r]^-{\Phi(\Q(Eq))}\ar[d]_{\Phi(\p_{(\wt{\U};p),Eq})} &
            \Phi(\wt{\U}) \ar[r]^-{\wt{\phi}}\ar[d]_{\Phi(p)} & \wt{\U}'
            \ar[d]^{p'}\\ \Phi((\wt{\U};p)) \ar[r]^{\Phi(Eq)} & \Phi(\U)
            \ar[r]^{\phi} & \U', }
\end{xy}
\end{eq}%
%
where the equality of the lower horizontal arrows follows from the
commutativity of the square (\ref{2015.04.06.eq6}). The left hand side square
of this diagram is a pullback square because $\Phi$ takes canonical squares to
pullback squares. The right hand side square is a pullback square by definition
of a functor of universe categories. Therefore the outer square is a pullback
square. The right hand side square of (\ref{2015.04.06.eq8}) is a canonical
square and therefore a pullback square. We conclude that the left hand square
of (\ref{2015.04.06.eq8}) is a pullback square.
\end{myproof}
%
\begin{lemma}
\label{2015.04.06.l6} Let $Eq$, $Eq'$ be as above. Let $(\Phi,\phi,\wt{\phi})$
be a functor of universe categories compatible with $Eq$, and $Eq'$. Then the
square
%
\begin{eq}\label{2015.04.06.eq9}
\begin{xy}
          \xymatrix@C=3pc{ \Phi(E\wt{\U}) \ar[r]^-{\wt{\phi}_{E}}
            \ar[d]_{\Phi(pE\wt{\U})} & E\wt{\U}' \ar[d]^{pE\wt{\U}'}\\ \Phi(\U)
            \ar[r]^-{\phi} & \U' }
\end{xy}
\end{eq}%
%
is a pullback square.
%
\end{lemma} 
%
\begin{myproof}
It follows from the fact that the square (\ref{2015.04.06.eq9}) is equal to the
vertical composition of the squares of Lemmas \ref{2015.04.06.l4} and
\ref{2015.04.06.l5} with the square (\ref{2015.04.06.eq10-bis}).
\end{myproof}
%


\begin{definition}
\label{2015.04.06.def5} Let $Eq$, $Eq'$ be as above and let $\Omega$,
$\Omega'$ be J1-structures over $Eq$ and $Eq'$ respectively. A universe
category functor $(\Phi,\phi,\wt{\phi})$ is said to be compatible with $\Omega$
and $\Omega'$ if the square
%
$$
\begin{xy}
          \xymatrix@C=3pc{ \Phi(\wt{\U}) \ar[r]^-{\Phi(\Omega)}
            \ar[d]_{\wt{\phi}} & \Phi(\wt{\U})\ar[d]^{\wt{\phi}}\\ \wt{\U}'
            \ar[r]^-{\Omega'} & \wt{\U}' }
\end{xy}
$$
%
commutes.
\end{definition}
%
\begin{lemma}
\label{2015.04.10.l7} Let $Eq,\Omega$ and $Eq',\Omega'$ be as above and let
${\bf\Phi}$ be a universe category functor compatible with $Eq,Eq'$ and
$\Omega,\Omega'$. Then the square
%
$$
\begin{xy}
          \xymatrix@C=3pc{ \Phi(\wt{\U}) \ar[r]^-{\wt{\phi}}
            \ar[d]_{\Phi(\omega)} & \wt{\U}'\ar[d]^-{\omega'}\\ \Phi(E\wt{\U})
            \ar[r]^-{\wt{\phi}_{E}} & E\wt{\U}' }
\end{xy}
$$
%
commutes.
\end{lemma}
%
\begin{myproof}
Since $E\wt{\U}'=((\wt{\U}';p');Eq')$ it is sufficient to verify that the
compositions of the two paths in the square with $\p_{(\wt{\U}';p'),Eq'}$ and
$\Q(Eq')$ coincide. We have:
%
$$\wt{\phi}\circ\omega'\circ \Q(Eq')=\wt{\phi}\circ\Omega'$$
%
by definition of $\omega'$. On the other hand
%
$$\Phi(\omega)\circ \wt{\phi}_{E}\circ \Q(Eq')=\Phi(\omega)\circ
\Phi(\Q(Eq))\circ \wt{\phi}=\Phi(\Omega)\circ\wt{\phi},$$
%
where the first equation holds by definition of $\wt{\phi}_{E}$. The proof
follows now from the assumption that ${\bf\Phi}$ is compatible with $\Omega$
and $\Omega'$.
\end{myproof}
%


To formulate the condition of compatibility of a universe functor with full
J-structures on $\C$ and $\C'$ we will use Theorem
\ref{2015.04.10.th3}.

Let ${\bf\Phi}=(\Phi,\phi,\wt{\phi}) : (\C,p,pt) \to (\C',p',pt')$ be a functor of universe categories. In
view of Lemma \ref{2015.04.06.l6}, if $\Phi$ is compatible with $Eq$ and $Eq'$
then the triple $\gls{PhiE}:=(\Phi,\phi,\wt{\phi}_{E})$ is a functor of
universe categories as well, from $(\C,pE\wt{\U},pt)$ to $(\C',pE\wt{\U}',pt')$.  If, in addition, ${\bf\Phi}$ is compatible with
$\Omega$ and $\Omega'$ then, by Lemma \ref{2015.04.10.l7}, the morphisms $\omega$
and $\omega'$ satisfy the conditions on morphisms $g$ and $g'$ of Section
\ref{twouniv}.

Let
%
\begin{align*}
  \xi_{\bf\Phi}&:\Phi(\I_p(\U))\sr \I_{p'}(\U') \\
  \wt{\xi}_{\bf\Phi}&:\Phi(\I_p(\wt{\U}))\sr \I_{p'}(\wt{\U}')
\end{align*}
%
denote the compositions $\chi_{\bf\Phi}(\U)\circ \I_{p'}(\phi)$ and
$\chi_{\bf\Phi}(\wt{\U})\circ \I_{p'}(\wt{\phi})$, respectively\footnote{These maps were introduced in \cite[\S 6]{fromunivwithPi}, and were recalled above.}, and let
%
\begin{align*}
  \gls{zetaPhi}   &: \Phi(\I_{pE\wt{\U}}(\U))\sr \I_{pE\wt{\U}'}(\U') \\
  \gls{wtzetaPhi} &: \Phi(\I_{pE\wt{\U}}(\wt{\U}))\sr \I_{pE\wt{\U}'}(\wt{\U}')
\end{align*}
%
be given by the following compositions.
\begin{align}
  \zeta_{\bf\Phi}      & := \chi_{{\bf\Phi}_{E}}(\U)\circ \I_{pE\wt{\U}'}(\phi) \label{zeta-Phi-defn} \\
  \wt{\zeta}_{\bf\Phi} & := \chi_{{\bf\Phi}_{E}}(\wt{\U})\circ \I_{pE\wt{\U}'}(\wt{\phi}) \label{wt-zeta-Phi-defn}
\end{align}

 $$
and $$;
(Don't confuse these maps with $\zeta_i$ and $\wt\zeta_i$, which were introduced above.)
Note that
$\zeta_{\bf\Phi}=\xi_{{\bf\Phi}_{E}}$, but $\wt{\zeta}_{\bf\Phi}$ is different
from $\wt{\xi}_{{\bf\Phi}_{E}}$, since the latter is equal to the composition
$\chi_{{\bf\Phi}_{E}}(E\wt{\U})\circ \I_{pE\wt{\U}'}(\wt{\phi}_{E})$. Applying
Theorem \ref{2015.04.10.th3} in this context we get the following.
%
\begin{theorem}
\label{2015.04.10.th1} Let ${\bf\Phi}$ be a functor of universe categories
compatible with the J1-structures $(Eq,\Omega)$ and $(Eq',\Omega')$ on $p$ and
$p'$ respectively. Then the morphisms $\xi_{\bf\Phi}, \wt{\xi}_{\bf\Phi},
\zeta_{\bf\Phi}, \wt{\zeta}_{\bf\Phi}$ form a morphism from the square
%
$$
\begin{xy}
          \xymatrix@C=3pc{ \Phi(\I_{pE\wt{\U}}(\wt{\U}))
            \ar[r]^-{\Phi(\I^{\omega}(\wt{\U}))} \ar[d]_{\Phi(\I_{pE\wt{\U}}(p))} &
            \Phi(\I_p(\wt{\U}))\ar[d]^-{\Phi(\I_p(p))}\\ \Phi(\I_{pE\wt{\U}}(\U))
            \ar[r]^-{\Phi(\I^{\omega}(\U))} & \Phi(\I_p(\U)) }
\end{xy}
$$
%
to the square
%
$$
\begin{xy}
          \xymatrix@C=3pc{ \I_{pE\wt{\U}'}(\wt{\U}')
            \ar[r]^-{\I^{\omega'}(\wt{\U}')} \ar[d]_-{\I_{pE\wt{\U}'}(p')} &
            \I_{p'}(\wt{\U}')\ar[d]^-{\I_{p'}(p')}\\ \I_{pE\wt{\U}'}(\U')
            \ar[r]^-{\I^{\omega}(\U')} & \I_{p'}(\U') }
\end{xy}
$$
%
\end{theorem}
%
Let $\gls{RPhi}$ denote the composite map
%
\begin{align}
  \label{RPhi-defn}
  \Phi((\I_{pE\wt{\U}}&(\U), \I^{\omega}(\U))\times_{\I_p(\U)} (\I_p(\wt{\U}), \I_p(p))) \\
      & \sr \Phi(\I_{pE\wt{\U}}(\U), \I^{\omega}(\U))\times_{\Phi(\I_p(\U))}\Phi(\I_p(\wt{\U}), \I_p(p)) \\
      & \sr (\I_{pE\wt{\U}'}(\U'),\I^{\omega'}(\U'))\times_{\I_{p'}(\U')}(\I_{p'}(\wt{\U}'),\I_{p'}(p')),
\end{align}
%
where the second arrow is defined by $\xi_{\bf\Phi}, \wt{\xi}_{\bf\Phi}$ and
$\zeta_{\bf\Phi}$ in view of Theorem \ref{2015.04.10.th1}.
%
\begin{definition}
\label{2015.04.06.def6} Let $Eq$, $Eq'$, $\Omega$ and $\Omega'$ be as
above. Let $Jp$ and $Jp'$ be J2-structures over $(Eq,\Omega)$ and
$(Eq',\Omega')$ respectively.  A universe category functor
$(\Phi,\phi,\wt{\phi})$ is said to be compatible with $Jp$ and $Jp'$ if it is
compatible with $Eq$, $Eq'$ and $\Omega$, $\Omega'$ in the sense of Definitions
\ref{2015.04.06.def4} and \ref{2015.04.06.def5} respectively, the square
%
$$
\begin{xy}
          \xymatrix@C=3pc{ \I_{pE\wt{\U}'}(\wt{\U}')
            \ar[r]^-{\I^{\omega'}(\wt{\U}')} \ar[d]_-{\I_{pE\wt{\U}'}(p')} &
            \I_{p'}(\wt{\U}')\ar[d]^-{\I_{p'}(p')}\\ \I_{pE\wt{\U}'}(\U')
            \ar[r]^-{\I^{\omega}(\U')} & \I_{p'}(\U') }
\end{xy}
$$
commutes, and the square
$$
\begin{CD}
\Phi((\I_{pE\wt{\U}}(\U), \I^{\omega}(\U))\times_{\I_p(\U)} (\I_p(\wt{\U}), \I_p(p)))
@>R_{\bf\Phi}>>
(\I_{pE\wt{\U}'}(\U'),\I^{\omega'}(\U'))\times_{\I_{p'}(\U')}(\I_{p'}(\wt{\U}'),\I_{p'}(p'))\\ @V\Phi(Jp)
VV @VVJp' V\\ \Phi(\I_{pE\wt{\U}}(\wt{\U})) @>\wt{\zeta}_{\bf\Phi}>>
\I_{pE\wt{\U}'}(\wt{\U}')
\end{CD}
$$
%
commutes.
\end{definition}




\subsection{Homomorphisms of C-systems compatible with J-structures}
%



%
\begin{definition}
\label{2015.04.06.def1} Let $H:\CC\sr \CC'$ be a homomorphism of C-systems.
%
\begin{enumerate}
\item Let $\Id$ and $\Id'$ be J0-structures on $\CC$ and $\CC'$ respectively.  Then
  $H$ is called a homomorphism of C-systems with J0-structures $(\CC,\Id)\sr
  (\CC,\Id')$ if for each $\Gamma\in Ob(\CC)$ and $o,o'\in\Obwt_1(\Gamma)$ such
  that $\partial(o)=\partial(o')$, one has
%
$$H(\Id_{\Gamma}(o,o'))=\Id'_{H(\Gamma)}(H(o),H(o'))$$
%
(the right hand side of the equality makes sense because $H$ commutes with
  $\partial$).
% 
\item Let $\Id$, $\Id'$ be as above and let $\refl$, $\refl'$ be J1-structures
  over $\Id$ and $\Id'$ respectively. A homomorphism of C-systems with
  J0-structures $H:(\CC,\Id)\sr (\CC',\Id')$ is called a homomorphism of
  C-systems with J1-structures
%
$$(\CC,\Id,\refl)\sr (\CC',\Id',\refl')$$
%
if for all $\Gamma\in Ob(\CC)$ and $o\in \Obwt_1(\Gamma)$ one has
%
$$H(\refl(o))=\refl'(H(o))$$
%
\end{enumerate}
\end{definition}
%
For a C-system $\CC$ with a J0-structure $\Id$ and a J1-structure $\refl$ over
$\Id$ define $Jdom(\CC,\Id,\refl)$ as the set of quadruples $(\Gamma,T,P,s0)$,
where $\Gamma\in Ob$, $T\in Ob_1(\Gamma)$, $P\in Ob_1(\Idx(T))$ and $s0\in
\Obwt(\rf_T^*(P))$. Equivalently we can say that $Jdom(\CC,\Id,\refl)$ is the
subset in $Ob\times Ob\times Ob\times \Obwt$ that consists of quadruples
$(\Gamma,T,P,s0)$, where $\ft(T)=\Gamma$, $\ft(P)=\Idx(T)$ and
$\partial(s0)=\rf_T^*(P)$. Then a J2-structure is defined by a map $Jdom\sr
\Obwt$ with some properties.
%
\begin{lemma}
\label{2015.04.06.l3} Let $H:\CC\sr \CC'$ be a homomorphism of C-systems. Let
$\Gamma,X,Y\in Ob(\CC)$, $m,n\in\nn$ and suppose that
$\ft^m(X)=\ft^{n}(Y)=\Gamma$. Let $f:X\sr Y$ be a morphism over $\Gamma$ and let
$F:\Gamma'\sr \Gamma$ be a morphism. Then
%
$$H(F^*(f))=H(F)^*(H(f))$$
%
\end{lemma}
%
\begin{myproof}
This is easy to show from the defining properties of $F^*(f)$ and
$H(F)^*(H(f))$.
\end{myproof}
%
%
\begin{lemma}
\label{2015.04.06.l2} Let $\Id$, $\Id'$, $\refl$ and $\refl'$ be as in
Definition \ref{2015.04.06.def1} and let
%
$$H:(\CC,\Id,\refl)\sr (\CC',\Id',\refl')$$
%
be a homomorphism of C-systems with J1-structures. Then for all elements
$(\Gamma,T,P,s0)$ of $Jdom(\Id,\refl)$ one has $(H(\Gamma),H(T),H(P),H(s0))\in
Jdom(\Id',\refl')$.
\end{lemma}
%
\begin{myproof}
We have $\ft(H(T))=H(\ft(T))=H(\Gamma)$ and $\ft(H(P))=H(\ft(P))=H(\Idx(T))$. We
also have $\partial(H(s0))=H(\partial(s0))=H(\rf_T^*(P))$. By Lemma
\ref{2015.04.06.l3} we further have
%
$$H(\rf_T^*(P))=H(\rf_T)^*(H(P))$$
%
It remains to show that the following two equations.
\begin{align}
  H(\Idx(T)) & = \Idx'(H(T))   \label{HIdxT-eqn} \\
  H(\rf_T)   & = \rf_{H(T)}'   \label{HrfT-eqn}
\end{align}
They follow by a lengthy computation from the defining
equations (\ref{2015.04.06.eq1}) and (\ref{2015.04.02.eq1}), which we omit.
\end{myproof}
%
\begin{definition}
\label{2015.04.06.def2} Let $\Id$, $\Id'$, $\refl$ and $\refl'$ be as in
Definition \ref{2015.04.06.def1} and let $\J$, $\J'$ be J2-structures over
$(\Id,\refl)$ and $(\Id',\refl')$ respectively. A homomorphism of C-systems with
J1-structures
%
$$H:(\CC,\Id,\refl)\sr (\CC',\Id',\refl')$$
%
is called a homomorphism of C-systems with J-structures
%
$$(\CC,\Id,\refl,\J)\sr (\CC',\Id',\refl',\J')$$
%
if for all $\Gamma\in Ob(\CC)$, $T\in Ob_1(\Gamma)$, $P\in Ob_1(\Idx(T))$ and
$s0\in \Obwt(\rf_T^*(P))$ one has
%
$$H(\J(\Gamma,T,P,s0))=\J'(H(\Gamma),H(T),H(P),H(s0)),$$
%
where the right hand side of the equation makes sense by Lemma
\ref{2015.04.06.l2}.
\end{definition}
%











\subsection{Functoriality of the J-structures $(\Id_{Eq},\refl_{\Omega},\J_{Jp})$}
%
\label{2015.04.12.sec1}
%
Let us first recall that by \cite[Construction 3.3]{Cfromauniverse} any
universe category functor $${\bf \Phi}=(\Phi,\phi,\wt{\phi}) : (\C,p,pt) \to (\C',p',pt')$$ defines a
homomorphism of C-systems
%
$$H = \gls{HPhi} :\toCC({\C},p)\sr \toCC({\C}',p').$$
%
To define $H$ on objects, one defines by induction on $n$, for all $\Gamma\in
Ob_n(\toCC({\C},p))$, pairs $(H(\Gamma),\psi_{\Gamma})$, where
$H(\Gamma)\in Ob(\toCC({\C}',p'))$ and $\psi_{\Gamma}$ is a morphism
%
$$\gls{psiGamma} : int'(H(\Gamma))\sr \Phi(int(\Gamma)),$$
%
as follows. For $n=0$ one has $H(())=()$ and $\psi_{()}:pt'\sr \Phi(pt)$ is the
unique morphism to a final object $\Phi(pt)$. For $T = (\Gamma,F)\in Ob_{n+1}$ one
has
%
$$H(T)=(H(\Gamma),\psi_{\Gamma}\circ\Phi(F)\circ \phi),$$
%
in other words, that
\begin{eq}
  \label{u1-H-eqn}
  u_1(H(T)) = \psi_{\ft(T)} \circ \Phi(u_1(T)) \circ \phi.
\end{eq}%
and
\begin{eq}
  \label{ft-H-eqn}
  \ft(H(T)) = H(\ft(T)).
\end{eq}%
Moreover, $\psi_{(\Gamma,F)}$ is the unique morphism $int'(H(\Gamma,F))\sr \Phi(int(\Gamma,F))$ such that
%
\begin{eq}
  \label{psi-Q-compat}
  \psi_{(\Gamma,F)}\circ \Phi(\Q(F))\circ\wt{\phi}=\Q'(\psi_{\Gamma}\circ\Phi(F)\circ\phi)
\end{eq}%
and
\begin{eq}
  \label{psi-p-compat}
  \psi_{(\Gamma,F)}\circ \Phi(\p_{\Gamma,F})=\p_{H((\Gamma,F))}\circ \psi_{\Gamma}
\end{eq}%
%%
Observe that $\psi_{\Gamma}$ is automatically an isomorphism. The action of $H$
on morphisms is given, for $f:\Gamma\sr\Gamma'$, by
%
$$H(f)=\psi_{\Gamma}\circ\Phi(f)\circ\psi_{\Gamma'}^{-1}$$
%



\begin{lemma}
\label{2015.04.12.l1} Let ${\bf\Phi}$ be a universe category functor as above
that is compatible (as defined in \ref{2015.04.06.def4}) with the J0-structures $Eq$ and $Eq'$ on $p$ and $p'$
respectively. Then the homomorphism of C-systems $H=H({\bf\Phi})$ is a
homomorphism of C-systems with J0-structures relative to $\Id_{Eq}$ and
$\Id_{Eq'}$.
\end{lemma}
%
\begin{myproof}
Let $\Id=\Id_{Eq}$ and $\Id'=\Id_{Eq'}$. We need to check that for all
$\Gamma\in Ob(\toCC({\C},p))$ and $o,o'\in \Obwt_1(\Gamma)$ such that
$\partial(o)=\partial(o')$ one has
%
$$H(\Id(o,o'))=\Id'(H(o),H(o'))$$
%
Since
%
$$\partial(H(\Id(o,o'))=H(\Gamma)=H(\ft(\partial(o)))=\ft(\partial(H(o)))=\partial(\Id'(H(o),H(o')))$$
%
it suffices to check that
%
$$u_1(H(\Id(o,o')))=u_1(\Id'(H(o),H(o'))).$$
%
We have:
%
\begin{align*}
  u_1(H&(\Id(o,o'))) \\
  & = \psi_{\Gamma}\circ\Phi(u_1(\Id(o,o')))\circ\phi & \by{\cite[Lemma 6.1(1)]{fromunivwithPi}} \\
  & = \psi_{\Gamma}\circ\Phi((\wt{u}_1(o)*\wt{u}_1(o'))\circ Eq)\circ\phi & \by{\ref{IdEq-eqn}} \\
  & = \psi_{\Gamma}\circ\Phi(\wt{u}_1(o)*\wt{u}_1(o'))\circ \Phi(Eq)\circ\phi & \by{functoriality of $\Phi$} \\
  & = \psi_{\Gamma}\circ\Phi(\wt{u}_1(o)*\wt{u}_1(o'))\circ \Phi\wt{\U}p\circ Eq' & \by{\ref{2015.04.06.eq6}} \\
  & = \psi_{\Gamma}\circ((\Phi(\wt{u}_1(o))\circ\wt{\phi})*(\Phi(\wt{u}_1(o'))\circ\wt{\phi}))\circ Eq' & \by{Lemma \ref{2015.04.10.l6}} \\
  & = ((\psi_{\Gamma}\circ\Phi(\wt{u}_1(o))\circ\wt{\phi})*(\psi_{\Gamma}\circ\Phi(\wt{u}_1(o'))\circ\wt{\phi}))\circ Eq' & \by{\ref{star-functoriality}} \\
  & = (\wt{u}_1(H(o))*\wt{u}_1(H(o')))\circ Eq' & \by{\cite[Lemma 6.1(2)]{fromunivwithPi}} \\
  & = u_1(\Id'(H(o),H(o))) & \by{\ref{IdEq-eqn}}
\end{align*}
\end{myproof}

\begin{lemma}
\label{2015.04.12.l2} Let ${\bf\Phi}$ be a universe category functor as above
that is compatible with the (J0,J1)-structures $(Eq,\Omega)$ and
$(Eq',\Omega')$ on $p$ and $p'$ respectively. Then the homomorphism of
C-systems $H=H({\bf\Phi})$ is a homomorphism of C-systems with
(J0,J1)-structures relative to $(\Id_{Eq},\refl_{\Omega})$ and
$(\Id_{Eq'},\refl_{\Omega'})$.
\end{lemma}
%
\begin{myproof}
Let $\refl=\refl_{\Omega}$ and $\refl'=\refl_{\Omega'}$. The compatibility
condition is
%
\begin{eq}
  \label{Omega-compat}
  \Phi(\Omega)\circ\wt{\phi}=\wt{\phi}\circ\Omega'
\end{eq}
%
We need to check that for $\Gamma\in Ob(\toCC({\C},p))$ and $s\in
\Obwt_1(\Gamma)$ one has
%
$$H(\refl(s))=\refl'(H(s))$$
%
We have
%
\begin{align*}
  H(\refl(s))
    & = H(\wt{u}_1^{-1}(\wt{u}_1(s)\circ\Omega)) & \by{\ref{refl-defn}} \\
    & = \wt{u}_1^{-1}(\psi_{\Gamma}\circ\Phi(\wt{u}_1(s)\circ\Omega)\circ\wt{\phi}) & \by{\cite[Lemma 6.1(2)]{fromunivwithPi}} \\
    & = \wt{u}_1^{-1}(\psi_{\Gamma}\circ\Phi(\wt{u}_1(s))\circ\Phi(\Omega)\circ \wt{\phi}) & \by{functoriality of $\Phi$} \\
    & = \wt{u}_1^{-1}(\psi_{\Gamma}\circ\Phi(\wt{u}_1(s))\circ\wt{\phi}\circ\Omega') & \by{\ref{Omega-compat}} \\
    & = \wt{u}_1^{-1}(\wt{u}_1(H(s))\circ\Omega') & \by{\cite[Lemma 6.1(2)]{fromunivwithPi}} \\
    & = \refl'(H(s)) & \by{\ref{refl-defn}}
\end{align*}
\end{myproof}
%

To prove the functoriality of the full J-structures we will need some lemmas
first.

Recall that in \cite{Csubsystems} we let $\p_{\Gamma,n}:\Gamma\sr \ft^n(\Gamma)$
denote the composition $\p_{\Gamma}\circ \dots\circ \p_{\ft^{n-1}(\Gamma)}$ of $n$ canonical projections.
%
\begin{lemma}
\label{2015.05.10.l1} Let ${\bf\Phi}$ be a universe category functor and
$\Gamma\in Ob(\toCC({\C},p))$ be such that $l(\Gamma)\ge n$. Then the
square
%
$$
\begin{CD}
int'(H(\Gamma)) @>\psi_{\Gamma}>> \Phi(int(\Gamma))\\ @Vp_{H(\Gamma),n} VV
@VV\Phi(\p_{\Gamma,n})V\\ int'(H(\ft^n(\Gamma))) @>\psi_{\ft^n(\Gamma)}>>
\Phi(int(\ft^n(\Gamma)))
\end{CD}
$$
%
commutes.
\end{lemma}
%
\begin{myproof}
It follows by induction from the defining relation
$\psi_{\Gamma}\circ \Phi(\p_{\Gamma})=\p_{H(\Gamma)}\circ \psi_{\ft(\Gamma)}$
of $\psi_{\Gamma}$.
\end{myproof}








\begin{lemma}
\label{2015.05.06.l3} Let $Eq$ and $Eq'$ be J0-structures on $({\C},p)$ and
$({\C}',p')$ respectively; let ${\bf\Phi} : ({\C},p,pt) \to ({\C'},p',pt')$ be a universe category functor compatible
with $Eq$ and $Eq'$ (as defined in \ref{2015.04.06.def4});
let $\Idx'$ be the analogue, for the category $\toCC(\C',p')$, of $\Idx$;
let ${\bf\Phi}_{E}^2$ be the analogue, for the universe $pE\wt\U$, of ${\bf\Phi}^2$ (as defined in \ref{defn-Phi2});
and let $H = H({\bf\Phi})$. Then for all $\Gamma\in Ob(\toCC({\C},p))$, 
$T \in Ob_1(\Gamma)$, $P\in Ob_1(\Idx(T))$, and $o\in \Obwt(P)$, one has:
%
\begin{enumerate}
\item $(u'_{1,H(\Gamma)}(H(T)), u_{1,\Idx'(H(T))}'(H(P)))$ is a well defined
  element of $\D_{pE\wt{\U}'}(\Phi(int(\Gamma)),\U')$, and it is equal to 
  $ \D_{pE\wt{\U}'}(\psi_{\Gamma},\wt{\U}')(\D_{pE\wt{\U}'}(int'(H(\Gamma)),\phi)({\bf\Phi}_{E}^2(u_{1,\Gamma}(T),u_{1,\Idx(T)}(P)))); $
  and
\item $(u'_{1,H(\Gamma)}(H(T)), \wt{u}_{1,\Idx'(H(T))}'(H(o)))$ is a well
  defined element of $\D_{pE\wt{\U}'}(\Phi(int(\Gamma)),\wt{\U}')$, and it is equal to 
  $ \D_{pE\wt{\U}'}(\psi_{\Gamma},\wt{\U}')(\D_{pE\wt{\U}'}(int'(H(\Gamma)),\wt{\phi})({\bf\Phi}_{E}^2(u_{1,\Gamma}(T),\wt{u}_{1,\Idx(T)}(o)))). $
\end{enumerate}
\end{lemma}
%
\begin{remark}\rm
  Since
  \begin{align}
    \gls{u2}  (T)    & := (u_1(\ft(T)),u_1(T))                \in \D_p(int(\Gamma),   \U) \label{u2-defn} & \text{and} \\
    \gls{wtu2}(s)    & := (u_1(\ft(\partial(s))),\wt{u}_1(s)) \in \D_p(int(\Gamma),\wt\U), \label{wt-u2-defn}
  \end{align}
this lemma is very similar
to \cite[Lemma 6.1(3,4)]{fromunivwithPi}, but its proof is more involved
because of the interaction of the two universe functors.  (For the introduction of $u_2$ and $\wt u_2$ see
\cite[Problem 3.5]{fromunivwithPi}.)
\end{remark}
%
\begin{myproof}
We will consider only the second assertion; the proof of the first one is
similar and simpler.

To prove that the pair
$$(u'_{1,H(\Gamma)}(H(T)), \wt{u}_{1,\Idx'(H(T))}'(H(o)))$$
is a well defined element of
$\D_{pE\wt{\U}'}(\Phi(int(\Gamma)),\wt{\U}')$ we need to show that
$\ft(\partial(H(o)))=\Idx'(H(T))$ and that the source of
$\wt{u}_{1,\Idx'(H(T))}'(H(o)))$ equals $(int'(H(\Gamma)); u'_1(H(T)))_{E}$,
i.e., that
%
$$int'(\Idx'(H(T)))=(int'(H(\Gamma)); u'_1(H(T)))_{E}$$
%
The former is a corollary of our assumptions and Lemma \ref{2015.04.12.l1}, and
the latter is a corollary of \cite[Problem 3.3(1)]{fromunivwithPi} and the
first equation of Lemma \ref{2015.03.27.l1}.

Let $X=int(\Gamma)$, $F=u_{1,\Gamma}(T)$, and $\wt{G}=\wt{u}_{1,\Idx(T)}(o)$, so that
$(F,\wt G) \in \D_{pE\wt{\U}}(X,\wt\U)$.
By the definitions we have
%
\begin{align*}
  \D_{pE\wt{\U}'}&(\psi_{\Gamma},\_)(\D_{pE\wt{\U}'}(\_,\wt{\phi})({\bf\Phi}_{E}^2(F,\wt{G}))) \\
    & = \D_{pE\wt{\U}'}(\psi_{\Gamma},\_)(\D_{pE\wt{\U}'}(\_,\wt{\phi})(\Phi(F)\circ \phi, \iota\circ \Phi(\wt{G}))) & \by{def'n of ${\bf\Phi}_{E}^2$; cf.{} \ref{defn-Phi2}} \\
    & = \D_{pE\wt{\U}'}(\psi_{\Gamma},\_)(\Phi(F)\circ \phi, \iota\circ \Phi(\wt{G})\circ\wt{\phi}) & \by{def'n of $\D_{pE\wt{\U}'}$ on morphisms} \\
    & = (\psi_{\Gamma}\circ\Phi(F)\circ \phi, \Q(\psi_{\Gamma},\Phi(F)\circ\phi)_{E'}\circ\iota\circ \Phi(\wt{G})\circ\wt{\phi}), & \by{def'n of $\D_{pE\wt{\U}'}$ on morphisms}
\end{align*}
%
where
%
$$\iota:(\Phi(X);\Phi(F)\circ \phi)_{E'}\sr \Phi((X;F)_{E})$$
%
is the unique morphism such that the following two equations are satisfied.
%
\begin{align}
  \iota \circ \Phi(\p^E_{X,F})               & = p^{E'}_{\Phi(X),\Phi(F)\circ\phi}         \label{iota1} \\
  \iota \circ \Phi(\Q(F)_E)\circ \wt{\phi}_E & = \Q(\Phi(F)\circ\phi)_{E'}                 \label{iota2}
\end{align}
(This map $\iota$ is analogous to the one defined in \ref{iota-diagram}.)

On the other hand, by \cite[Lemma 6.1(1)]{fromunivwithPi},
%
$$u_{1,H(\Gamma)}(H(T))=\psi_{\Gamma}\circ \Phi(u_{1,\Gamma}(T))\circ \phi$$
and
\begin{align*}
  \wt{u}_{1,\Idx'(H(T))}(H(o))
    & = \wt{u}_{1,H(\Idx(T))}(H(o)) \\
    & = \psi_{H(\Idx(T))}\circ \Phi(\wt{u}_{1,\Idx(T)}(o))\circ \wt{\phi}
\end{align*}
%
by \cite[Lemma 6.1(1,2)]{fromunivwithPi}.  Applying that and \ref{u1-H-eqn},
we see that to prove the lemma it is sufficient to show that
%
$$\psi_{\Idx(T)}=\Q(\psi_{\Gamma},\Phi(F)\circ\phi)_{E'}\circ\iota.$$
%
Both sides are morphisms with codomain
%
$$\Phi(int(\Idx(T)))=\Phi((X;F)_E),$$
%
and since ${\bf\Phi}_E$ is a universe category functor it is sufficient to show
that the compositions of the two sides with $\Phi(\p^E_{X,F})$ and
$\Phi(\Q(F)_E)\circ \wt{\phi}_E$ are the same, i.e., that the following two equations hold.
\begin{align}
  \psi_{\Idx(T)} \circ \Phi(\p^E_{X,F})               & =\Q(\psi_{\Gamma},\Phi(F)\circ\phi)_{E'}\circ\iota \circ \Phi(\p^E_{X,F})                \label{eqn-11a} \\
  \psi_{\Idx(T)} \circ \Phi(\Q(F)_E)\circ \wt{\phi}_E & = \Q(\psi_{\Gamma},\Phi(F)\circ\phi)_{E'}\circ\iota \circ \Phi(\Q(F)_E)\circ \wt{\phi}_E \label{eqn-11b}
\end{align}
We establish equation \ref{eqn-11a} as follows.
%
\begin{align*}
        \Q(\psi_{\Gamma},&\Phi(F)\circ\phi)_{E'} \circ \iota \circ \Phi(\p^E_{X,F}) \\
    & = \Q(\psi_{\Gamma}, \Phi(F)\circ\phi)_{E'} \circ \p^{E'}_{\Phi(X),\Phi(F)\circ\phi}   & \by{\ref{iota1}} \\
    & = \p^{E'}_{int'(H(\Gamma)),\psi_{\Gamma} \circ \Phi(F)\circ\phi} \circ \psi_\Gamma    & \by{commutativity of \ref{QfFE-diag}} \\
    & = \p_{H(\Idx(X,F)),3} \circ \psi_\Gamma                                               & \by{\ref{2015.03.27.l1-eqn2}} \\
    & = \p_{H(\Idx(T)),3} \circ \psi_\Gamma                                                 & \by{def'n of $X$ and $F$} \\
    & = \psi_{\Idx(T)} \circ \Phi(\p_{\Idx(T),3})                                           & \by{\ref{2015.05.10.l1}} \\
    & = \psi_{\Idx(T)} \circ \Phi(\p_{\Idx(X,F),3})                                         & \by{def'n of $X$ and $F$} \\
    & = \psi_{\Idx(T)} \circ \Phi(\p^E_{X,F}).                                              & \by{\ref{2015.03.27.l1-eqn3}}
\end{align*}

Since
%
\begin{align*}
        \Q(\psi_{\Gamma},\Phi(F)\circ\phi&)_{E'}\circ \iota\circ \Phi(\Q(F)_E)\circ \wt{\phi}_E \\
    & = \Q(\psi_{\Gamma},\Phi(F) \circ\phi)_{E'}\circ \Q(\Phi(F)\circ\phi)_{E'}                 & \by{\ref{iota2}} \\
    & = \Q(\psi_{\Gamma}\circ \Phi(F)\circ \phi)_{E'},                                           & \by{\ref{QfFE-defn}}
\end{align*}
%
equation \ref{eqn-11b} reduces to
%
\begin{eq}
  \label{2015.05.10.eq2.0}
  \psi_{\Idx(T)}\circ \Phi(\Q(F)_E)\circ \wt{\phi}_E = \Q(\psi_{\Gamma}\circ \Phi(F)\circ \phi)_{E'}.
\end{eq}%
%

We have
%
$$\wt{\phi}_E=(\Phi(\p_{(\wt{\U};p),Eq})\circ
\Phi\wt{\U}p)*(\Phi(\Q(Eq))\circ\wt{\phi}).$$
%
Therefore (\ref{2015.05.10.eq2.0}) is equivalent to two equations:
%
\begin{align}
  \psi_{\Idx(T)}\circ \Phi(\Q(F)_E)\circ \Phi(\p_{(\wt{\U};p),Eq})\circ \Phi\wt{\U}p
  & = \Q(\psi_{\Gamma}\circ \Phi(F)\circ \phi)_{E'}\circ \p_{(\wt{\U}';p'),Eq'} \label{2015.05.10.eq2a} \\
  \psi_{\Idx(T)}\circ \Phi(\Q(F)_E)\circ \Phi(\Q(Eq))\circ\wt{\phi}
  & = \Q(\psi_{\Gamma}\circ \Phi(F)\circ \phi)_{E'}\circ \Q'(Eq'). \label{2015.05.10.eq2b}
\end{align}
%
The first equality we will have to decompose further into two using the fact
that by Lemma \ref{2015.04.10.l5} we have
%
$$\Phi\wt{\U}p=(\Phi(\p_{\wt{\U},p})\circ\wt{\phi})*(\Phi(\Q(p))\circ \wt{\phi}).$$
%
Therefore (\ref{2015.05.10.eq2a}) is equivalent to two equations:
%
\begin{align}
  \psi_{\Idx(T)}\circ \Phi(\Q(F)_E)\circ \Phi(\p_{(\wt{\U};p),Eq})\circ \Phi(\p_{\wt{\U},p})\circ\wt{\phi}
  & = \Q(\psi_{\Gamma}\circ \Phi(F)\circ \phi)_{E'}\circ \p_{(\wt{\U}';p'),Eq'}\circ \p_{\wt{\U}',p'} \label{2015.05.10.eq2aa} \\
  \psi_{\Idx(T)}\circ \Phi(\Q(F)_E)\circ \Phi(\p_{(\wt{\U};p),Eq})\circ \Phi(\Q(p))\circ\wt{\phi}
  & = \Q(\psi_{\Gamma}\circ \Phi(F)\circ \phi)_{E'}\circ \p_{(\wt{\U}';p'),Eq'}\circ \Q'(p'). \label{2015.05.10.eq2ab}
\end{align}
%
To prove (\ref{2015.05.10.eq2b}) observe first two useful equalities:
%
\begin{align}
  u_1(\Idx(T))&=\Q(\Q(F),p)\circ Eq \label{usefulequal1} \\
  \Q(u_1(\Idx(T)))&=\Q(F)_{E}\circ \Q(Eq), \label{usefulequal2}
\end{align}
%
where the first follows from the proof of Lemma \ref{2015.03.27.l1}, and the
second is the combination of the first with the third equality of the same
lemma.

Now we have:
%
\begin{align*}
        \psi_{\Idx(T)}\circ \Phi&(\Q(F)_E) \circ \Phi(\Q(Eq))\circ\wt{\phi} \\
    & = \psi_{\Idx(T)}\circ \Phi (\Q(F)_E  \circ \Q(Eq))     \circ\wt{\phi} & \by{functoriality of $\Phi$} \\
    & = \psi_{\Idx(T)}\circ \Phi (\Q(u_1(\Idx(T))))          \circ\wt{\phi} & \by{\ref{usefulequal2}} \\
    & = \Q'(\psi_{\Gamma}\circ \Phi(u_1(\Idx(T)))\circ \phi)                & \by {\ref{psi-Q-compat}} \\
    & = \Q'(u_1(H(\Idx(T))))                                                & \by{\ref{u1-H-eqn}} \\
    & = \Q'(u_1(\Idx'(H(T))))                                               & \by{\ref{HIdxT-eqn}} \\
    & = \Q(u_1(H(T)))_{E'}\circ \Q'(Eq')                                    & \by{\ref{usefulequal2}} \\
    & = \Q(\psi_{\Gamma}\circ \Phi(F)\circ \phi)_{E'} \circ \Q'(Eq').       & \by{\ref{u1-H-eqn}} \\
\end{align*}
The equality (\ref{2015.05.10.eq2b}) is proved.

To prove (\ref{2015.05.10.eq2aa}) observe two equalities:
\begin{align}
  \Q(F)_E\circ \p_{(\wt{\U};p),Eq} & =\p_{\Idx(T)}\circ \Q(\Q(F),p) \label{usefulequal3} \\
  \Q(\Q(F),p)\circ \p_{\wt{\U},p} & =\p_{\ft(\Idx(T)}\circ \Q(F).  \label{usefulequal4}
\end{align}
%
The same equalities hold for $F' := \psi_{\Gamma}\circ\Phi(F)\circ
\phi=u_1'(H(T))$, and the equation (\ref{2015.05.10.eq2aa}) becomes
%
$$\psi_{\Idx(T)}\circ \Phi(\p_{\Idx(T)})\circ \Phi(\p_{\ft(\Idx(T))})\circ
\Phi(\Q(F))\circ \wt{\phi}=\p_{\Idx'(H(T))}\circ \p_{\ft(\Idx'(H(T)))}\circ \Q(F')$$
%
Using the defining equations for $\psi$ we rewrite the left hand side as follows:
%
$$\psi_{\Idx(T)}\circ \Phi(\p_{\Idx(T)})\circ \Phi(\p_{\ft(\Idx(T))})\circ
\Phi(\Q(F))\circ \wt{\phi}=\p_{H(\Idx(T))}\circ \p_{\ft(H(\Idx(T)))}\circ
\psi_{T}\circ \Phi(\Q(F))\circ \wt{\phi}.$$
%
It remains to show that
%
\begin{eq}
  \Q(\psi_{\Gamma}\circ \Phi(F)\circ \phi)=\psi_{(\Gamma,F)}\circ \Phi(\Q(F))\circ \wt{\phi},  \label{eqn13}
\end{eq}%
which is the defining equation of $\psi_{(\Gamma,F)}$.

We prove \ref{2015.05.10.eq2ab} as follows.
%
\begin{align*}
  \psi_{\Idx(T)}\circ \Phi&(\Q(F)_E)\circ \Phi(\p_{(\wt{\U};p),Eq})\circ \Phi(\Q(p))\circ \wt{\phi} \\
    & = \psi_{\Idx(T)} \circ \Phi(\Q(F)_E \circ \p_{(\wt{\U};p),Eq})\circ \Phi(\Q(p))\circ \wt{\phi}    & \by{functoriality of $\Phi$} \\
    & = \psi_{\Idx(T)} \circ \Phi(\p_{\Idx(T)}\circ \Q(\Q(F),p))\circ \Phi(\Q(p))\circ\wt{\phi}         & \by{commutativity of \ref{pEU-diagram}} \\
    & = \psi_{\Idx(T)} \circ \Phi(\p_{\Idx(T)}) \circ \Phi(\Q(\Q(F),p) \circ \Q(p))\circ \wt\phi        & \by{functoriality of $\Phi$} \\
    & = \psi_{\Idx(T)} \circ \Phi(\p_{\Idx(T)}) \circ \Phi(\Q(\Q(F) \circ p)) \circ \wt\phi             & \by{commutativity of \ref{2015.04.06.l0.sq}} \\
    & = \p_{H(\Idx(T))}\circ \psi_{\ft(\Idx(T))}\circ \Phi(\Q(\Q(F) \circ p)) \circ \wt\phi             & \by{\ref{psi-p-compat}} \\
    & = \p_{H(\Idx(T))}\circ \psi_{((\Gamma,F),\Q(F)\circ p)}\circ\Phi(\Q(\Q(F)\circ p))\circ\wt{\phi}  & \by{\cite[Lemma 3.2]{fromunivwithPi}} \\
    & = \p_{H(\Idx(T))}\circ \Q'(\psi_{(\Gamma,F)} \circ \Phi(Q(F)\circ p) \circ \phi)                  & \by{\ref{psi-Q-compat}} \\
    & = \p_{H(\Idx(T))}\circ \Q'(u_1(H(\ft(\Idx(T)))))                                                  & \by{\ref{u1-H-eqn}} \\
    & = \p_{H(\Idx(T))}\circ \Q'(u_1(\ft(H(\Idx(T)))))                                                  & \by{\ref{ft-H-eqn}} \\
    & = \p_{\Idx'(H(T))}\circ \Q'(u_1(\ft(\Idx'(H(T)))))                                                & \by{\ref{HIdxT-eqn}} \\
    & = \p_{\Idx'(H(T))}\circ \Q'(\Q'(u_1(H(T)))\circ p')                                               & \by{\ref{2015.03.27.l1-IDx}} \\
    & = \p_{\Idx'(H(T))}\circ \Q'(\Q'(u_1(H(T))),p')\circ \Q'(p')                                       & \by{commutativity of \ref{2015.04.06.l0.sq}} \\
    & = \Q(u_1(H(T)))_{E'}\circ \p_{(\wt{\U}';p'),Eq'}\circ \Q'(p')                                     & \by{commutativity of \ref{2015.05.08.constr1}} \\
    & = \Q(\psi_{\Gamma}\circ \Phi(F)\circ \phi)_{E'}\circ \p_{(\wt{\U}';p'),Eq'}\circ \Q'(p')          & \by{\ref{u1-H-eqn}} 
\end{align*}
That finishes the proof of \ref{2015.05.06.l3}.
\end{myproof}

\begin{lemma}
\label{2015.04.12.l3} Let ${\bf\Phi}$ be a universe category functor as above
that is compatible with the (J0,J1,J2)-structures $(Eq,\Omega,Jp)$ and
$(Eq',\Omega',Jp)$ on $p$ and $p'$ respectively. Then the homomorphism of
C-systems $H=H({\bf\Phi})$ is a homomorphism of C-systems with
(J0,J1,J2)-structures relative to $(\Id_{Eq},\refl_{\Omega},\J_{Jp})$ and
$(\Id_{Eq'},\refl_{\Omega'},\J_{Jp'})$.
\end{lemma}
%
\begin{myproof}
Let $\Id=\Id_{\Omega}$, $\Id'=\Id_{\Omega'}$, $\refl=\refl_{\Omega}$,
$\refl'=\refl_{\Omega'}$, $\J=\J_{Jp}$ and $\J'=\J_{Jp'}$.

We need to verify that for all $\Gamma\in Ob(\toCC({\C},p))$, $T\in
Ob_1(\Gamma)$, $P\in Ob_1(\Idx(T))$ and $s0\in \Obwt(\rf^*_T(P))$ one has
%
$$H(\J(\Gamma,T,P,s0))=\J'(H(\Gamma),H(T),H(P),H(s0))$$
%

The defining equation for $\J'$ is
%
\begin{eq}
  \label{def-eqn-J'}
  \etaunshriek_{pE\wt{\U}'}(u'_1(H(T)),\wt{u}'_{1,{\Idx'(H(T)}}(\J')))=\phi(H(\Gamma),H(T),H(P),H(s0)) \circ Jp' 
\end{eq}
and to prove the lemma we need to show that $H(\J)$ satisfies this equation.

Using Lemma \ref{2015.05.06.l3} we have
%
\begin{align*}
  \etaunshriek_{pE\wt{\U}}(&u'_1(H(T)),\wt{u}'_{1,{\Idx'(H(T)}}(H(\J)))) \\
    & = \etaunshriek_{pE\wt{\U}'}(\D_{pE\wt{\U}'}(\psi_{\Gamma},\_)(\D_{pE\wt{\U}'}(\_,\wt{\phi})(\Phi_E^2(u_1(T),\wt{u}_{1,\Idx(T)}(\J))))) \\
    & = \psi_{\Gamma}\circ \etaunshriek_{pE\wt{\U}'}(\Phi_E^2(u_1(T),\wt{u}_{1,\Idx(T)}(\J)))\circ \I_{pE\wt{\U}'}(\wt{\phi})
\end{align*}
%
We further have:
%
\begin{align*}
  \psi_{\Gamma}\circ \etaunshriek_{pE\wt{\U}'}&(\Phi_E^2(u_1(T),\wt{u}_{1,\Idx(T)}(\J)))\circ \I_{pE\wt{\U}'}(\wt{\phi}) \\
    & = \psi_{\Gamma}\circ \Phi(\etaunshriek_{pE\wt{\U}}(u_1(T),\wt{u}_{1,\Idx(T)}(\J)))\circ \chi_{{\bf\Phi}_E}(\wt{\U})\circ \I_{pE\wt{\U}'}(\wt{\phi}) & \by{\cite[Lemma 5.8]{fromunivwithPi}} \\
    & = \psi_{\Gamma}\circ \Phi(\phi(\Gamma,T,P,s0)\circ Jp)\circ \chi_{{\bf\Phi}_E}(\wt{\U})\circ \I_{pE\wt{\U}'}(\wt{\phi})                             & \by{\ref{2015.05.08.rem1}} \\
    & = \psi_{\Gamma}\circ \Phi(\phi(\Gamma,T,P,s0)\circ Jp)\circ \wt{\zeta}_{\bf\Phi}.                                                                   & \by{def'n of $\wt{\zeta}_{\bf\Phi}$ in \ref{wt-zeta-Phi-defn}}
\end{align*}
%

It remains to show that
%
$$\psi_{\Gamma}\circ \Phi(\phi(\Gamma,T,P,s0)\circ Jp)\circ
\wt{\zeta}_{\bf\Phi}=\phi(H(\Gamma),H(T),H(P),H(s0))\circ Jp'$$
%
By the compatibility condition of Definition \ref{2015.04.06.def6} we see that
it is sufficient to prove that
%
$$\psi_{\Gamma}\circ \Phi(\phi(\Gamma,T,P,s0))\circ R_{\Phi} =
\phi(H(\Gamma),H(T),H(P),H(s0))$$
%
Let
%
\begin{align*}
  \pr_1 & :(\I_{pE\wt{\U}}(\U),\I^{\omega})\times_{\I_{p}(\U)}(\I_{p}(\wt{\U}),\I_p(p))\sr \I_{pE\wt{\U}}(\U) \\
  \pr_2 & :(\I_{pE\wt{\U}}(\U),\I^{\omega})\times_{\I_{p}(\U)}(\I_{p}(\wt{\U}),\I_p(p))\sr \I_{p}(\wt{\U})
\end{align*}
%
be the projections and let $\pr_1'$, $\pr_2'$ be their analogues in $\mathcal
C'$. Then one has the following two equations.
%
\begin{align}
  R_{\Phi}\circ \pr_1' & =\Phi(\pr_1)\circ \zeta_{\Phi}      \label{RPhi-eqn1} \\
  R_{\Phi}\circ \pr_2' & =\Phi(\pr_2)\circ \wt{\xi}_{\Phi}.  \label{RPhi-eqn2}
\end{align}
%
On the other hand, the defining relations of $\phi(\Gamma,T,P,s0)$ are
%
\begin{align}
  \phi(\Gamma,T,P,s0)\circ \pr_1&=\etaunshriek_{pE\wt{\U}}(F,G) \label{phiGTPs-defn1} \\
  \phi(\Gamma,T,P,s0)\circ \pr_2&=\etaunshriek_p(F,\wt{H}),     \label{phiGTPs-defn2}
\end{align}
and the defining relations of $\phi(H(\Gamma),H(T),H(P),H(s0))$ are
%
\begin{align}
  \phi(H(\Gamma),H(T),H(P),H(s0)) \circ \pr'_1 &=\etaunshriek_{pE\wt{\U}'}(F',G')          \label{phiGTPs-defn1'} \\
  \phi(H(\Gamma),H(T),H(P),H(s0)) \circ \pr'_2 &=\etaunshriek_{p'}        (F',\wt H'),     \label{phiGTPs-defn2'}
\end{align}
%
where
%
\begin{align*}
  F&=u_{1,\Gamma}(T)           & F'&=u'_{1,H(\Gamma)}(H(T))         \\
  G&=u_{1,\Idx(T)}(P)          & G'&=u'_{1,\Idx'(H(T))}(H(P))       \\
  \wt{H}&=\wt{u}_{1,T}(s0),    & \wt H' &=\wt u'_{1,H(T)}(H(s0))
\end{align*}
%

We need to prove
%
\begin{eq}
\label{2015.05.10.eq3a} \psi_{\Gamma}\circ \Phi(\phi(\Gamma,T,P,s0))\circ
R_{\Phi}\circ \pr_1' = \phi(H(\Gamma),H(T),H(P),H(s0))\circ \pr_1'
\end{eq}%
%
and
%
%
\begin{eq}
\label{2015.05.10.eq3b} \psi_{\Gamma}\circ \Phi(\phi(\Gamma,T,P,s0))\circ
R_{\Phi}\circ \pr_2' = \phi(H(\Gamma),H(T),H(P),H(s0))\circ \pr_2'
\end{eq}%
%
We establish (\ref{2015.05.10.eq3a}) as follows.
%
\begin{align*}
        \psi_{\Gamma}\circ \Phi&(\phi(\Gamma,T,P,s0))\circ R_{\Phi}\circ \pr_1'                                         \\
    & = \psi_{\Gamma}\circ \Phi (\phi(\Gamma,T,P,s0))\circ \Phi(\pr_1)\circ \zeta_{\Phi}                            & \by{\ref{RPhi-eqn1}}         \\
    & = \psi_{\Gamma}\circ \Phi (\phi(\Gamma,T,P,s0) \circ      \pr_1)\circ \zeta_{\Phi}                            & \by{functoriality of $\Phi$} \\
    & = \psi_{\Gamma}\circ \Phi(\etaunshriek_{pE\wt{\U}}(F,G))\circ \zeta_{\Phi}                                    & \by{\ref{phiGTPs-defn1}}     \\
    & = \psi_{\Gamma}\circ \Phi(\etaunshriek_{pE\wt{\U}}(F,G))\circ \chi_{{\bf \Phi}_E}(U) \circ \I_{pE\wt{\U}'}(\phi) & \by{def'n of $\zeta_{\Phi}$ in \ref{zeta-Phi-defn}} \\
    & = \psi_{\Gamma}\circ\etaunshriek_{pE\wt{\U}'}({\bf\Phi}^2_E(F,G))\circ \I_{pE\wt{\U}'}(\phi),                 & \by{\cite[Lemma 5.8]{fromunivwithPi}} \\
    & = \etaunshriek_{pE\wt{\U}'}(\D_{pE\wt{\U}}(\psi_{\Gamma},\_)(\D_{pE\wt{\U}}(\_,\phi)({\bf\Phi}^2(F,G))))      & \by{naturality of $\etashriek_{pE\wt{\U}'}$}    \\
    & = \etaunshriek_{pE\wt{\U}'} (u_{1,H(\Gamma)}'(H(T)),u_{1,\Idx'(H(T))}     (H(P))),                            & \by{Lemma \ref{2015.05.06.l3}(1)} \\
    & = \etaunshriek_{pE\wt{\U}'} ((F',G'))                                                                         & \by{def'n of $F'$ and $G'$} \\
    & = \phi(H(\Gamma),H(T),H(P),H(s0)) \circ \pr_1'                                                                & \by{\ref{phiGTPs-defn1'}}
\end{align*}

For (\ref{2015.05.10.eq3b}), we proceed as follows.
%
\begin{align*}
  \psi_{\Gamma}\circ \Phi&(\phi(\Gamma,T,P,s0))\circ R_{\Phi}\circ \pr_2'  \\
    & = \psi_{\Gamma}\circ \Phi(\phi(\Gamma,T,P,s0))\circ \Phi(\pr_2)\circ \wt{\xi}_{\Phi}  & \by{\ref{RPhi-eqn2}} \\
    & = \psi_{\Gamma}\circ \Phi(\phi(\Gamma,T,P,s0)\circ \pr_2)\circ \wt{\xi}_{\Phi}        & \by{functoriality of $\Phi$} \\
    & = \psi_{\Gamma}\circ \Phi(\etaunshriek_p((F,\wt{H})          ))\circ \wt{\xi}_{\Phi}  & \by{\ref{phiGTPs-defn2}} \\
    & = \psi_{\Gamma}\circ \Phi(\etaunshriek_p((u_1(T),\wt{u}_1(s0)))\circ \wt{\xi}_{\Phi}  & \by{def'n of $F$ and $\wt H$} \\
    & = \psi_{\Gamma}\circ \Phi(\etaunshriek_p((u_1(\ft(\partial(s0))),\wt{u}_1(s0)))\circ \wt{\xi}_{\Phi} & \by{the type of $s0$} \\
    & = \psi_{\Gamma}\circ \Phi(\etaunshriek_p(\wt u_{2,\Gamma}(s0)))\circ \wt{\xi}_{\Phi}  & \by{def'n of $\wt u_2$ in \ref{wt-u2-defn}} \\
    & = \etaunshriek_{p'}(\wt{u}_{2,H(\Gamma)}'(H(s0)))                                     & \by{\cite[Lemma 6.2(2)]{fromunivwithPi}} \\
    & = \etaunshriek_{p'}((u'_1(\ft(\partial(H(s0)))),\wt{u}'_1(H(s0))))                    & \by{def'n of $\wt u_2$ in \ref{wt-u2-defn}} \\
    & = \etaunshriek_{p'}((u'_1(\ft(\partial(H(s0)))),\wt H'))                              & \by{def'n of $\wt H'$} \\
    & = \etaunshriek_{p'}((u'_1(\ft(H(\partial(s0)))),\wt H'))                              & \by{functoriality of $H$} \\
    & = \etaunshriek_{p'}((u'_1(H(\ft(\partial(s0)))),\wt H'))                              & \by{\ref{ft-H-eqn}} \\
    & = \etaunshriek_{p'}((u'_1(H(T))),\wt H'))                                             & \by{the type of $s0$} \\
    & = \etaunshriek_{p'}((F',\wt H'))                                                      & \by{def'n of $F'$} \\
    & = \phi(H(\Gamma),H(T),H(P),H(s0))\circ \pr_2'                                         & \by{\ref{phiGTPs-defn2'}}
\end{align*}
This finishes the proof of \ref{2015.04.12.l3}.
\end{myproof}

\section{Index of symbols}
\printglossaries

\section{Bibliography}
\bibliography{alggeom}
\bibliographystyle{plain}

\end{document}

%% Local Variables:
%% compile-command: "make From_a_univ_with_paths.pdf "
%% TeX-master: t
%% End:
